

\documentclass[addpoints,answers]{exam}

\usepackage{blkarray}
\usepackage{amsmath}
\usepackage{amssymb}    
\usepackage{amsfonts}
\usepackage{tikz}
\usepackage{verbatim}

\sloppy

\usetikzlibrary{arrows,shapes}

\tikzstyle{vertex}=[circle,fill=black,minimum size=3pt,inner sep=0pt]
\tikzstyle{edge} = [draw,thick,-]

\checkboxchar{$\Box$}
\checkedchar{$\blacksquare$}
\CorrectChoiceEmphasis{}


\begin{document}

    \pagestyle{headandfoot}
    \runningheadrule
    \firstpageheader{Math 152}{Homework 1}{April 6, 2018}
    \runningheader{Math 152}{Homework 1, Page \thepage\ of \numpages}{April 6, 2018}

    \firstpagefooter{}{}{}
    \runningfooter{}{}{}
    \begin{flushright}
        \makebox[0.4\textwidth]{Name: Huize Shi}

        \vspace{0.2in}

        \makebox[0.4\textwidth]{Pid: A92122910}
    \end{flushright}

    \begin{questions}
        \question
            Alice and Bob play the following game.
            \begin{itemize}
                \item Initially, there are $20$ numbers: $10$ numbers $1$ and $10$ numbers
                    $2$.
                \item On each step one of the players select two numbers; and if they were
                    the same, replace them by $2$; otherwise, replace them by $1$.
                \item Alice make the first move and they do moves one after another.
            \end{itemize}
            Who is the winner?
            \begin{solutionorbox}[\stretch{1}]
            \end{solutionorbox}
            \newpage
 
        \question
            In the subtraction game where players may subtract $1$, $2$ or $5$ chips on
            their turn, identify the N and P positions.
            \begin{solutionorbox}[\stretch{1}]
            \end{solutionorbox}
            \newpage
 
        \question
            Is the Nim position $(1, 3, 5)$ an N-position (explain your answer)?
            \begin{solutionorbox}[\stretch{1}]
            \end{solutionorbox}
            \newpage
 
        \question
            Consider the Mis\'ere subtraction game where players may subtract 1, 5 or 6
            chips on their turn, identify the N and P positions.
            \begin{solutionorbox}[\stretch{1}]
            \end{solutionorbox}
            \newpage
 
\end{questions}
\end{document}
