\documentclass{article}
\usepackage{listings}
\usepackage{mathtools, amssymb, amsmath}
\usepackage{tikz}
\usepackage{enumitem}
\usepackage[margin=1in]{geometry}
\lstset {
	tabsize=2
}

\title{Homework 6: Continuations}
\author{Jack Shi - A92122910 \\
				Wyatt Guidry - A12994977 \\
				Austin Coleman - A12888539}

\begin{document}
\maketitle

\section{Why CPS?}
\begin{enumerate}
	% 1
	\item \textbf{Answer:}
	\begin{lstlisting}
		fun div(x, y, cc) {
			cc(x / y)
		}

		fun f(x, y, cc) {
			div(2, y, r => cc(3 * y * r))
		}
		div(3, 4, r => f(r, 5, top_cc))
	\end{lstlisting}

	% 2.
	\item $\beta,\ \eta$ optimization\\
		For part \textbf{a} and \textbf{b}, first transform the functions into
		$\lambda$-calculus form.
		\begin{align*}
		div &\equiv \lambda x.\lambda y. \lambda cc.(cc\ (/\ x\ y))\\
		f &\equiv \lambda x.\lambda y. \lambda cc.((div)\ 2\ y\ \lambda r.((cc)\
		(3*y*r)))\\
		(div)\ &3\ 4\ \lambda r.((f)\ r\ 5\ top\_cc)\\
		(\lambda x.\lambda y. \lambda cc.(cc\ (/\ x\ y)))\ &3\ 4\ \lambda r.((f)\ r\ 5\
		top\_cc)\\
		\end{align*}
		\begin{enumerate}
			% a.
			\item \textbf{Answer:} For the sake of clarity and readability, I did not
				rewrite multiplication as prefix function application.

				First simplify $f$:
			\begin{align*}
			f &\equiv\  \lambda x.\lambda y. \lambda cc.((div)\ 2\ y\ \lambda r.((cc)\
			(3*y*r)))\\
			&=\ \ \lambda x.\lambda y. \lambda cc.((\lambda x.\lambda y. 
			\lambda cc.(cc\ (/\ x\ y)))\ 2\ y\ \lambda r.((cc)\ (3*y*r)))\\
			&=_{\beta}\ \lambda x.\lambda y. \lambda cc.(\lambda r.((cc)\ (3*y*r))\ (/\ 2\ y))\\
			&=_{\beta}\ \lambda x.\lambda y. \lambda cc.(\lambda r.((cc)\ (3*y*r))\ (/\ 2\ y))\\
			&=_{\beta}\ \lambda x.\lambda y. \lambda cc.((cc)\ (3*y*(/\ 2\ y)))
			\end{align*}
			Subsitute \texttt{f} in to the original equation.
			\begin{align*}
			&\ \ (\lambda x.\lambda y. \lambda cc.(cc\ (/\ x\ y)))\ 3\ 4\ \lambda r.((f)\
			r\ 5\ top\_cc)\\
			=&\ \ (\lambda x.\lambda y. \lambda cc.(cc\ (/\ x\ y)))\ 3\ 4\ \lambda
			r.((\lambda x.\lambda y. \lambda cc.((cc)\ (3*y*(/\ 2\ y))))\ r\ 5\ top\_cc)\\
			=&_{\beta}\ (\lambda x.\lambda y. \lambda cc.(cc\ (/\ x\ y)))\ 3\ 4\ \lambda
			r.((top\_cc)\ (3*5*(/\ 2\ 5)))\\
			=&_{\beta}\ (\lambda r.((top\_cc)\ (3*5*(/\ 2\ 5))))\ (/\ 3\ 4)\\
			=&_{\beta}\ (top\_cc)\ (3*5*(/\ 2\ 5))\\
			=&\ \ (top\_cc)\ (6)\\
			\end{align*}

			% b.
			\item \textbf{Answer:} For the sake of clarity and readability, I did not
				rewrite multiplication as prefix function application.

				Simplify $f$ using $\eta$ reduction:
			\begin{align*}
			f &\equiv\  \lambda x.\lambda y. \lambda cc.((div)\ 2\ y\ \lambda r.((cc)\
			(3*y*r)))\\
			&=\ \ \lambda x.\lambda y. \lambda cc.((\lambda x.\lambda y. 
			\lambda cc.(cc\ (/\ x\ y)))\ 2\ y\ \lambda r.((cc)\ (3*y*r)))\\
			&=_{\eta}\ \lambda x.\lambda y. \lambda cc.((\lambda x. \lambda cc.(cc\
			(/\ x\ y)))\ 2\ \lambda r.((cc)\ (3*y*r)))\\
			&=_{\beta}\ \lambda x.\lambda y. \lambda cc.(\lambda r.((cc)\ (3*y*r))\
			(/\ 2\ y))\\
			&=_{\beta}\ \lambda x.\lambda y. \lambda cc.((cc)\ (3*y*(/\ 2\ y)))
			\end{align*}

			By $\eta$ reduction, \texttt{f} converged to the same result. The rest of
			the reduction follows the same as part \textbf{a.}.

			\begin{align*}
			&\ \ (\lambda x.\lambda y. \lambda cc.(cc\ (/\ x\ y)))\ 3\ 4\ \lambda r.((f)\
			r\ 5\ top\_cc)\\
			=&\ \ (\lambda x.\lambda y. \lambda cc.(cc\ (/\ x\ y)))\ 3\ 4\ \lambda
			r.((\lambda x.\lambda y. \lambda cc.((cc)\ (3*y*(/\ 2\ y))))\ r\ 5\ top\_cc)\\
			=&_{\beta}\ (\lambda x.\lambda y. \lambda cc.(cc\ (/\ x\ y)))\ 3\ 4\ \lambda
			r.((top\_cc)\ (3*5*(/\ 2\ 5)))\\
			=&_{\beta}\ (\lambda r.((top\_cc)\ (3*5*(/\ 2\ 5))))\ (/\ 3\ 4)\\
			=&_{\beta}\ (top\_cc)\ (3*5*(/\ 2\ 5))\\
			=&\ \ (top\_cc)\ (6)\\
			\end{align*}

		\end{enumerate}

	% 3
	\item Error throwing and catching
	\begin{enumerate}
		% a
		\item \textbf{Answer:}
		\begin{lstlisting}
			fun div(x, y, cc_ok, cc_fail) {
				if (y != 0) {
					cc_ok(x / y)
				} else {
					cc_fail(``Divide by zero'')
				}
			}

			fun f(x, y, cc) {
				div(2, y, r => cc(3 * y * r), r => 0)
			}
			
			div(3, 4, r => f(r, 5, top_cc_ok), top_cc_fail)
		\end{lstlisting}

		% b
		\item \textbf{Answer:} In case of \texttt{Node.js}, when the call back begin
			to execute, the high level \texttt{try/catch} block is not responsible for
			handling the callback errors. The issue is similar here. If throw is used,
			then the high level \texttt{try/catch} might not catch if the function
			that throw an error is used as an callback for some function in the
			\texttt{try/catch} block.

	\end{enumerate}
\end{enumerate}

\section{Tail Recursion and Continuations}
\begin{enumerate}
	% 1
	\item \textbf{Answer: }With compiler optimizations, since the recursive call
		is the last instruciton of the current function, there is no need to keep
		the previous stack frame. This allows tail recursive calls to all be
		executed in a single stack space which means that the space requirement is
		independent from n which governs the number of recursive calls.
	% 2
	\item \textbf{Answer: }This program require a stack for \texttt{fact}, a stack
		for tail recursive \texttt{f} call and n stacks for \texttt{g}. The stack
		for \texttt{fact} can be freed when the function is done. The stack for
		\texttt{f} is reused because of tail recursive optimization. The memory is
		freed right before \texttt{fact} stack is freed. The n \texttt{g} stacks are
		spawned when base case is hit and destroyed when the value is calculated for
		the base case.
\end{enumerate}

\end{document}

