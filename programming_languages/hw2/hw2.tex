\documentclass{article}
\usepackage{amsmath}

\begin{document}
\section{Reduction and variable capture}

\begin{enumerate}
	% Question 1
	\item Answer:
		\subparagraph{Non capture avoiding}
			\begin{align*}
				&(\lambda x.(\lambda y.x))\ y &\\
				&=_{\beta} \lambda y.y &
			\end{align*}
		\subparagraph{Capture avoiding}
			\begin{align*}
				&(\lambda x.(\lambda y.x))\ y &\\
				&=_{\alpha} (\lambda x.(\lambda z.x))\ y &\\
				&=_{\beta} \lambda z.y &
			\end{align*}

	% Question 2
	\item Answer: 
		\subparagraph{Non capture avoiding}
			\begin{align*}
				&(\lambda x.(\lambda y.x))\ (\lambda y.x) &\\
				&=_{\beta} \lambda y.(\lambda y.x)
			\end{align*}
		\subparagraph{Capture avoiding}
			\begin{align*}
				&(\lambda x.(\lambda y.x))\ (\lambda y.x) &\\
				&=_{\alpha} (\lambda x.(\lambda z.x))\ (\lambda y.x) &\\
				&=_{\beta} \lambda zy.x
			\end{align*}

	% Question 3
	\item Answer: 
		\subparagraph{Non capture avoiding}
			\begin{align*}
				&(\lambda x.(\lambda y.x))\ (\lambda y.y) &\\
				&=_{\beta} \lambda y.(\lambda y.y)
			\end{align*}
		\subparagraph{Capture avoiding}
			\begin{align*}
				&(\lambda x.(\lambda y.x))\ (\lambda y.y) &\\
				&=_{\alpha} (\lambda x.(\lambda z.x))\ (\lambda y.y) &\\
				&=_{\beta} \lambda zy.y
			\end{align*}

	% Question 4
	\item Answer: 
		\subparagraph{Non capture avoiding}
			\begin{align*}
				&(\lambda xyz.\ \lambda fgh.\ f\ x\ (g\ y)\ (h\ z))\ h\ (\lambda ab.\ a\
				(g\ b))\ f&\\
				&=_{\beta} (\lambda yz.\ \lambda fgh.\ f\ h\ (g\ y)\ (h\ z))\ (\lambda ab.\ a\
				(g\ b))\ f&\\
				&=_{\beta} (\lambda z.\ \lambda fgh.\ f\ h\ (g\ (\lambda ab.\ a\ (g\
				b)))\ (h\ z))\ f&\\
				&=_{\beta} \lambda fgh.\ f\ h\ (g\ (\lambda ab.\ a\ (g\ b)))\ (h\ f)&\\
			\end{align*}
		\subparagraph{Capture avoiding}
			\begin{align*}
				&(\lambda xyz.\ \lambda fgh.\ f\ x\ (g\ y)\ (h\ z))\ h\ (\lambda ab.\ a\
				(g\ b))\ f&\\
				&=_{\alpha} (\lambda xyz.\ \lambda qgr.\ (q\ x)\ (g\ y)\ (r\ z))\ h\ (\lambda ab.\ a\
				(g\ b))\ f&\\
				&=_{\beta} (\lambda yz.\ \lambda qgr.\ (q\ h)\ (g\ y)\ (r\ z))\ (\lambda ab.\ a\
				(g\ b))\ f&\\
				&=_{\beta} (\lambda z.\ \lambda qgr.\ (q\ h)\ (g\ (\lambda ab.\ a\ (g\
				b)))\ (r\ z))\ f&\\
				&=_{\beta} \lambda qgr.\ (q\ h)\ (g\ (\lambda ab.\ a\ (g\
				b)))\ (r\ f)&
			\end{align*}

	% Question 5
	\item Answer: 
		\begin{align*}
			&(\lambda xy.z)\ z\ z&\\
			&=  (\lambda xy.z)\ (z\ z)&\\
			&=  \lambda y.z &
		\end{align*}

	% Question 6
	\item Answer: 
		\begin{align*}
			&(\lambda x.(\lambda y.(x\ y)))&\\
			&=_{\eta}  \lambda x.x&\\
		\end{align*}

	% Question 7
	\item Answer: 
		\begin{align*}
			S &= \lambda xyz.((x\ z)\ (y\ z))&\\
			K &= \lambda xy.x &\\
			I &= \lambda x.x&\\
		\end{align*}
		S K S = (S K) S by expression left association. The following are done with
		call-by-name.
		\begin{align*}
			S\ K &= (\lambda xyz.((x\ z)\ (y\ z))) (\lambda xy.x) &\\
					 &=_{\beta} (\lambda yz.(((\lambda xy.x)\ z)\ (y\ z)))&\\
					 &=_{\beta} (\lambda yz.((\lambda y.z)\ (y\ z)))&\\
					 &=_{\beta} \lambda yz.z&\\
			(S\ K)\ S &= (\lambda yz.z)\ (\lambda xyz.((x\ z)\ (y\ z)))
					 &={\beta} (\lambda z.z)
					 &={\alpha} (\lambda x.x) = I
		\end{align*}
		S K S hence shown reduces to I.
\end{enumerate}

\section{Variable Binding and Closure}
\begin{enumerate}
	% Question 1:
	\item Answer: The function is recursive and never terminates. This is because
		the x(2) call on line 2 refers to the definition of x on line 2 which makes
		it recursive.
	% Question 2:
	\item Answer: x passed in as argument on line two binds to the previous
		definition on line 1 because at the time of reference, x on line 2 is not
		assigned yet since the function is not executed at that point.
	% Question 3:
	\item Answer: You get the following error: Error reassigning x. This is
		because Haskel use static types. Variables are not allowed to change values.
\end{enumerate}

\section{Closure and access links}
	\begin{enumerate}
		% question 1
		\item See scaned document

		% question 2
		\item 18

		% question 3
		\item This does not terminate because the function is infinitely recursive
			without a base case.
	\end{enumerate}
\section{Memory management and high-order functions}
	\begin{enumerate}
		% question 1
		\item See scaned document

		% question 2
		\item The value is 10 because. Within function scope g, x is set to 7. This
			is passed in as argument to h which is f. Therefore the y argument in f
			resolves to 7. x is 5 by access link to the parent scope. The function f
			therefore returns 5 + 7 - 2 which is 10.

	\end{enumerate}

\section{More subsitution and variable capture}
\begin{enumerate}

	% Problem 1
	\item Answer: 
		\begin{align*}
			&((\lambda x.((\lambda x.x)\ 2) + x)\ x)[x := 3]&\\
			&=_{\beta}(((\lambda x.x)\ 2) + x)[x := 3]&\\
			&=_{\beta}(2 + x)[x := 3]&\\
			&= 2 + 3&\\
			&= 5&
		\end{align*}
		Explaination: In this case it would not matter because the only x that is
		not captured is one that is ok to replace. All other x are captured by
		lambda expressions. 3 can not possibly be captured by any lambda.

	% Problem 2
	\item Answer: 
		\begin{align*}
			&(\lambda y.(\lambda xyz.z)\ y\ z\ y)[z := w]&\\
			&=(\lambda y.((((\lambda xyz.z)\ y)\ z)\ y))[z := w]&\\
			&=_{\beta} (\lambda y.(((\lambda yz.z)\ y)\ z))[z := w]&\\
			&=_{\beta} (\lambda y.((\lambda z.z)\ y))[z := w]&\\
			&=_{\beta} (\lambda y.y)[z := w]&\\
			&= \lambda y.y&
		\end{align*}
		Explaination: In this case it would not matter whether capture avoiding is
		taken into account. w is not caputured anywhere, therefore subsitution is
		legal in this case without needing to avoid capturing w.

	% Problem 3
	\item Answer:
		\begin{align*}
			&(\lambda p.(\lambda x.p(x\ x)) (\lambda x.p))[x:=p]&\\
			&=_{\alpha} (\lambda z.(\lambda x.(z(x\ x))) (\lambda x.z))[x:=p]&\\
			&=_{\beta} (\lambda x.((\lambda x.z)(x\ x))) [x:=p]&\\
			&=_{\beta} (\lambda x.z) [x:=p]&\\
			&=_{\beta} (\lambda x.z)&
		\end{align*}

		Explaination: Yes, capture avoiding is a must because p is captured by the
		lambda expression. This violates subsitution rules therefore capture
		avoiding (alpha renaming) must be performed.
\end{enumerate}

\end{document}

