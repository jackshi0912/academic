\documentclass{article}
\usepackage[margin=1in]{geometry}
\usepackage{listings}
\usepackage{enumitem}
\usepackage{amsmath}
\lstset{tabsize=2}

\usepackage{color}
\definecolor{lightgray}{rgb}{.6,.6,.6}
\definecolor{purple}{rgb}{0.65, 0.12, 0.82}

\lstdefinelanguage{JavaScript}{
  keywords={typeof, const, new, true, false, catch, function, return, null, catch, 
						switch, var, if, in, while, do, else, case, break},
  keywordstyle=\color{blue}\bfseries,
  ndkeywords={class, export, boolean, throw, implements, import, this},
  ndkeywordstyle=\color{lightgray}\bfseries,
  identifierstyle=\color{black},
  sensitive=false,
  comment=[l]{//},
  morecomment=[s]{/*}{*/},
  commentstyle=\color{purple}\ttfamily,
  stringstyle=\color{red}\ttfamily,
  morestring=[b]',
  morestring=[b]"
}

\lstset{
   language=JavaScript,
   extendedchars=true,
   basicstyle=\ttfamily,
   showstringspaces=false,
   showspaces=false,
   numbers=left,
	 xleftmargin=3em,
	 numberstyle=\color{lightgray},
   tabsize=2,
   breaklines=true,
   showtabs=false,
   captionpos=b
}

\title{
	Homework 1: Javascript and $\lambda$ calculus \\
	\large CSE 130: Programminig Languages
}
\date{Jan 23, 2018}
\author{
	Jack Shi - A92122910 \\
	Wyatt Guidry - A12994977\\
	Austin Coleman - A12888539 }

\begin{document}
\maketitle

\section*{Notation}
For this problem set we may use (and you may find it helpful to use) a less
verbose syntax for λ-calculus
programs. Specifically, instead of writing λx.(λy.(λz.e)), we may drop the
parentheses and write this term as
λx.λy. . . . λz.e. In general, the body of a λ abstraction extends as far right
as possible. To further shorten
notation, we may also drop the $\lambda$’s in our example term and write it as λxyz.e
when the variable names are
obvious and not ambiguous.
\par Similarly, instead of writing (e 1 e 2 ) e 3 we’ll simply write e 1 e 2 e 3 . We
can drop parentheses in function
applications since applications in $\lambda$-calculus are always assumed to be
left-associative.

\section{JavaScript hoisting and block-scoping [12pts]}
Recall that JavaScript variable declarations using var can have one of two
scopes: function scope or global
scope. Variables declared inside a function have function scope while variables
declared outside of functions
have global scope. Regardless of where variables are actually declared and
initialized, their declarations (but
not initializations) are hoisted to the top of the nearest (function or global)
scope; vars are not block scoped.
This is different from let and const variables which are block scoped, as in
most languages you might have
encountered.
\par In this problem, you will implement block scoping for var declarations using
first-class functions. In each
part, you will be asked what the result of running a snippet of code is, and—to
receive full credit \\

\begin{enumerate}[label=\textbf{\arabic*}.]
	\item {[1pts]} What does f(x) return? What values of x and y are used in the computation and why?
		\begin{lstlisting}
		var x = 5;
		function f(y) {return x+y}
		f(x);
		\end{lstlisting}

		\subparagraph{Answer: }
			\begin{quote}
			The function returns 10. x used in computation is resolved to the global
			variable x which is 5. x is passed in as parameter y, which points to the
			same global variable, resolves to 5.
			\end{quote}


	\item {[1pts]} What does f(x) return? What values of x and y are used
		in the computation and why?
		\begin{lstlisting}
		var x = 5;
		function f(y) {return x+y}
		if (true) {
			var x = 10;
		}
		f(x);
		\end{lstlisting}

		\subparagraph{Answer: }
			\begin{quote}
			The function returns 20. x used in computation is resolved to 10 because
			at the time of calling f, x has already been reasigned to value 10. x is
			passed in as parameter y, which resolves to the global variable x that is
			reassigned to 10 before it is referenced as a parameter y. y therefore
			resolves to 10 during computation.
			\end{quote}


	\item {[1pts]} What does f(z) return? What values of x and y are used in
		the computation and why?What does f(x) return? What values of x and y are used
		in the computation and why?

		\begin{lstlisting}
		var x = 5;
		function f(y) { return x + y; }
		if (true) {
			var z = 20;
		}
		f(z);
		\end{lstlisting}

		\subparagraph{Answer: }
			\begin{quote}
			The function f(z) returns 25. Since x is not captured by the
			fucntion, it evaluates to x in the global scope which is 5. z is passed in
			as parameter y. At the time of using z as a parameter y, z is declared
			and assigned value 20. y therefore resolves to 20 during computation.
			\end{quote}


	\item {[2pts]} What does f(x) return? What values of x and y are used
		in the computation and why? Explain how this behavior differs from that of 
		part (1) above.

		\begin{lstlisting}
		(function () {
			var x = 5;
		})();
		function f(y) { return x + y; }
		f(x);
		\end{lstlisting}

		\subparagraph{Answer: }
			\begin{quote}
			f(x) causes an ReferenceError because x is not defined. The x in
			computation is undefined because there is no variable x within any scope
			that is visible to the function f. Since an undefined variable is passed
			in as parameter y, y is therefore also undefined.
			\par In \textbf{part 1}, var x is global scope, 
			therefore references to x correctly resolves
			to 5.
			\end{quote}


	\item{[2pts]} What does f(x) return? What values of x and y are used
		in the computation and why? Explain how this behavior differs from that of part 
		(2) above.

		\begin{lstlisting}
		var x = 5;
		function f(y) { return x + y; }
		if(true) {
			(function( {
				var x = 10;
			}) ();
		}
		f(x);
		\end{lstlisting}

		\subparagraph{Answer: }
			\begin{quote}
			f(x) returns 10. The variable x in computation resolves to 5 because the 
			declaration that var x = 10 is in the a function scope that is not visible
			to f. The x that is passed in as y is also the global variable x.
			Therefore y in computation is resolved to global var x which is 5.
			\par In \textbf{part 2}, var x = 10 is not wrapped in some function. 
			It is a global scope expression therefore var x=10 is visible to funciton f.
			\end{quote}


	\item{[2pts]} What does f(z) return? What values of x and y are used
		in the computation and why? Explain how this behavior differs from that of 
		part (3) above.

		\begin{lstlisting}
		var x = 5;
		function f(y) { return x + y; }
		if(true) {
			(function () {
				var z = 20;
			}) ();
		}
		f(z);
		\end{lstlisting}

		\subparagraph{Answer: }
			\begin{quote}
			f(z) causes an ReferenceError because z is not defined in a scope visible
			to the caller. The value x is defined in global scope and is resolved to 5. 
			z is passed into f as parameter y. Since var z is declared within a anonymous 
			function where the variable is only visible within the function scope, 
			y is therefore undefined because the caller is unable to resolve variable
			z.
			\par This is different from \textbf{part 3} because in \textbf{part 3}, 
			z is in global scope therefore f(z) works because z is visible from the scope 
			where it is referenced.
			\end{quote}

	\item {[3pts]} Rene thinks that Joe’s compiler may have been a bit too
		clever, leading it to behave incorrectly
		in some cases. Is she right? If so, explain where Joe’s compiler got it wrong
		above and describe a
		general wrapping algorithm that Joe should have used instead.

		\subparagraph{Answer: }
			\begin{quote}
			Keep a list of tuples that represents the edges that connects expressions.
			Go through the code and do the following:
			\begin{enumerate}[label=\textbf{\arabic*}.]
				\item Whenever an expression that references a variable is encountered,
					create an edge between this expression and a declaration of the
					variable happened earlier within the same scope.
				\item Whenever an variable declaration/assignment is encountered,
					create an edge between this declaration/assignment and any earlier
					declaration/assignment in the same scope if it exists.
			\end{enumerate}
			Once the list of edges has been obtained, simply running DFS would give
			you all the disjointed components that need to be captured within the same
			scope. For each component, find the expressions that happened ealiest and
			lastest. Go through these boundaries for each component, merge them if
			they overlapse. Use these new boundaries as guidlines and add function
			scopping around each to mimic block scopping properly.
			\end{quote}

\end{enumerate}

\section{$\lambda$-calculus $\Leftrightarrow$ JavaScript [14pts]}
To get you more comfortable with $\lambda$-calculus syntax, in this problem, we’ll be
converting λ-calculus expressions
to JavaScript expressions and back. At first, you may find it helpful to think
about λ expressions in terms
of their JavaScript counterparts; but once your comfortable with the λ-calculus
syntax you may find the
converse to be true!
\par First, we’ll convert some λ terms to JavaScript. You may use JavaScript arrow
functions, but don’t have
to.

\begin{enumerate}[label=\textbf{\arabic*}.]
	\item Convert $\lambda xfgh.(f\ x)\ (g\ x)\ (h\ x)$ to the equivalent JavaScript
	expression.
	\subparagraph{Answer: }
		$\lambda xfgh.(f\ x)\ (g\ x)\ (h\ x) \equiv \lambda xfgh.(((f\ x)\ (g\ x))\ (h\ x))$
		\begin{lstlisting}
		(x, f, g, h) => ((f(x)) (g(x))) (h(x))
		\end{lstlisting}

	\item Convert the following term to the equivalent JavaScript expression.
		\begin{lstlisting}[escapeinside={(*}{*)}]
		(*$\lambda a_1a_2c_1c_2$*)		(*\textbf{if}*) (*$a_1 + a_2 < 10$*)
						   (*\textbf{then}*) (*$c_1, a_2, a_2$*)
						   (*\textbf{else}*) (*$c_2, a_2, a_1$*)
		\end{lstlisting}

		\subparagraph{Answer: } $\quad$
		\begin{lstlisting}[escapeinside={(*}{*)}]
		(a1,a2,c1,c2) => (a1 + a2 > 10) ? [c1,a2,a2] : [c2,a2,a1];
		\end{lstlisting}
	
	\item Convert $(\lambda x.y)((\lambda x.x\ x)(\lambda y.y\ z))$ to the
		equivalent JavaScript expression.
		\subparagraph{Answer: } $\quad$
		\begin{lstlisting}
		(x=>y)((x=>x(x))((y=>y)(z))
		\end{lstlisting}

	\item Write the JavaScript program f(g(3)) and helper function f and g as a
		single $\lambda$-calculus expression.
		\begin{lstlisting}
		const f = (x) => x*2;
		const g = function (y) { return y-1; }
		f(g(3))
		\end{lstlisting}

		\subparagraph{Answer: } $\lambda x.(x*2) (\lambda y.(y-1) (3))$

\end{enumerate}

\section {$\lambda$-calculus and macro processors [10pts]}
Macro processors, such as cpp and m4, do a form of program manipulation before
the program is further
compiled. You can think of this as symbolic program “pre-evaluation.” For
example, if a program contains
the macro
\begin{lstlisting}
#define square(x) ((x)*(x))
\end{lstlisting}
macro expansion of a statement containing square(y+3) will replace square(y+3) 
with (y+3)*(y+3). One way to build a macro processor is to associate a lambda 
expression with each macro definition, and then do $\beta$-reduction wherever 
the macro is used. For example, we can represent square as the lambda
expression $\lambda x.(x * x)$ and $\beta$-reduce:
$$\lambda x.(x*x))(y+3)=_\beta (y+3)*(y+3)$$
f square(y+3) appears in the program. Some problems may arise, however, when
variables that appear in
macros also appear in the program.


\begin{enumerate}[label=\textbf{\arabic*}.]
	\item Suppose a program contains three macros:
		\begin{lstlisting}
		#define f(x) (x+x)
		#define g(y) (y-2)
		#define h(z) (f(g(z)))
		\end{lstlisting}
		Macro h can be written as the following lambda expression:
	$$\lambda z.(\lambda x.x+x)((\lambda y.y-2) z) \equiv \lambda z.f(g\ z)$$
	Simplify the expression h(3) by using $\beta$-reduction. Do not simplify the
	arithmetic. Only reduce one step at a time. Use as many or as few lines as you need.
	\begin{align*}
		&(\lambda z(\lambda x.x + x)((\lambda y.y - 2)z))3&\\
		=&(\lambda x.x+x)((\lambda y.y-2)3)&\\
		=&(\lambda x.x+x)(3-2)&\\
		=&(3-2)+(3-2)&\\
	\end{align*}

	\item A problem arises if a local variable inside a macro has the same name as
		an argument the macro is
		invoked with. Assuming that typeof is supported, the following macro works
		as long as neither of the
		parameters is named temp.
		\begin{lstlisting}
		#define swap(a,b) { typeof(a) temp = a; a = b; b = temp; }
		\end{lstlisting}
		Show what happens if you macro-expand swap(x, temp) without doing anything
		special to recognize
		that temp is bound in the body of the macro. You do not need to convert this
		macro to lambda notation.

		\subparagraph{Answer: } 
		\begin{lstlisting}
		swap(x,temp) = { typeof(x) temp = x; x = temp; temp = temp; }
		\end{lstlisting}

	\item Explain what the problem is and how you can solve it.
		\subparagraph{Answer: } The problem is that FV(swap)={temp} and parameter
		temp $\in FV(swap)$. This means that the variable temp in function swap
		must be renamed to a name that is not a, b, or temp.

	\item Show how your solution would expand swap(x, temp) properly.
		\subparagraph{Answer: } Lets say variable temp in function swap is renamed
		to cat. The expanssion of swap(x, temp) would look like the following:
		\begin{lstlisting}
		swap(x,temp) = { typeof(x) cat = x; x = temp; temp = cat; }
		\end{lstlisting}

\end{enumerate}

\section{Free Variables [6pts]}
The F V function we discussed in class is a function that operates on syntax.
(If fact, you can think of F V as
one of the simplest possible static analysis.) Since F V is defined recursively,
it must be total, i.e., it must be
able to take any program (term) written in λ-calculus as input. To this end, the
function is defined for every
kind of term:

	\begin{align*}
		FV(x) &= x&\\
		FV(\lambda x.e) &= FV(e)\setminus\{x\}&\\
		FV(e_1\ e_2) &= FV(e_1)\cup FV(e_2)&\\
	\end{align*}
	Given this, find the free variables in the terms below:

	\begin{enumerate}
		\item $FV(\lambda x.(\lambda y.y))=\emptyset$
		\item $FV(\lambda x.(\lambda y.x))=\emptyset$
		\item $FV((\lambda x.x\ y)(\lambda y.y\ x))=\{x,y\}$
		\item $FV((\lambda p.\lambda q.\lambda r.p\ q\ r)(\lambda p.\lambda q.p\ q\
			r))=\{r\}$
		\item $FV(x\ y)=\{x,y\}$
		\item $FV(((\lambda f.\lambda y.\lambda x.f(y(x)))(\lambda x.y + z))(\lambda
			z.y-x))=\{x,y,z\}$
	\end{enumerate}

\section{$\lambda$-calculus reduction [6pts]}
In class, we talked about $\beta$-reduction (reduction via capture-avoiding
substitution) and $\alpha$-conversion (variable
renaming). These two rules can be used to fully describe the semantics of
$\lambda$-calculus and evaluate (a la term
rewriting) any $\lambda$ program.\\
There is, however, a third rule called $\eta$-conversion that is often used in
compiler optimizations and, in
Haskell, for point-free programming (as we will see). η-conversion says that you
wrap any function term in a
lambda expression: $\lambda x.e\ x =_\eta e$ if $x \notin FV(e)$. Since this is just another
equation in our toolbox, you can
also drop abstractions according to $\eta$-conversion.\\
In this problem you will reduce the following term ($\lambda x.\lambda y.x\ 
y)(\lambda x.x\ y)$ according
to our equations theory.
At every step note if you are doing a $\beta$-reduction, $\alpha$-conversion, or
$\eta$-conversion. You should do all possible
reductions to get the shortest possible expression. There are multiple ways to
reduce this expression; below
you will consider two different ways, both starting with an $\alpha$-conversion. You
may not do multiple steps at
once or use more lines than allocated.
\par First approach:
	\begin{align*}
		&(\lambda x.\lambda y.x\ y)(\lambda x.x\ y)&\\
		=_\alpha&\ (\lambda x.\lambda z.x\ z)(\lambda x.x\ y)&\\
		=_\beta&\ \lambda z.((\lambda x.x\ y)\ z)&\\
		=&\lambda z.(z\ y)&\\
	\end{align*}

\par Second approach:
	\begin{align*}
		&(\lambda x.\lambda y.x\ y)(\lambda x.x\ y)&\\
		=_\alpha&\ (\lambda x.\lambda z.x\ z)(\lambda x.x\ y)&\\
		=_\eta&\ (\lambda x.x)(\lambda x.x\ y)&\\
		=&\lambda x.(x\ y)&\\
	\end{align*}

\section*{Acknowledgements}

\end{document}

