\documentclass{article}
\usepackage{listings}
\usepackage{mathtools, amssymb}
\usepackage{tikz}
\usepackage{enumitem}
\usepackage[margin=1in]{geometry}
\lstset {
	tabsize=2
}

\title{Homework 4: Polymorphism}
\date{Feb 23, 2019}
\author{Jack Shi - A92122910 \\
				Wyatt Guidry - A12994977 \\
				Austin Coleman - A12888539}

\begin{document}
\maketitle
% 1
\section{[16pts] Type Polymorphism}
% 1.1
\subsection{Parametric Polymorphism}
\begin{enumerate}
	% 1. Polymorphic abstract data type for stacks
	\item \textbf{Polymorphic abstract data type for stacks}
		\begin{enumerate}
      % a.
			\item Complete implementation
				\begin{lstlisting}
	data Stack a = Stack [a]
	push :: Stack [a] -> a -> Stack [a]
	push (Stack xs) x = Stack (x:xs)

	pop :: Stack [a] -> Stack [a]
	pop (Stack []) = Stack []
	pop (Stack xs) = Stack (drop 1 xs)
				\end{lstlisting}

      % b.
			\item Explain type functions
				\textbf{Answer: }\\
				\texttt{push} has type \texttt{Stack [a] -> a -> Stack
				[a]}. This means that the \texttt{push} function takes a stack that is
				constructed from a list of type \texttt{a}, and a new element of type
				\texttt{a}, and return a new stack that is constructed from a list of
				type \texttt{a}. The return has the same type as the first argument
				since adding an element to the stack does not change it's type.\\
				\texttt{pop} has type \texttt{pop::Stack [a] -> Stack [a]}. It is a
				function that takes a \texttt{Stack} that is constructed from a list of
				type \texttt{a} and the return is the same type because removing an
				element from a \texttt{Stack} does not change the type of the \texttt{Stack}.
		\end{enumerate}
\end{enumerate}

% 1.2
\subsection{Ad-hoc Polymorphism}
\begin{enumerate}
	% 1.
	\item \textbf{Collection interface via type classes}
		\begin{enumerate}
      % a.
			\item Complete implementation
				\begin{lstlisting}
	class Collection c where
		push :: c -> int -> c
		pop :: c -> c
	
	data Stack = Stack [int] deriving Show

	-- | Make Stack an instance of the Collection class
	instance Collection (Stack [c]) where
		push (Stack xs) x = Stack (x:xs)
		pop (Stack s) = if (lenght s) == 0 
										then Stack s
										else Stack (drop 1 s)
	
	data Queue = Queue [Int] deriving Show
	
	-- | Make Queue an instance of the Collection class

	instance Collection (Queue [c]) where
		push (Queue xs) x = Queue (xs ++ [x])
		pop (Queue s) = if (lenght q) == 0 
										then Queue q
										else Queue (drop 1 q)
				\end{lstlisting}
		\end{enumerate}
\end{enumerate}

% 2
\section{[18pts] Implementing Haskell Typeclasses}
\begin{enumerate}
	% 1
	\item \textbf{Answer: }\\
	\texttt{MyEqD} represent a dictionary that contains the implementation for the
	operation \texttt{===}. The type generated is an operator that takes and
	dictionary and two arguments then return a Boolean. The generated function
	provide a template for pattern match for this specific dictionary. When the
	operator is invoked, the pattern matching will automatically type match to the
	correct dictionary therefore \texttt{===} would be able to find the correct
	implementation for that specific type.

	% 2
	\item Complete implementation
		\begin{lstlisting}
	data Tree a = Leaf a | Node a (Tree a) (Tree a)
								deriving Show
	instance MyEq (Tree a) where
		(===) (Leaf a) (Leaf b) = abs a == abs b
		(===) (Node a1 (Tree b1) (Tree c1)) (Node a2 (Tree b2) (Tree c2)) =
			abs a1 == abs a2 && b1 (===) b2 && c1 (===) c2
		(===) _ _ 						  = False
		\end{lstlisting}

	% 3
	\item Complete implementation
		\begin{lstlisting}
	dMyEqTree :: MyEqD a -> Tree a -> Tree a -> Bool
	dMyEqTree elDict = MkMyEqD myEqTree
		where myEqTree (Leaf a) (Leaf b) = (===) elDict a  b
		 			myEqTree (Node a1 (Tree b1) (Tree c1)) (Node a2 (Tree b2) (Tree c2)) =
						(===) elDict a1 a2 && b1 === b2 && c1 === c2
		      myEqTree _ _ = False
		\end{lstlisting}

	% 4
	\item \textbf{Answer: }\\
		The compiler will change the type of \texttt{cmp} to \texttt{cmp::MyEqD a
		-> a -> a -> String}. This function will now be passed in a dictionary and
		the dictionary will be passed into the \texttt{===} operator for
		implementation lookup when invoked in cmp.

	% 4
	\item Complete implementation: \\
		\begin{lstlisting}
	cmp :: MyEq a -> a -> a -> String
	cmp dMyEqInt a b = if (===) dMyEqInt a b
												then ``Equal''
												else ``Not Equal''
	result = cmp (dMyEqTree dMyEqInt) tree1 tree2
		\end{lstlisting}
\end{enumerate}
\end{document}

