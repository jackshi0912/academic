\documentclass[12pt]{article}

\usepackage[margin=1in]{geometry}
\usepackage{amsmath, amsthm, amsfonts, mathtools, amssymb}
\usepackage{faktor}
\usepackage{enumitem}

\begin{document}

\null\hfill\begin{tabular}[t]{l@{}}
	\textbf{Name: }Huize Shi - A92122910 \\
	\textbf{Discussion: }A04 \\
	\textbf{Homework: }6
\end{tabular}
\noindent\rule{\textwidth}{0.5pt}

\begin{enumerate}
	% 1
	\item
		\begin{enumerate}
			% 1 a
			\item Suppose p is a prime number. Prove that $x^p -x + 1$ has no zero in
				$\mathbb{Z}_p$.\\
				By Fermat's Little Theorem, $x^p \overset{p}{\equiv} x$. Hence the
				following is true:
				\begin{align*}
					x^p - x + 1 &\overset{p}{\equiv} 0\\
					1 &\overset{p}{\not\equiv} 0
				\end{align*}
				Ergo, $x^p -x + 1$ has no zero in $\mathbb{Z}_p$.

			% 1 b
			\item Prove that $x^3 - x + 1$ is irreducible in $\mathbb{Z}_3[x]$.

		\end{enumerate}

	% 2
	\item
		\begin{enumerate}
			% 2 a
			\item Prove that $f(x)=x^3 - 2$ is irreducible in $\mathbb{Q}[x]$.\\
				Since $2 \le deg\ f \le 3$, it is sufficient to show that $f$ does not
				have a zero in $\mathbb{Q}[x]$. Suppose towards contrary $\exists b, c
				\in \mathbb{Z}, c > 0, gcd(b,c)=1, f(\frac{b}{c})=0$.
				\begin{align*}
					f\left(\frac{b}{c}\right) = 0 &= \left(\frac{b}{c}\right)^3 - 2\\
						b^3 &= 2c^3\\
						c \mid b^3,\ gcd(c,b) = 1 &\Rightarrow c \mid 1 \\
						&\Rightarrow c = 1\\
						b \mid 2c^3,\ gcd(c,b) = 1 &\Rightarrow b \mid 2 \\
						&\Rightarrow b = \pm 1,\ b = \pm 2
				\end{align*}
				This limits the possibility of zeros to $x= \pm 1$ and $x= \pm 2$.
				\begin{align*}
				f(1) = -1 &\neq 0\\
				f(-1) = -3 &\neq 0\\
				f(2) = 6 &\neq 0\\
				f(-2) = -10 &\neq 0
				\end{align*}
				Hence shown $f(x)=x^3 - 2$ is irreducible in $\mathbb{Q}[x]$.
			% 2 b
			\item Let $\phi_{\sqrt[3]{2}} : \mathbb{Q}[x] \mapsto \mathbb{R}$ be the
				evaluation map $\phi_{\sqrt[3]{2}}(f(x))=f(\sqrt[3]{2})$. We know that
				$\phi_{\sqrt[3]{2}}$ is a ring homomorphism.
				\paragraph{(b-1) Prove that $ker\ \phi_{\sqrt[3]{2}} = \langle x^3-2
									 \rangle$}
				\begin{align*}
					\phi_{\sqrt[3]{2}}(x^3-2) &= (\sqrt[3]{2})^3 - 2\\
					&= 2-2 = 0
				\end{align*}
				Hence shown $\langle x^3-2 \rangle \subseteq ker\ \phi_{\sqrt[3]{2}}$.
				By part \textbf{a}, we know $x^3 - 2$ is irreducible. This implies that
				$\langle x^3-2 \rangle$ is maximal. Since $ker\ \phi_{\sqrt[3]{2}} \neq
				\mathbb{Q}[x]$, $\langle x^3-2 \rangle \subseteq ker\ \phi_{\sqrt[3]{2}}
				\Rightarrow \langle x^3-2 \rangle = ker\ \phi_{\sqrt[3]{2}}$.

				\paragraph{(b-2) Prove that $Im\ \phi_{\sqrt[3]{2}} = \{a_0 +
					\sqrt[3]{2}a_1 + (\sqrt[3]{2})^2a_2 \mid a_0, a_1, a_2 \in
					\mathbb{Q}\}$}
					\begin{align*}
						\forall f \in \mathbb{Q}[x],\ f(x) &= q(x)(x^3-2) + r(x)\\
						r(x) &= a_0 + xa_1 + x^2a_2\\
						\phi(f) &= q(\sqrt[3]{2})((\sqrt[3]{2})^3-2) + r(\sqrt[3]{2})\\
							&= 0 + r(\sqrt[3]{2})\\
							&= a_0 + \sqrt[3]{2}a_1 + (\sqrt[3]{2})^2a_2
					\end{align*}

				\paragraph{(b-3) Let $\mathbb{Q}[\sqrt[3]{2}] := \{a_0 +
					\sqrt[3]{2}a_1 + (\sqrt[3]{2})^2a_2 \mid a_0, a_1, a_2 \in
					\mathbb{Q}\}$. Prove that $\faktor{\mathbb{Q}[x]}{\langle x^3-2
					\rangle}\simeq \mathbb{Q}[\sqrt[3]{2}]$\\}
					In (b-2), we showed that $\phi$ is surjective. In (b-1) we showed that
					$ker\ \phi_{\sqrt[3]{2}} = \langle x^3-2 \rangle$. The $1^{st}$
					isomorphism theorem gives that $\faktor{\mathbb{Q}[x]}{\langle x^3-2
					\rangle} \simeq \mathbb{Q}[\sqrt[3]{2}]$.

					\paragraph{(b-4) Prove that $\mathbb{Q}[\sqrt[3]{2}]$ is a field.\\}
					Since we know that $x^3-2$ is irreducible, $\langle x^3-2 \rangle$ is
					therefore a maximal ideal. This means that
					$\faktor{\mathbb{Q}[x]}{\langle x^3-2 \rangle}$ is a field. By
					isomorphic relationship shown in (b-3) $\mathbb{Q}[\sqrt[3]{2}]$ is a
					field.
		\end{enumerate}

	% 3
	\item 
	\begin{enumerate}
		% 3 a
		\item Prove that $\sqrt{-21}$ is irreducible in $\mathbb{Z}[\sqrt{-21}]$\\
			$\sqrt{-21} \neq 0$, $\sqrt{-21}$ is not a zero divisor since
			$\mathbb{Z}[\sqrt{-21}]$ is a subring of $\mathbb{C}$.\\
			\begin{align*}
				\sqrt{-21}(a + b\sqrt{-21}) &= 1 \\
				\sqrt{-21}a - 21b &= 1 \\
				a &= 0\\
				b &= \frac{1}{-21}
			\end{align*}
			This is impossible since $\frac{1}{-21} \not\in
			\mathbb{Z}$. Hence $\sqrt{-21} \not\in
			\mathcal{U}(\mathbb{Z}[\sqrt{-21}])$.\\
			\begin{align*}
				\sqrt{-21} &= (a+b\sqrt{-21})(c+d\sqrt{-21})\\
				21 &= (a^2 + 21b^2)(c^2  + 21d^2)\\
				a^2 + 21b^2 &= 3 \Rightarrow b = 0 \Rightarrow a^2 = 3\\
				a^2 + 21b^2 &= 7 \Rightarrow b = 0 \Rightarrow a^2 = 7\\
			\end{align*}
			Since $a \in \mathbb{Z}$, this is impossible since $3, 7$ are not perfect
			sqaures. This means either $(a^2 + 21b^2)=1$ or $(c^2  + 21d^2)=1$. This
			means Either $(a^2 + 21b^2)$ or $(c^2  + 21d^2)$ must be a unit hence
			become $1$ under absolute norm (multiplied by conjugate).\\
			Hence shown $\sqrt{-21}$ is irreducible in $\mathbb{Z}[\sqrt{-21}]$.

		% 3 b
		\item Prove that $\langle\sqrt{-21}\rangle$ is not a prime ideal of
			$\mathbb{Z}[\sqrt{-21}]$\\
			\begin{align*}
				3 \cdot (-7) &= -21 = \sqrt{-21}\sqrt{-21} \in \langle\sqrt{-21}\rangle
			\end{align*}
			Claim $3, -7 \not\in \langle\sqrt{-21}\rangle$. Assume to the contrary
			that $3 \in \langle\sqrt{-21}\rangle$:
			\begin{align*}
				3 &= \sqrt{-21}(a + b\sqrt{-21})\\
				3 &= \sqrt{-21}a - 21b\\
				a &= 0\\
				b &= \frac{-3}{21} = \frac{-1}{7}
			\end{align*}
			This is impossible since $b \in \mathbb{Z}$.
			\begin{align*}
				-7 &= \sqrt{-21}(a + b\sqrt{-21})\\
				-7 &= \sqrt{-21}a - 21b\\
				a &= 0\\
				b &= \frac{7}{21} = \frac{1}{3}
			\end{align*}
			This is impossible since $b \in \mathbb{Z}$.
			Hence we have shown $3, -7 \not\in \langle\sqrt{-21}\rangle$,
			$3\cdot(-7)=-21 \in \langle\sqrt{-21}\rangle$. Ergo,
			$\langle\sqrt{-21}\rangle$ is not a prime ideal of
			$\mathbb{Z}[\sqrt{-21}]$.

		% 3 c
		\item Prove that $\mathbb{Z}[\sqrt{-21}$ is not a PID.\\
				Assume towards contrary that $\mathbb{Z}[\sqrt{-21}$ is a PID. Since
					$\sqrt{-21}$ is irreducible in $\mathbb{Z}[\sqrt{-21}$,
						$\langle\sqrt{-21}\rangle$ is a maximal ideal of
						$\mathbb{Z}[\sqrt{-21}$. This means that $\langle\sqrt{-21}\rangle$
							is also a prime ideal. This is a contradiction by the result of
							part \textbf{b}. Hence $\mathbb{Z}[\sqrt{-21}$ is not a PID.
	\end{enumerate}

	% 4
	\item let $\omega := \frac{-1 + \sqrt{3}}{2}$. $\omega^2 + \omega + 1 = 0$;
		$\omega + \bar{\omega} = -1$ and $\omega\bar{\omega} = 1$ where
		$\bar{\omega}$ is the complex conjugate of $\omega$. Let $\mathbb{Z}[\omega]
		:= \{a + b\omega \mid a,b \in \mathbb{Z}\}$. We know that
		$\mathbb{Z}[\omega]$ is a subring of $\mathbb{C}$. Let $\mathbb{Q}[\omega]
		:= \{a+b\omega \mid a, b \in \mathbb{Q}\}$.
		\begin{enumerate}
			% 4 a
			\item Prove that $\faktor{\mathbb{Q}[x]}{\langle x^2 + x + 1 \rangle}
				\simeq \mathbb{Q}[\omega]$ and $\mathbb{Q}[\omega]$ is a field.\\
				The evaluation map $\phi_{\omega}:\mathbb{Q}[x] \mapsto
				\mathbb{Q}[\omega]$ by $\phi_{\omega}(f(x)) = f(\omega)$ is a ring
				homomorphism. $Im\ \mathbb{Q}[\omega] = \mathbb{Q}[\omega]$ by
				definition of evaluation map.
				\begin{align*}
					\phi(x^2 + x + 1) &= \omega^2 + \omega + 1 = 0
				\end{align*}
				This shows that $\langle x^2 + x + 1 \rangle \subseteq ker\ \phi$. \\
				Show that $x^2 + x + 1$ is irreducible in $\mathbb{Q}[x]$:\\
				Since we know that $\mathbb{Q}[x]$ is a integral domain, $x^2
				+ x + 1$ is not a zero divisor. $x^2 + x + 1 \neq 0$. Assume $x^2 + x +
				1$ has a zero in $\mathbb{Q}[x]$, $\frac{a}{b}$, $gcd(a,b) = 1$, $b>0$.
				\begin{align*}
					\left(\frac{a}{b} \right)^2 + \frac{a}{b} + 1 &= 0\\
					a^2 + ab + b^2 &= 0\\
					a^2 &= b(-a-b)\\
					b \mid a^2,\ gcd(a,b)=1 &\Rightarrow b \mid 1 \Rightarrow b=1\\
					b^2 = a(-a-b)\\
					a \mid b^2,\ gcd(a,b)=1 &\Rightarrow a \mid 1 \Rightarrow a= \pm 1\\
					1^2 + 1 + 1 &= 3 \neq 0\\
					(-1)^2 - 1 + 1 &= 1 \neq 0
				\end{align*}
				Since the degree is between $2$ and $3$, this means $x^2 + x + 1$ is
				irreducible. Hence $\langle x^2 + x + 1 \rangle$ is maximal, $\langle
				x^2 + x + 1 \rangle \subseteq ker\ \phi \Rightarrow \langle
				x^2 + x + 1 \rangle = ker\ \phi$.\\
				By $1^{st}$ isomorphism theorem, we have $\faktor{\mathbb{Q}[x]}{\langle
				x^2 + x + 1 \rangle} \simeq \mathbb{Q}[\omega]$. Since $\langle x^2 + x
				+ 1 \rangle$ is maximal, we know $\faktor{\mathbb{Q}[x]}{\langle
				x^2 + x + 1 \rangle}$ is a field, and by isomorphism
				$\mathbb{Q}[\omega]$ is also a field.

			% 4 b
			\item Prove that for any $z \in \mathbb{Q}[\omega]$ there is $u \in
				\mathbb{Z}[\omega]$ such that $\mid z - u \mid \le
				\frac{\sqrt{3}}{3}$.\\
				As the image suggests, any $z$ chosen on the plain of
				$\mathbb{Q}[\omega]$ plain, there exist a $u \in \mathbb{Z}[w]$ such
				that $z$ fall within the regular hexagon centered around $u$. Hence the
				maximum distance would be the distance between a vertex and the center
				of the hexagon.
				\begin{align*}
					|z-u| &\le \frac{1}{2} \cdot cos\left(\frac{\pi}{6}\right)^{-1}\\
					|z-u| &\le \frac{1}{2} \cdot \frac{2}{\sqrt{3}}\\
					|z-u| &\le \frac{\sqrt{3}}{3}
				\end{align*}

			% 4 c
			\item Prove that for any $a \in \mathbb{Z}[\omega]$ and $b \in
				\mathbb{Z}[\omega] \setminus \{0\}$, there are $q, r \in
				\mathbb{Z}[\omega]$ such that 
				\begin{align*}
					a &= bq + r\\
					r &= a - bq\\
					r &= b(\frac{a}{b} - q)\\
					|r| &\leq \frac{\sqrt{3}}{3} |b|
				\end{align*}
				Consider $\frac{a}{b} \in \mathbb{Q}[\omega]$, by part \textbf{b} we
				know $\exists q \in \mathbb{Z}[\omega]$ such that $|\frac{a}{b} - q| \le
				\frac{\sqrt{3}}{3}$.
				\begin{align*}
				\left|\frac{a}{b} - q\right| &\le \frac{\sqrt{3}}{3}\\
				\left|b\left(\frac{a}{b} - q\right)\right| &\le \frac{\sqrt{3}}{3}|b|\\
				|r| &\le \frac{\sqrt{3}}{3}|b|
				\end{align*}
				
			% 4 d
			\item Prove that $\mathbb{Z}[\omega]$ is a Euclidean domain
				\begin{align*}
					\mathcal{N}(a) &= |a|^2\\
					|a|^2 &= 0 \Rightarrow a=0\\
					|r| \le \frac{\sqrt{3}}{3}|b| &\Rightarrow |r|^2 \le
					\left|\frac{\sqrt{3}}{3}b\right|^2\\
					&\Rightarrow \mathcal{N}(r) \le
					\mathcal{N}\left(\frac{\sqrt{3}}{3}b\right)\\
					\left|\frac{\sqrt{3}}{3}\right|^2 > 1 &\Rightarrow \mathcal{N}(r) <
					\mathcal{N}(b)\\
				\end{align*}
				Hence shown $\mathbb{Z}[\omega]$ is a Euclidean domain.
			
			% 4 e
			\item Show that $\mathbb{Z}[\omega]$ is a PID.\\
				By theorem, a Euclidean domain is a PID. Hence shown that
				$\mathbb{Z}[\omega]$ is a PID by part \textbf{d}.
		\end{enumerate}

	% 5
	\item Suppose $a, b \in \mathbb{Z}$ and $a^2 + ab+b^2 = p$ is a prime number
		$>3$.
	\begin{enumerate}
		% 5 a
		\item Prove that $a-b\omega$ is irreducible in $\mathbb{Z}[\omega]$.\\
			$a-b\omega \neq 0$, since $\mathbb{Z}[\omega]$ is a integral domain,
			$a-b\omega$ is not a zero divisor.\\
			Assume $a-b\omega$ is a unit:
			\begin{align*}
				(a-b\omega)(c+d\omega) &= 1\\
				|(a-b\omega)(c+d\omega)|^2 &= |1|^2\\
				(a-b\omega)(a-b\bar\omega)(c+d\omega)(c+d\bar\omega) &= 1\\
				(a^2 -ab\bar\omega - ab\bar\omega + b^2\omega\bar\omega)
				(c^2 + cd\bar\omega + cd\bar\omega + d^2\omega\bar\omega) &= 1\\
				(a^2 -ab(\bar\omega +\bar\omega) + b^2\omega\bar\omega)
				(c^2 + cd(\bar\omega + \bar\omega) + d^2\omega\bar\omega) &= 1\\
				(a^2 + ab + b^2)(c^2-cd+d^2) &= 1\\
				p(c^2-cd+d^2) &= 1\\
				c^2-cd+d^2 &= \frac{1}{p}
			\end{align*}
			This is impossible since $c^2-cd+d^2 \in \mathbb{Z}$. Hence $a-b\omega$ is
			not a unit.

			\begin{align*}
				a-b\omega &= (c+d\omega)(e+f\omega)\\
				|a-b\omega|^2 &= |(c+d\omega)(e+f\omega)|^2\\
				a^2 + ab + b^2 &= (c^2-cd+d^2)(e^2-ef+f^2) = p
			\end{align*}
			Since this is set to equal a prime, $c^2-cd+d^2$ or $e^2-ef+f^2$ must
			be $1$. Since absolute norm is just multplication by conjugate, we know
			that either $c+d\omega$ or $e+f\omega$ is in
			$\mathcal{U}(\mathbb{Z}[\omega])$.\\
			Ergo $a-b\omega$ is irreducible in $\mathbb{Z}[\omega]$.
			
		% 5 b
		\item Prove that $\exists \alpha \in \mathbb{Z}_p$ such that
			\paragraph{(b-1) $\alpha^2 + \alpha + 1 = 0$ in $\mathbb{Z}_p$}
				\begin{align*}
					a^2 + ab+b^2 &= p\\
					a^2 + ab+b^2 &\overset{p}{\equiv} 0
				\end{align*}
				Wants to show $b \neq 0$. Assume $b=0$:
				\begin{align*}
					p \mid b &\Rightarrow p \mid ab \Rightarrow p\mid a^2\\
					p^2 \mid b^2 &\Rightarrow p^2 \mid ab \Rightarrow p^2\mid a^2\\
					p^2 &\mid a^2 + ab+b^2
				\end{align*}
				This is a contradiction. Hence $b \neq 0$.
				\begin{align*}
					a^2 + ab+b^2 &\overset{p}{\equiv} 0\\
					\left(\frac{a}{b}\right)^2 + \frac{a}{b}+1 &\overset{p}{\equiv} 0
				\end{align*}
				Since $\mathbb{Z}_p$ is a field, $b \neq 0$, $\frac{a}{b} \in
				\mathbb{Z}_p$. Let $\alpha := \frac{a}{b}$. We have 
				$\alpha^2 + \alpha + 1 \overset{p}{\equiv} 0$ as specified.

			\paragraph{(b-2) $a - b\alpha = 0$ in $\mathbb{Z}_p$}
			\begin{align*}
				\alpha &:= \frac{a}{b}\\
				a - b\alpha &= a - b\frac{a}{b}\\
										&= a - a = 0
			\end{align*}
			$\alpha := \frac{a}{b}$ satisfies both conditions.

		% 5 c
		\item Let $\phi:\mathbb{Z}[\omega] \mapsto \mathbb{Z}_p$, $\phi_\alpha(c+d\omega) :=
			c+d\alpha$ where $\alpha$ is given in part \textbf{b}. Prove that $\phi$
			is a ring homomorphism.
			\paragraph{Closed under addition:} 
			\begin{align*}
			(c+d\alpha) + (e+f\alpha) &= (c+e) + (d+f) (\alpha)\\
			\end{align*}
			\paragraph{Closed under multiplication:}
			\begin{align*}
				(c+d\alpha) \cdot (e+f\alpha) &= ce + (de + cf)\alpha + df\alpha^2\\
				&= ce + (de + cf + df\frac{a}{b})\alpha
			\end{align*}
			Hence shown $\phi$ is a ring homomorphism.

		% 5 d
		\item Prove that $ker\ \phi = \langle a-b\omega \rangle$
			\begin{align*}
				\phi_\alpha(a-b\omega) &= a-b\alpha = 0\\
				a-b\alpha \in ke\ \phi &\Rightarrow \langle a-b\omega \rangle \subseteq
				ke\ \phi
			\end{align*}
			By part \textbf{5 a}, we know that $a-b\omega$ is irreducible in
				$\mathbb{Z}[\omega]$, since $\mathbb{Z}[\omega]$ is a PID, we know that
				$\langle a-b\omega \rangle$ is a maximal ideal, $\langle a-b\omega
				\rangle \subseteq ke\ \phi$ means that $ker\ \phi = \langle a-b\omega
				\rangle$.

		% 5 e
			\item Prove that $\faktor{\mathbb{Z}[\omega]}{\langle a-b\omega \rangle}
				\simeq \mathbb{Z}_p$\\
				Show that $\phi$ is surjective: $\forall z \in \mathbb{Z}_p, \phi(z) =
				z$.\\
				By part \textbf{5 d}, we know $ker\ \phi = \langle a-b\omega \rangle$.
				By part \textbf{5 c}, we know that $\phi$ is a ring homomorphism. By the
				$1^{st}$ isomorphism theorem, $\faktor{\mathbb{Z}[\omega]}{\langle
				a-b\omega \rangle} \simeq \mathbb{Z}_p$.

	\end{enumerate}

\end{enumerate}

\end{document}

