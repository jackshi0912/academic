\documentclass[12pt]{article}

\usepackage[margin=1in]{geometry}
\usepackage{amsmath, amsthm, amsfonts, mathtools, amssymb}
\usepackage{faktor}
\usepackage{enumitem}

\begin{document}

\null\hfill\begin{tabular}[t]{l@{}}
	\textbf{Name: }Huize Shi - A92122910 \\
	\textbf{Discussion: }A04 \\
	\textbf{Homework: }4
\end{tabular}
\noindent\rule{\textwidth}{0.5pt}

\begin{enumerate}
	% 1
	\item Prove that $\faktor{\mathbb{Q}[x]}{\langle x^2-2 \rangle} \simeq
		\mathbb{Q}[\sqrt{2}]$.
		\paragraph{Define ring homomorphism:}$f:\mathbb{Q}[x] \mapsto
		\mathbb{Q}[\sqrt{2}]$ by evaluation map $f(g(x)) = g(\sqrt{2})$. We have
		already proven in class that evalution maps are homomorphic.

		\paragraph{Show $\langle x^2-2 \rangle = ker\ f$:}$\ $\\
		First show $\langle x^2-2 \rangle \subseteq ker\ f$:
			\begin{align*}
				f(x^2-2) &= (\sqrt{2})^2 - 2\\
								 &= 2-2\\
								 &= 0
			\end{align*}
			Hence shown $x^2-2 \in ker\ f$, $\langle x^2-2 \rangle \subseteq ker\
			f$\\\\

		Then show $ker\ f \subseteq \langle x^2-2 \rangle$: \\
		Given generic value in $ker\ f$, $g(x)$, show that it is in (divisible by)
		$\langle x^2-2 \rangle$.
		\begin{align*}
			g(x) \in ker\ f &\Rightarrow g(x)=q(x)\cdot(x^2-2) + r(x) \text{ where
			$deg(r) \le 1$}\\
			&\Rightarrow r(x) = l_1x+l_2
		\end{align*}
		Show that $r$ is $0$ if it is in the kernel.
		\begin{align*}
			f(r(x)) = l_1\sqrt{2} + l_2 = 0 \Rightarrow l_1=l_2=0
		\end{align*} 
		This implication is true since otherwise $l_2$ will need to be a multiple
		of $\sqrt{2}$ which is not in $\mathbb{Q}$. Hence shown $x^2-2 \mid ker\
		f$, $ker\ f \subseteq \langle x^2-2 \rangle$.\\\\
	  Hence shown $ker\ f=\langle x^2-2 \rangle$.

		\paragraph{$f$ is surjective:} Given any $a + b\sqrt{2} \in
		\mathbb{Q}[\sqrt{2}]$, $f(a+bx) = a + b\sqrt{2}$. 

		By $1^{st}$ isomorphism theorem, $\faktor{\mathbb{Q}[x]}{\langle x^2-2 \rangle}
		\simeq \mathbb{Q}[\sqrt{2}]$.

	% 2
	\item Prove that $\faktor{\mathbb{Z}[i]}{\langle 2+i \rangle} \simeq
		\faktor{\mathbb{Z}}{5\mathbb{Z}}$.
		\paragraph{Define ring homomorphism:}$f:\mathbb{Z}[i] \mapsto
		\faktor{\mathbb{Z}}{5\mathbb{Z}}$ by $f(a+bi) = a-2b$.
		\begin{align*} 
		f((a+bi)+(c+di)) &= f((a+c) + (b+d)i)\\
										 &= (a+c) - 2(b+d)\\
										 &= a - 2b + c -2d\\
										 &= f(a+bi) + f(c+di)
		\end{align*} 
		\begin{align*} 
			f((a+bi)\cdot(c+di)) &\overset{?}{=} f(a+bi)\cdot f(c+di)\\
		 f((ac-bd) + (ad+bc)i) &\overset{?}{=} (a-2b)(c-2d)\\
										(ac-bd) - 2(ad+bc) &= (ac-bd) - 2(ad+bc)
		\end{align*} 

		\paragraph{Show $\langle 2+i \rangle = ker\ f$:}$\ $\\
		First show $\langle 2+i \rangle \subseteq ker\ f$:
			\begin{align*}
				f(2+i) &= 2 - 2\\
							 &= 0
			\end{align*}
			Hence shown $2+i \in ker\ f$, $\langle 2+i \rangle \subseteq ker\
			f$\\\\
		Then show $ker\ f \subseteq \langle 2+i \rangle$:\\
		Given generic element in $ker\ f$, $a+bi$:
		\begin{align*}
			\frac{a+bi}{2+i} = a' + b'i &= (q_1 + e_1) + (q_2 + e_2)i \text{ where
				$q\in\mathbb{Z}$, $|e| < \frac{1}{2}$} \\
																	&= (q_1 + q_2i) + ( e_1 + e_2i)\\
														 a+bi &= (2+i)(q_1 + q_2i) + (2+i)( e_1 + e_2i)
		\end{align*}
		Let $r = r_1 + r_2i =  (2+i)( e_1 + e_2i)$. 
		\begin{align*}
			|r|^2 &= |2+i|^2 \cdot |e_1 + e_2i|^2\\
						&= 5 \cdot (e_1^2+e_2^2) \le 2.5\\
			|r|^2 &= r_1^2 + r_2^2 \le 2.5 \Rightarrow |r_i| \le 1
		\end{align*}
		Since $r \in ker\ f$, $f(r) = r_1 - 2r_2 + 5\mathbb{Z}$. $5 \mid r_1 -
		2r_2$, and $r_i \in {-1, 0, 1} \Rightarrow r_1 = r_2 = 0 \Rightarrow r=0$.

		\paragraph{$f$ is surjective:} Given any element in
		$\faktor{\mathbb{Z}}{5\mathbb{Z}}$, $a + 5\mathbb{Z}$, $f(a + 0i) = a +
		5\mathbb{Z}$. Hence $f$ is surjective.

		By $1^{st}$ isomorphism theorem, $ \faktor{\mathbb{Z}[i]}{\langle 2+i
		\rangle} \simeq \faktor{\mathbb{Z}}{5\mathbb{Z}}$

	% 3
	\item Suppose $m,n \in \mathbb{Z}^{\ge 2}$ and $gcd(m,n)=1$. Prove that 
		\begin{equation*}
		\faktor{\mathbb{Z}}{mn\mathbb{Z}} \simeq
		\faktor{\mathbb{Z}}{m\mathbb{Z}} \times
		\faktor{\mathbb{Z}}{n\mathbb{Z}}
		\end{equation*}

		\paragraph{Ring homomorphism:}
		\begin{align*}
			f : \mathbb{Z} &\mapsto \faktor{\mathbb{Z}}{m\mathbb{Z}} \times
				\faktor{\mathbb{Z}}{n\mathbb{Z}}\\
				f(z) &= (z+m\mathbb{Z}, z+n\mathbb{Z})
		\end{align*}
		\begin{align*}
				f(z+z) &= (2z+m\mathbb{Z}, 2z+n\mathbb{Z})\\
							 &= (z+m\mathbb{Z}, z+n\mathbb{Z}) + (z+m\mathbb{Z},
							 z+n\mathbb{Z})\\
							 &= f(z) + f(z)
		\end{align*}
		\begin{align*}
				f(zz) &= (zz+m\mathbb{Z}, zz+n\mathbb{Z})\\
							&= (z+m\mathbb{Z}, z+n\mathbb{Z})(z+m\mathbb{Z}, z+n\mathbb{Z})\\
							&= f(z)f(z)
		\end{align*}

		\paragraph{Show $ker\ f = mn\mathbb{Z}$} $\ $\\
		First show $mn\mathbb{Z} \subseteq ker\ f$:\\
			\begin{align*}
				f(mnz) &= (mnz + m\mathbb{Z}, mnz + n\mathbb{Z})\\
							 &= (0, 0)
			\end{align*}
		Then show $ker\ f \subseteq mn\mathbb{Z}$:\\
			\begin{align*}
				f(z) &= (0, 0)\\
				(z + m\mathbb{Z}, z + n\mathbb{Z}) &= 0
			\end{align*}
		This implies $m \mid z$, $n \mid z$. Since $gcd(m,n) = 1$, this implies $z
		\in mn\mathbb{Z}$\\
		Hence shown $ker\ f = mn\mathbb{Z}$.

	% 4
	\item Prove that $\faktor{\mathbb{Z}[x]}{n\mathbb{Z}[x]} \simeq
		\mathbb{Z}_n[x]$.
		\begin{align*}
			f:\mathbb{Z}[x] &\mapsto \mathbb{Z}_n[x]\\
			f\left(\sum_{i=0}^n a_ix^i\right) &= \sum_{i=0}^n a_ix^i + n\mathbb{Z}
		\end{align*}
		$f$ is clearly surjective and homomorphic since it sends every element to
		their corresponding elements mod n (multiplication and addition maps with
		mod n applied before or after does not matter).

		\paragraph{Show $ker\ f = n\mathbb{Z}[x]$:} Showing subsets for both
		directions.\\
		First show $n\mathbb{Z}[x] \subseteq ker\ f$:\\
		\begin{align*}
			f(na_ix^i) =na_ix^i + 5\mathbb{Z} \overset{n}{\equiv} 0
		\end{align*}
		Then show $ker\ f \subseteq n\mathbb{Z}[x]$. Since $f$ maps $g$ to $g$ where
		all coefficients are remainders divided by n, $g(x) \in
		ker\ f$ implies that the coefficients of $g(x)$ must be $0\ mod\ n$. This
		immplies that the coefficients of the kernel of $f$ is in $n\mathbb{Z}$. This
		means that the kernel of $f$ is in $n\mathbb{Z}[x]$.\\
		
		\paragraph{Show that $f$ is surjective:}This is true by the definition of
		$f$:
		\begin{align*}
		f\left(\sum_{i=0}^n a_ix^i\right) &= \sum_{i=0}^n a_ix^i +
		n\mathbb{Z}\\
		\sum_{i=0}^n a_ix^i + n\mathbb{Z} &= \mathbb{Z}_n[x]
		\end{align*}
		By $1^{st}$ isomorphism theorem, $\faktor{\mathbb{Z}[x]}{n\mathbb{Z}[x]} \simeq
		\mathbb{Z}_n[x]$.

	% 5
	\item Prove that $\faktor{\mathbb{Q}[x]}{\langle x^2-2x+6\rangle} \simeq
		\{c_0 + c_1A \mid c_0, c_1 \in \mathbb{Q}\}$, where $A=\begin{bmatrix}0 &
			-6 \\ 1 & 2\end{bmatrix}$.
		\begin{align*}
			\phi_A:\mathbb{Q}[x] &\mapsto M_2(\mathbb{Q})\\
			\phi_A\left(\sum^n_{i=0}a_ix^i\right) &= \sum^n_{i=0}a_iA^i
		\end{align*}

		\paragraph{Show $\langle x^2 - 2x + 6\rangle = ker\ \phi_A$:} $\ $\\
			First show that $\langle x^2-2x+6\rangle \subseteq ker\ \phi_A$:
			\begin{align*}
				\phi_A(x^2 - 2x + 6) &= A^2 - 2A + 6\\
														 &= \begin{bmatrix}-6 & -12 \\ 2 & -2\end{bmatrix} -
																\begin{bmatrix}0 & -12 \\ 2 & 4\end{bmatrix} +
																\begin{bmatrix}6 & 0 \\ 0 & 6\end{bmatrix}\\
														 &= \begin{bmatrix}0 & 0 \\ 0 & 0\end{bmatrix}
			\end{align*}

			Then show that $ker\ \phi_A \subseteq \langle x^2-2x+6\rangle$:
			\begin{align*}
				f(x) \in ker\ \phi_A &\Rightarrow f(x) = p(x)(x^2 - 2x + 6) + r(x)
				\text{ where $deg(r) \le 1$}\\
				&\Rightarrow r(x) = l_1 x + l_2 \\
				&\Rightarrow l_1A + l_2 = 0\\
				&\Rightarrow A = l_2 l_1^{-1} \text{ However, $l$ might not be invertable}\\
				&\Rightarrow l_1 = l_2 = 0
			\end{align*}

		\paragraph{Show that $\phi_A$ is surjective:}Show $Im\ \phi_A = \{c_0 +
			c_1A \mid c_0, c_1 \in \mathbb{Q}\}$\\
			Show $\sum^n_{i=0}a_ix^i \subseteq \{c_0 + c_1A \mid c_0, c_1 \in
				\mathbb{Q}\}$ \\
			Since $\phi_A(x^2 -2x + 6) = A^2 - 2A + 6 = 0 \Rightarrow A^2 = 2A - 6
			$, therefore we can reduce degree from two to one. For equation of any
			degree, we can reduce the degree to first degree which is in $\{c_0 +
			c_1A \mid c_0, c_1 \in \mathbb{Q}\}$. This means that $\phi_A$ is
			surjective. \\

		By $1^{st}$ isomorphism theorem, $\faktor{\mathbb{Q}[x]}{\langle x^2-2x+6\rangle} \simeq
		\{c_0 + c_1A \mid c_0, c_1 \in \mathbb{Q}\}$.

\end{enumerate}

\end{document}

