\documentclass[12pt]{article}

\usepackage[margin=1in]{geometry}
\usepackage{amsmath, amsthm, amsfonts, mathtools, amssymb}
\usepackage{faktor}
\usepackage{enumitem}

\begin{document}

\null\hfill\begin{tabular}[t]{l@{}}
	\textbf{Name: }Huize Shi - A92122910 \\
	\textbf{Discussion: }A04 \\
	\textbf{Homework: }5
\end{tabular}
\noindent\rule{\textwidth}{0.5pt}

\begin{enumerate}
	% 1
	\item
		\begin{enumerate}
			% 1 a
			\item Prove that $\sqrt{-10}$ is irreducible in $\mathbb{Z}[\sqrt{-10}] =
				\{a+\sqrt{-1}b \mid a,b\in \mathbb{Z}\}$
				\begin{align*}
					\sqrt{-10} &= (a+\sqrt{-10}b)(c+\sqrt{-10}d)\\
					10 &= (a^2+10b^2)(c^2+10d^2)\\
					a^2 + 10b^2 &\in \{1, 2, 5, 10\}
				\end{align*}
				If $b\neq0$, $a^2 + 10b^2 \ge 10$. This means that $(c^2+10d^2) = 1$,
				which implies $(c+\sqrt{-10}d)(c-\sqrt{-10}d)=1$, and this shows that
				$(c+\sqrt{-10}d) \in \mathbb{U}(\mathbb{Z}[\sqrt{-10}])$.\\
				If $b=0$, $a^2 + 10b^2 \in \{1, 2, 5, 10\} \Rightarrow a^2=a=1$ since 1
				is the only perfect square option. This implies the following:
				\begin{align*}
					a^2 + 10b^2 &= 1\\
					(a+\sqrt{-10}b)(a-\sqrt{-10}b) &= 1\\
					a+\sqrt{-10}b \in \mathbb{U}(\mathbb{Z}[\sqrt{-10}])
				\end{align*}
				Hence shown $\sqrt{-10}$ is irreducible in $\mathbb{Z}[\sqrt{-10}$.

			% 1 b
			\item Show that $2 \times 5 \in \langle \sqrt{-10} \rangle$ and $2 \not\in
				\langle \sqrt{-10} \rangle$ and $5 \not\in \langle \sqrt{-10} \rangle$.
				\begin{align*}
					2 \times 5 = 10 &= -\sqrt{-10} \times \sqrt{-10} \in \langle\sqrt{-10}\rangle
				\end{align*}
				Assume towards contradiction that $2 \in \langle\sqrt{-10}\rangle$.
				\begin{align*}
					2 &= \sqrt{-10} \cdot (a + b\sqrt{-10})\\
						&= \sqrt{-10}a - 10b\\
						&\Rightarrow a=0,\ b= -\frac{1}{5}
				\end{align*}
				$b=-\frac{1}{5} \not\in \mathbb{Z}$, Contradiction!!\\
				Assume towards contradiction that $5 \in \langle\sqrt{-10}\rangle$.
				\begin{align*}
					5 &= \sqrt{-10} \cdot (a + b\sqrt{-10})\\
						&= \sqrt{-10}a - 10b\\
						&\Rightarrow a=0,\ b= -\frac{1}{2}
				\end{align*}
				$b=-\frac{1}{2} \not\in \mathbb{Z}$, Contradiction!!\\
				Hence shown $2 \times 5 \in \langle \sqrt{-10} \rangle$ and $2 \not\in
				\langle \sqrt{-10} \rangle$ and $5 \not\in \langle \sqrt{-10} \rangle$.

			% 1 c
			\item Prove that $\mathbb{Z}[-10]$ is not a PID.
				\begin{proof} Assume towards contrary that $\mathbb{Z}[-10]$ is a PID.
					By part a, we have shown that $\sqrt{-10}$ is irriducible. This means
					that $\langle \sqrt{-10} \rangle$ is maximal therefore prime. By part
					b, we have shown that it is not prime. This means that the assumption
					is false, $\mathbb{Z}[-10]$ is not a PID.
				\end{proof}
		\end{enumerate}

	% 2
	\item We are told that $p(x)=x^4-2x^3+ 2x^2-2x+2$ is irreducible in
		$\mathbb{Q}[x]$ and $\alpha \in \mathbb{C}$ is a zero of $p(x)$. Let
		\begin{align*}
			\phi_\alpha: \mathbb{Q}[x] &\mapsto \mathbb{C}\\
			\phi_\alpha(f(x)) &:= f(\alpha)
		\end{align*}
		We know that $\phi_\alpha$ is a ring homomorphism.
		\begin{enumerate}
			% 2 a
			\item Prove that $ker\ \phi_\alpha = \langle p(x) \rangle$\\
				$\langle p(x) \rangle \subseteq ker\ \phi_\alpha$ since
				$\phi_\alpha(p(x)) = 0$.\\
				Since we know $p(x)$ is irreducible in $\mathbb{Q}[x]$, $\langle p(x)
				\rangle$ is therefore a maximal ideal. Since we have shown $\langle p(x)
				\rangle \subseteq ker\ \phi_\alpha \subsetneq \mathbb{Q}[x]$, by
				definition of maimal idea, $ker\ \phi_\alpha = \langle p(x) \rangle$.

			% 2 b
			\item Prove that $Im\ \phi_\alpha = \{c_0 + c_1 \alpha + c_2 \alpha^2 + c_3
					\alpha^3 \mid c_0, c_1, c_2, c_3 \in \mathbb{Q}\}$\\
					First show that $\{c_0 + c_1 \alpha + c_2 \alpha^2 + c_3
					\alpha^3 \mid c_0, c_1, c_2, c_3 \in \mathbb{Q}\} \subseteq Im\
					\phi_\alpha$:
					\begin{align*}
						\phi_\alpha(c_0 + c_1 x + c_2 x^2 + c_3 x^3) = c_0 + c_1 \alpha +
						c_2 \alpha^2 + c_3 \alpha^3
					\end{align*}

					Then show that $Im\ \phi_\alpha \subseteq \{c_0 + c_1 \alpha + c_2
						\alpha^2 + c_3 \alpha^3 \mid c_0, c_1, c_2, c_3 \in \mathbb{Q}\}$:
						\begin{align*}
							\forall f(x) &\in \mathbb{Q}[x], \exists q(x), r(x) \in  \mathbb{Q}[x]\\
							f(x) &= q(x)p(x) + r(x)\\
							\phi_\alpha(f) &= q(\alpha)p(\alpha) + r(\alpha)\\
							\phi_\alpha(f) &= 0 + r(\alpha)\\
							\phi_\alpha(f) &= c_0 + c_1 \alpha + c_2 \alpha^2 + c_3 \alpha^3
						\end{align*}
					Hence shown $Im\ \phi_\alpha = \{c_0 + c_1 \alpha + c_2 \alpha^2 + c_3
					\alpha^3 \mid c_0, c_1, c_2, c_3 \in \mathbb{Q}\}$.

			% 2 c
			\item Prove that $\faktor{\mathbb{Q}[x]}{\langle p(x) \rangle} \simeq
				\{c_0 + c_1 \alpha + c_2 \alpha^2 + c_3 \alpha^3 \mid c_0, c_1, c_2, c_3
					\in \mathbb{Q}\}$\\
				By $1^{st}$ isomorphism theorem, $\faktor{\mathbb{Q}[x]}{ker\
				\phi_\alpha} \simeq Im\ \phi_\alpha$. By part a, we have shown that
				$ker\ \phi_\alpha = \langle p(x) \rangle$. By part b, we have shown that 
				$Im\ \phi_\alpha = \{c_0 + c_1 \alpha + c_2 \alpha^2 + c_3 \alpha^3 \mid
				c_0, c_1, c_2, c_3 \in \mathbb{Q}\}$. By subsituting the corresponding
				parts we obtain the following:
				$$\faktor{\mathbb{Q}[x]}{\langle p(x) \rangle} \simeq
				\{c_0 + c_1 \alpha + c_2 \alpha^2 + c_3 \alpha^3 \mid c_0, c_1, c_2, c_3
					\in \mathbb{Q}\}$$

			% 2 d
			\item Prove that $\{c_0 + c_1 \alpha + c_2 \alpha^2 + c_3 \alpha^3 \mid
					c_0, c_1, c_2, c_3 \in \mathbb{Q}\}$ is a field.
					We know that $p(x)$ is irreducible, therefore $\langle p(x) \rangle$
					is a maximal ideal. This means that $\faktor{\mathbb{Q}[x]}{\langle
					p(x) \rangle}$ is a field. By isomorphism established in part c, we
					have $\{c_0 + c_1 \alpha + c_2 \alpha^2 + c_3 \alpha^3 \mid
					c_0, c_1, c_2, c_3 \in \mathbb{Q}\}$ is a field.

		\end{enumerate}

	% 3
	\item We are told that $R=\bigg\{\begin{bmatrix} a & b\\ b & a
		\end{bmatrix} \mid a, b \in \mathbb{Z}\bigg\}$ is a unital commutative ring.
		Let $\phi : R \mapsto \mathbb{Z}, \phi\left(\begin{bmatrix} a & b\\ b & a
		\end{bmatrix}\right) = a-b$

		\begin{enumerate}
			% 3 a
			\item Prove that $\phi$ is a ring homomorphism.
				\paragraph{Addition:}
					\begin{align*}
						\phi\left(\begin{bmatrix} a & b\\ b & a \end{bmatrix}
						+ \begin{bmatrix} c & d\\ d & c \end{bmatrix}\right) &=
							\phi\left(\begin{bmatrix} a+c & b+d\\ b+d & a+c
							\end{bmatrix}\right)\\
							&= (a+c) - (b+d)\\
							&= (a-b) + (c-d)\\
							&= \phi\left(\begin{bmatrix} a & b\\ b & a \end{bmatrix}\right)
							+ \phi\left(\begin{bmatrix} c & d\\ d & c \end{bmatrix}\right)
					\end{align*}

				\paragraph{Multiplication:}
					\begin{align*}
						\phi\left(\begin{bmatrix} a & b\\ b & a \end{bmatrix}
						\begin{bmatrix} c & d\\ d & c \end{bmatrix}\right) &=
							\phi\left(\begin{bmatrix} ac+bd & ad+bc\\ bc+ad & bd+ac
							\end{bmatrix}\right)\\
						&=\phi\left(\begin{bmatrix} ac+bd & ad+bc\\ ad+bc& ac+bd
							\end{bmatrix}\right)\\
						&=(ac+bd)-(ad+bc)\\
						&=ac-ad-bc+bd\\
						&=(a-b)(c-d)\\
						&=\phi\left(\begin{bmatrix} a & b\\ b & a \end{bmatrix}\right)
						\phi\left(\begin{bmatrix} c & d\\ d & c \end{bmatrix}\right)
					\end{align*}
					Hence shown $\phi$ is a ring homomorphism.

			% 3 b
			\item Find $ker\ \phi$.
				\begin{align*}
					\phi\left(\begin{bmatrix} a & b\\ b & a \end{bmatrix}\right) = a-b &= 0\\
					a &= b\\
					ker\ \phi &= \bigg\{\begin{bmatrix} a & a\\ a & a \end{bmatrix} \mid
					a \in \mathbb{Z}\bigg\}
				\end{align*}

			% 3 c
			\item Prove that $\faktor{R}{ker\ \phi} \simeq \mathbb{Z}$\\
				By $1^{st}$ isomorphism theorem, showing that $\phi: R \mapsto
				\mathbb{Z}$ is surjective completes the isomorphism proof.
				\begin{align*}
					\forall z \in \mathbb{Z}, \phi\left(\begin{bmatrix} z & 0\\ 0 & z
					\end{bmatrix}\right) = z-0 = z\\
				\end{align*}
				By $1^{st}$ isomorphism theorem, $\faktor{R}{ker\ \phi} \simeq
				\mathbb{Z}$.

			% 3 d
			\item Is $ker\ \phi$ a prime ideal?\\
				Yes, since $\mathbb{Z}$ is a integral domain.
			\item Is $ker\ \phi$ a maximal ideal?\\
				No, since $\mathbb{Z}$ is not a field.
		\end{enumerate}

		% 4
	\item
		\begin{enumerate}
			% 4 a 
			\item Show that $x^2 - 5 = 0$ has no zero in $\mathbb{Q}[\sqrt{2}]$.\\
			Suppose towards contrary that $\exists \alpha \in \mathbb{Q}[\sqrt{2}]$
			such that $m_\alpha(x) = x^2-5 \in \mathbb{Q}[x]$. 
			\begin{align*}
				\alpha &= a+b\sqrt{2}\\
				\phi_\alpha(m_\alpha) &= (a+b\sqrt{2})^2 -5=0\\
				0 &= a^2 + 2ab\sqrt{2} + 2b^2 - 5\\
				ab &= 0\\
				a^2 + 2b^2 &= 5
			\end{align*}
			Since $\mathbb{Q}[\sqrt{2}]$ is a subring of $\mathbb{C}$, it is an
				integral domain hence contain no zero divisors. This means either $a$ or
				$b$ must be 0.
			\paragraph{Case $a=0$:}
				\begin{align*}
					2b^2&=5\\
					b&=\sqrt{\frac{5}{2}}
				\end{align*}
				Since $b \in \mathbb{Q}$, this is impossible.
			\paragraph{Case $b=0$:}
				\begin{align*}
					a^2&=5\\
					a&=\sqrt{5}
				\end{align*}
				Since $a \in \mathbb{Q}$, this is impossible.\\
			Hence shown $x^2 - 5 = 0$ has no zero in $\mathbb{Q}[\sqrt{2}]$.

			% 4 b
			\item Prove that $\mathbb{Q}[\sqrt{2}] \not\simeq \mathbb{Q}[\sqrt{5}]$.\\
				Suppose that $\phi : \mathbb{Q}[\sqrt{2}] \mapsto \mathbb{Q}[\sqrt{5}]$
				is an isomorphism. 
				\begin{align*}
				\phi(1) &= 1\\
				\phi(a) &= a, \forall a\in\mathbb{Q}\\
				\phi(2) &= \phi(\sqrt{2}^2)\\
							2 &= \phi(\sqrt{2})^2\\
							2 &= (a+b\sqrt{5})^2\\
							2 &= a^2 + 2ab\sqrt{5} + 5b^2\\
						 ab &= 0\\
		 a^2 + 5b^2 &= 2
				\end{align*}
			Since $\mathbb{Q}[\sqrt{5}]$ is a subring of $\mathbb{C}$, it is an
				integral domain hence contain no zero divisors. Either $a$ or $b$ must
				be $0$. If $a=0$:
				\begin{align*}
					5b^2 &= 2\\
					b &= \sqrt{\frac{2}{5}}
				\end{align*}
				This is a contradiction since $b \in \mathbb{Q}$. If $b=0$:
				\begin{align*}
					a^2 &= 2\\
					a &= \sqrt{2}
				\end{align*}
				This is a contradiction since $a \in \mathbb{Q}$.\\
				Hence shown such an isomorphic mapping does not exist,
				$\mathbb{Q}[\sqrt{2}] \not\simeq \mathbb{Q}[\sqrt{5}]$.
		\end{enumerate}

	% 5
	\item 
		\begin{enumerate}
			% 5 a
			\item Suppose p is an odd prime, and there is $a\in\mathbb{Z}_p$ such that
				$a^2 = -1$ in $\mathbb{Z}_p$. Prove that the multiplicative order of $a$
				is $4$.
				\begin{align*}
					a^2 &\overset{p}{\equiv} -1\\
					a^2 &\overset{p}{\equiv} p-1\\
					(a^2)^2 &\overset{p}{\equiv} (p-1)^2\\
					a^4 &\overset{p}{\equiv} p^2 -2p +1\\
					a^4 &\overset{p}{\equiv} 1
				\end{align*}
				We know $a\neq 1$ since $a^2 = -1$ which also tells us $a^2 \neq 1$.
				$a^3 \neq 1$ because as shown above, $a^4 = 1$, $a^3 = 1$ implies that
				$a = 1$ which is a contradiction.\\
				Hence shown the multiplicative order of a is $4$.

			% 5 b
			\item Use Lagrange's theorem to deduce: if $p$ is a prime and $p
				\overset{4}{\equiv} 3$, then there is no $a\in\mathbb{Z}_p$ such that
				$a^2 = -1$.\\
				First we examine the unit of $\mathbb{Z}_p$, $\mathcal{U}(\mathbb{Z}_p)
				= \mathbb{Z}_p \setminus \{0\}$. The order of
				$\mathcal{U}(\mathbb{Z}_p)$ is $p-1$. A subgroup of this unit would be
				one generated by $a$ with multiplication, $\langle a \rangle = \{a^n
				\mid n\in [0,3]\}$. Order of $\langle a \rangle$ is $4$. By Lagrange's
				theorem, 4 divides p-1:
				\begin{align*}
					4 &\mid p-1\\
					p-1 &\overset{4}{\equiv} 0\\
					p &\overset{4}{\equiv} 1
				\end{align*}
				Hence shown if $p \overset{4}{\equiv} 3$, then there is no
				$a\in\mathbb{Z}_p$ such that $a^2 = -1$

			% 5 c
			\item Suppose $p$ is a prime and $p \overset{4}{\equiv} 3$. Prove that $p$
				is irreducible in $\mathbb{Z}[i]$. \\
				$p \neq 0$ and $p$ has no multiplicative inverse in $\mathbb{Z}[i]$,
				hence not a unit.
				\begin{align*}
					p &= (a+bi)(c+di)\\
					|p|^2 &= |a+bi|^2|c+di|^2\\
					p^2 &= (a^2 + b^2)(c^2 + d^2)
				\end{align*}
				This means $(a^2 + b^2)$ must be either $1, p, p^2$.
				\paragraph{Case $(a^2 + b^2)=1$:}
				\begin{align*}
					(a^2 + b^2)&=1\\
					(a+bi)(a-bi) &= 1
				\end{align*}
				Hence shown $(a+bi)$ is a unit.
				\paragraph{Case $(a^2 + b^2)=p^2 \Rightarrow (c^2 + d^2)=1$:}
				\begin{align*}
					(c^2 + d^2)&=1\\
					(c+di)(c-di) &= 1
				\end{align*}
				This means that $(c+di)$ is a unit.
				\paragraph{Case $(a^2 + b^2)=p$:}
				\begin{align*}
					(a^2 + b^2) &\overset{p}{\equiv}0\\
				\end{align*}
				We know that $b\neq0$ since that would implie $a^2=p$ which is
				impossible since $p$ is prime.
				\begin{align*}
					\frac{a^2}{b^2} +1 &\overset{p}{\equiv}0\\
					\left(\frac{a}{b}\right)^2 &\overset{p}{\equiv} -1
				\end{align*}
				Since  $\mathbb{Z}_p$ is a field therefore $b \neq 0 \Rightarrow b^{-1}
				\in \mathbb{Z}_p \Rightarrow \frac{a}{b} \in \mathbb{Z}_p$. Hence by
				part b, we know that this is impossible. By contradiction, we have shown
				that $p$ is irreducible in $\mathbb{Z}[i]$.

 			% 5 d
			\item Use part (c) to show $\faktor{\mathbb{Z}[i]}{\langle p \rangle}$
				is a field if p is a prime if $p$ is a prime and $p \overset{4}{\equiv}
				3$.\\
				Since we know if $p$ is a prime and $p \overset{4}{\equiv} 3$, $p$ is
				irreducible in $\mathbb{Z}[i]$. This means $\langle p \rangle$ is a
				maximal ideal of $\mathbb{Z}[i]$. The factor ring of $\mathbb{Z}[i]$
				over its maximal ideal, $\faktor{\mathbb{Z}[i]}{\langle p \rangle}$ is
				therefore a field by lemma proven in class.
		\end{enumerate}
\end{enumerate}

\end{document}
