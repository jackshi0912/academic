\documentclass[12pt]{article}

\usepackage[margin=1in]{geometry}
\usepackage{amsmath, amsthm, amsfonts, mathtools}
\usepackage{enumitem}
\usepackage{dsfont}

\begin{document}

\null\hfill\begin{tabular}[t]{l@{}}
	\textbf{Name: }Huize Shi - A92122910 \\
	\textbf{Discussion: }A04 \\
	\textbf{Homework: }2
\end{tabular}
\noindent\rule{\textwidth}{0.5pt}

\begin{enumerate}
	% 1.
	\item
		\begin{enumerate}
			% 1 a
			\item Find all the solutions of $x^2 - x - 2$ in $\mathbb{Z}_{17}$.
				\begin{align*}
					x^2 - x - 2 =(x-2)(x+1)
				\end{align*}
				Since $17$ is prime, we know that $\mathbb{Z}_{17}$ is a field which is
				also a integral domain. This means that $\mathbb{Z}_{17}$ has no zero
				divisors which means there are only two zeros, $2$ and $16$.
			% 1 b
			\item Does $x^2 - x - 2$ have only two zeros in  $\mathbb{Z}_{18}$?
				No, since $18$ is a composite, we know that besides $2$ and $17$,
				$x-2=9$, $x=11$, and $x+1=2$, $x=1$ make at least one other pair of
				zeros since $9 \cdot 2 = 0\ (mod\ 18)$.
		\end{enumerate}

	% 2.
	\item
		By Lemma prooved in class, given ring R, if $ord(1_R)<\infty$,
		$Char(R)=ord(1_R)$.
		\paragraph{Characteristic of $\mathbb{Z}_4 \times \mathbb{Z}_{6}$}
		\begin{align*}
			1_{\mathbb{Z}_4 \times \mathbb{Z}_{6}} &= (1_{\mathbb{Z}_4},
			1_{\mathbb{Z}_{6}})\\
			ord(1_{\mathbb{Z}_4 \times \mathbb{Z}_{6}}) &= LCM\left(
			ord(1_{\mathbb{Z}_4}), ord(1_{\mathbb{Z}_{6}})\right)\\
			&= LCM(4, 6) = 12
		\end{align*}
		\paragraph{Characteristic of $\mathbb{Z}_6 \times \mathbb{Z}_{8} \times
		\mathbb{Z}_{9}$}
		\begin{align*}
			1_{\mathbb{Z}_6 \times \mathbb{Z}_{8} \times \mathbb{Z}_{9}} &=
			(1_{\mathbb{Z}_6}, 1_{\mathbb{Z}_{8}}, 1_{\mathbb{Z}_{9}})\\
			1_{\mathbb{Z}_6 \times \mathbb{Z}_{8} \times \mathbb{Z}_{9}} &=
			LCM\left(ord(1_{\mathbb{Z}_6}), ord(1_{\mathbb{Z}_{8}}),
			ord(1_{\mathbb{Z}_{9}})\right)\\
			&= LCM(6, 8, 9) = 72
		\end{align*}

	% 3.
	\item
		\begin{enumerate}
			% 3 a
			\item Show $(\mathbb{Q}[\sqrt{2}], +, \cdot)$ is a field:
				\paragraph{$(\mathbb{Q}[\sqrt{2}], +)$ is Abelian Group:}
					\subparagraph{Associative:}
						\begin{align*}
							&((a + b\sqrt{2}) + (c + d\sqrt{2})) + (e + f\sqrt{2})\\
							=& ((a+c) + (b+d)\sqrt{2}) + (e + f\sqrt{2})\\
							=& ((a+c)+e) + ((b+d)+f)\sqrt{2}\\
							=& (a+(c+e)) + (b+(d+f))\sqrt{2}\\
							=& (a + b\sqrt{2})  ((c+e) + (d+f)\sqrt{2}) \\
							=& (a + b\sqrt{2}) + ((c + d\sqrt{2}) + (e + f\sqrt{2}))
						\end{align*}
					\subparagraph{Identity:}
						\begin{align*}
							0 &= (0 + 0\sqrt{2})\\
							(a + b\sqrt{2}) + (0 + 0\sqrt{2}) &= a + b\sqrt{2}\\
							(0 + 0\sqrt{2}) + (a + b\sqrt{2}) &= a + b\sqrt{2}
						\end{align*}

					\subparagraph{Inverse:}
						\begin{align*}
							(a + b\sqrt{2}) + (-a - b\sqrt{2}) &= (0 + 0\sqrt{2}) = 0\\
							(-a - b\sqrt{2}) + (a + b\sqrt{2}) &= (0 + 0\sqrt{2}) = 0
						\end{align*}
					\subparagraph{Abelian:}
						\begin{align*}
							&(a + b\sqrt{2}) + (c + d\sqrt{2})\\
							=& (a+c) + (b+d)\sqrt{2}\\
							=& (c+a) + (d+b)\sqrt{2}\\
							=& (c + d\sqrt{2}) + (a + b\sqrt{2})
						\end{align*}
				\paragraph{$(\mathbb{Q}[\sqrt{2}], \cdot)$ is Associative:}
					\begin{align*}
						&((a + b\sqrt{2}) \cdot (c + d\sqrt{2})) \cdot (e + f\sqrt{2})\\
						=& (ac + (ad + bc)\sqrt{2} + 2bd) \cdot (e + f\sqrt{2})\\
						=& ((ac)e + (ad + bc + (ac)f + (ad)e + (bc)e +2(bd)f)\sqrt{2} +
						2((ad)f + (bc)f + bd + (bd)e))\\
						=& (a(ce) + (ad + bc + a(cf) + a(de) + b(ce) +2b(df))\sqrt{2} +
						2(a(df) + b(cf) + bd + b(de)))\\
						=& (a+b\sqrt{2})(ce + (cf + de)\sqrt{2} + 2df)\\
						=&(a + b\sqrt{2}) \cdot ((c + d\sqrt{2}) \cdot (e + f\sqrt{2}))
					\end{align*}
				
				\paragraph{$(\mathbb{Q}[\sqrt{2}], +, \cdot)$ is distributive:}
					\begin{align*}
						&(a + b\sqrt{2}) \cdot ((c + d\sqrt{2}) + (e + f\sqrt{2}))\\
						=& (a + b\sqrt{2}) \cdot (c + d\sqrt{2} + e + f\sqrt{2})\\
						=& ac + ad\sqrt{2} + ae + af\sqrt{2} + b\sqrt{2}c + 2bd
						+ b\sqrt{2}e + 2bf\\
						=& (ac + ad\sqrt{2} + b\sqrt{2}e + 2bf) + (ae + af\sqrt{2} +
						b\sqrt{2}c + 2bd) \\
						=&(a + b\sqrt{2}) \cdot (c + d\sqrt{2}) + (a + b\sqrt{2}) \cdot (e +
						f\sqrt{2})\\
					\end{align*}

			\paragraph{$(\mathbb{Q}[\sqrt{2}], +, \cdot)$ has unity:}
			\begin{align*}
				(a+b\sqrt{2}) \cdot (1+0\sqrt{2}) &= a+b\sqrt{2}\\
				(1+0\sqrt{2}) \cdot (a+b\sqrt{2}) &= a+b\sqrt{2}\\
			\end{align*}
			Hence $(1+0\sqrt{2})=1$ is the unity.

			\paragraph{$(\mathbb{Q}[\sqrt{2}], +, \cdot)$ has multiplicative inverse:}
			\begin{align*}
				\frac{1}{a+b\sqrt{2}} &= \frac{a-b\sqrt{2}}{(a+b\sqrt{2})(a-b\sqrt{2})}\\
				&= \frac{a-b\sqrt{2}}{a^2 - 2b^2}\\
				&= \frac{a}{a^2 - 2b^2} - \frac{b}{a^2 - 2b^2}\sqrt{2}
			\end{align*}
			Since $\mathbb{Q}$ is a field (therefore closed under addition,
			multiplication, and has multiplicative inverse), the inverse shown above
			is of the form $a + b\sqrt{2}$ where $a, b \in \mathbb{Q}$.
				
			% 3 b
		\item Prove that $\mathbb{Q}[\sqrt{2}]$ is the filed of fractions of
			$\mathbb{Z}[\sqrt{2}]$ ($\mathbb{Q}[\sqrt{2}]$ is proven to be a field in
			part a).
			\paragraph{Define Ring homomorphism:} $\theta: \mathbb{Z}[\sqrt{2}]
			\mapsto \mathbb{Q}[\sqrt{2}]$ by $\theta (a+b\sqrt{2}) =
			\frac{a}{1}+\frac{b}{1}\sqrt{2}$ where $a,b \in \mathbb{Z}$.
			\begin{proof} $\theta$ is a ring homomorphism:\\
				$\theta$ preserves addition:
				\begin{align*}
					\theta \left((a+b\sqrt{2}) + (c+d\sqrt{2})\right) &= \theta
						\left((a+c) + (b+d)\sqrt{2})\right)\\
					&= \frac{a+c}{1} + \frac{b+d}{1}\sqrt{2}\\
					&= \frac{a}{1} + \frac{b}{1}\sqrt{2} + \frac{c}{1} +
						\frac{d}{1}\sqrt{2}\\
					&= \theta (a+b\sqrt{2}) + \theta(c+d\sqrt{2})
				\end{align*}
				$\theta$ preserves multiplication:
				\begin{align*}
					\theta \left((a+b\sqrt{2}) \cdot (c+d\sqrt{2})\right) &= \theta
					((ac+2bd) + (ad + bc)\sqrt{2})\\
					&= \frac{ac+2bd}{1} + \frac{ad + bc}{1}\sqrt{2} \\
					&= \frac{ac}{1} + \frac{ad}{1}\sqrt{2} + \frac{bc}{1}\sqrt{2} +
						\frac{2bd}{1}\\
					&= (\frac{a}{1} + \frac{b}{1}\sqrt{2}) \cdot (\frac{c}{1} +
						\frac{d}{1}\sqrt{2})\\
					&= \theta (a+b\sqrt{2}) \cdot \theta(c+d\sqrt{2})
				\end{align*}
			\end{proof}
			\paragraph{Any element in $\mathbb{Q}[\sqrt{2}]$ has the form
			$\theta(a + b\sqrt{2})\theta(a + b\sqrt{2})^{-1}$ where $a,b \in
			\mathbb{Z}$} Let $\frac{a}{b}+\frac{c}{d}\sqrt{2}$ be a generic element of
			$\mathbb{Q}[\sqrt{2}]$.
				\begin{align*}
					\frac{a}{b}+\frac{c}{d}\sqrt{2} &= \theta(e +
					f\sqrt{2})\theta(g+h\sqrt{2})^{-1}\\
					&= (\frac{e}{1}+\frac{f}{1}\sqrt{2})
						\cdot (\frac{g}{g^2 - 2h^2} - \frac{h}{g^2 - 2h^2}\sqrt{2})\\
					&= \frac{eg}{g^2 - 2h^2} - \frac{eh}{g^2 - 2h^2}\sqrt{2} +
					\frac{fg}{g^2 - 2h^2}\sqrt{2} - \frac{2fh}{g^2 - 2h^2}\\
					&= \frac{eg-2fh}{g^2 - 2h^2} + \frac{fg-eh}{g^2 - 2h^2}\sqrt{2}
				\end{align*}
				Hence shown any element in $\mathbb{Q}[\sqrt{2}]$ has the form
			$\theta(a + b\sqrt{2})\theta(a + b\sqrt{2})^{-1}$ where $a,b \in
			\mathbb{Z}$.
		\end{enumerate}

	% 4
	\item 
		\begin{proof} Show $f: \mathbb{Z}[\sqrt{2}] \mapsto \left\{\begin{bmatrix}a & 2b
				\\ b & a \end{bmatrix} \mid a,b \in \mathbb{Z}\right\}$ by $f(a +
				b\sqrt{2}) = \begin{bmatrix}a & 2b \\ b & a \end{bmatrix}$ is
				isomorphism of rings:\\
				\paragraph{Homomorphism:}
				\subparagraph{Preserve addition:}
				\begin{align*}
					f((a + b\sqrt{2})+(c + d\sqrt{2})) &= f((a+c) + (b+d)\sqrt{2})\\
					&= \begin{bmatrix}a+c & 2(b+d) \\ b+d & a+c \end{bmatrix}\\
					&= \begin{bmatrix}a & 2b \\ b & a \end{bmatrix} + \begin{bmatrix}c &
						2d \\ d & c \end{bmatrix}\\
					&= f(a + b\sqrt{2})+ f(c + d\sqrt{2})
				\end{align*}
				\subparagraph{Preserve multiplication:}
				\begin{align*}
					f((a + b\sqrt{2}) \cdot (c + d\sqrt{2})) &= f((ac+2bd) + (ad + bc)\sqrt{2})\\
					&= \begin{bmatrix}ac+2bd & 2(ad + bc) \\ ad + bc & ac+2bd \end{bmatrix}\\
					&= \begin{bmatrix}a & 2b \\ b & a \end{bmatrix} \cdot \begin{bmatrix}c &
						2d \\ d & c \end{bmatrix}\\
					&= f(a + b\sqrt{2}) \cdot f(c + d\sqrt{2})
				\end{align*}

				\paragraph{Injective:} Define inverse of $f$: 
				\begin{align*}
				f^{-1}: \left\{\begin{bmatrix}a & 2b
				\\ b & a \end{bmatrix} \mid a,b \in \mathbb{Z}\right\} &\mapsto \mathbb{Z}[\sqrt{2}]\\
				f^{-1}\left(\begin{bmatrix}a & 2b \\ b & a \end{bmatrix}\right) &= a +
				b\sqrt{2} 
				\end{align*}
				This is well defined because $f^{-1}$ maps every element of the domain
				to a single value in the codomain.\\
				Hence shown $f: \mathbb{Z}[\sqrt{2}] \mapsto \left\{\begin{bmatrix}a & 2b
				\\ b & a \end{bmatrix} \mid a,b \in \mathbb{Z}\right\}$ by $f(a +
				b\sqrt{2}) = \begin{bmatrix}a & 2b \\ b & a \end{bmatrix}$ is
				isomorphism of rings.
		\end{proof}

		% 5
		\item Suppose $A$ is a unital commutative ring of characteristic $p>0$ where p
			is prime.
			\begin{proof} Show that $\forall x, y \in A, (x+y)^p=x^p + y^p$
				\begin{align*}
					(x+y)^p = \sum_{i=0}^{p} {p \choose i} x^iy^{p-i}
				\end{align*}
				Since $A$ is of order $p$, $a\in A, pa = 0$. Since we know ${p \choose
				0}$ and ${p \choose p}$ are $1_A$, it is suffice to show $p \mid {p \choose
				i}, i \in (0, p)$, since that would result in $1_A \cdot x^p + 0 + \dots + 0 +
				1_A \cdot y^p = x^p + y^p$.
				\begin{align*}
					p \choose i &= \frac{p!}{i!(p-i)!} \\
					&= \frac{p \cdot (p-1) \cdot \dots (p-i+1)} {i!}
				\end{align*}
				Since $p$ is prime, and $i < p$, none of the factors in $i!$ can divide
				$p$. However, ${p \choose i}$ is an integer, therefore $p$ must be a
				factor in ${p \choose i}$, hence $p divide {p \choose i}$.\\
				Hence shown given $A$ is a unital commutative ring of characteristic
				$p>0$ where p is prime $\forall x, y \in A, (x+y)^p=x^p + y^p$.
			\end{proof}

		% 6
		\item
			\begin{enumerate}
				% 6 a
				\item Find a zero-divisor in $\mathbb{Z}_5[i] = \{a + bi \mid a,b \in
					\mathbb{Z}_5\}$
					\begin{align*}
						(a + bi)(c + di) = (ac - bd) + (ad + bc)i = 0 \text{ in }
						\mathbb{Z}_5\\
						ac &\overset{5}{\equiv} bd\\
						ad &\overset{5}{\equiv} - bc
					\end{align*}
					$(3 + i)\cdot(4 + 2i) = (12 - 2) + (6 + 4)i = 10 + 10i
					\overset{5}{\equiv} 0$

				% 6 b
				\item Show that $x^2 + 1$ has no zero in $\mathbb{Z}_7$. 
					\begin{center}
						\begin{tabular}{ c|c c }
							$x$ & $x^2$ & $x^2 + 1$ \\
							\hline
							$0$ & $0$   & $1$ \\
							$1$ & $1$   & $2$ \\
							$2$ & $4$   & $5$ \\
							$3$ & $2$   & $3$ \\
							$4$ & $2$   & $3$ \\
							$5$ & $4$   & $5$ \\
							$6$ & $1$   & $2$ \\
						\end{tabular}
					\end{center}
					There is no $0$ in $\mathbb{Z}_7$.

				% 6 c
				\item Show that if either $a \neq 0$ or $a \neq 0$ in $\mathbb{Z}_7$,
					then $a^2 + b^2 \neq 0$.\\
					Since addition is commutative, without loss of generality, it is
					sufficient to show if $a \neq 0$, then $a^2+b^2 =
					a^2(1+\frac{b}{a}^2)$. By part b, we know $1+\frac{b}{a}^2 \neq 0$.
					Since we assume $a \neq 0 \Rightarrow a^2 \neq 0$, we conclude $a^2 +
					b^2 \neq 0$.

				% 6 d
				\item Show that $\mathbb{Z}_7[i] = \{a + bi \mid a, b \in
					\mathbb{Z}_7\}$ is a field. \\
					Since $7$ is prime, we know $\mathbb{Z}_7$ is a field, therefore we
					can conclude that $\mathbb{Z}_7[i]$ has multiplicative inverse.
					\paragraph{Integral domain:} Show no-zero divisors:
					\begin{align*}
						(a+bi)(c+di) &= 0\\
						\Rightarrow (a+bi)(a-bi)(c-di) &= 0\\
						\Rightarrow (a^2+b^2)(c^2+d^2) &= 0 \text{ in } \mathbb{Z}_7\\
					\end{align*}
					By part c, we know either $a$ or $b$ and $c$ or $d$ are $\neq 0$, then
					the product is not zero. This means either $a+bi = 0$ or $c+d = 0$.
			\end{enumerate}

\end{enumerate}

\end{document}

