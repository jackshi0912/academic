\documentclass[12pt]{article}

\usepackage[margin=1in]{geometry}
\usepackage{amsmath, amsthm, amsfonts, mathtools}
\usepackage{enumitem}
\usepackage{dsfont}

\begin{document}

\null\hfill\begin{tabular}[t]{l@{}}
	\textbf{Name: }Huize Shi - A92122910 \\
	\textbf{Discussion: }A04 \\
	\textbf{Homework: }1
\end{tabular}
\noindent\rule{\textwidth}{0.5pt}

\begin{enumerate}
	% 1
	\item Suppose $R_1, \dots, R_n$ are rings. Prove that $R_1, \dots, R_n$ are
		unital if and only if $R_1 \times, \dots, \times R_n$ is unital.
		\begin{proof} $(=>)$:
			Assume $R_1 \times, \dots, \times R_n$ is unital, wants to show $R_1,
			\dots, R_n$ are unital.
			\begin{align*}
				&R_1 \times, \dots, \times R_n \text{ are unital } \Rightarrow \exists
				\text{ unity }(\mathds{1}_1, \dots, \mathds{1}_n)\\
				\Rightarrow &(\mathds{1}_1, \dots, \mathds{1}_n) \cdot (r_1, \dots, r_n)
				= (r_1, \dots, r_n) \forall r_i \in R_i,\ i\in \mathbb{Z},\ 1 \le i \le n \\
				\Rightarrow &(\mathds{1}_1 \cdot r_1, \dots, \mathds{1}_n \cdot r_n) 
				= (r_1, \dots, r_n) \forall r_i \in R_i,\ i\in \mathbb{Z},\ 1 \le i \le n \\
				\Rightarrow &(r_1, \dots, r_n) \cdot (\mathds{1}_1, \dots, \mathds{1}_n) 
				= (r_1, \dots, r_n) \forall r_i \in R_i,\ i\in \mathbb{Z},\ 1 \le i \le
				n \\
				\Rightarrow &(r_1 \cdot \mathds{1}_1, \dots, r_n \cdot \mathds{1}_n) 
				= (r_1, \dots, r_n) \forall r_i \in R_i,\ i\in \mathbb{Z},\ 1 \le i \le n \\
				\Rightarrow &r_i \cdot \mathds{1}_i = \mathds{1}_i \cdot r_i = r_i,\
				\forall r_i \in R_i,\ i\in \mathbb{Z},\ 1 \le i \le n
			\end{align*}
			This shows that $\mathds{1}_i$ is the unity for each ring $R_i$. Hence
			$R_1 \times, \dots, \times R_n$ are unital rings. \\
			
			$(<=)$: Assume $R_1, \dots, R_n$ are unital, wants to show $R_1 \times,
			\dots, \times R_n$ is unital. By assumption, $\exists\ \mathds{1}_1 \dots
			\mathds{1}_n$, unity for each ring $R_1, \dots, R_n$. Wants to show
			$(\mathds{1}_1 \dots \mathds{1}_n)$ is the unity of $R_1 \times, \dots,
			\times R_n$.
			\begin{align*}
				(\mathds{1}_1, \dots, \mathds{1}_n) \cdot (r_1, \dots, r_n)
				&=(\mathds{1}_1 \cdot r_1, \dots, \mathds{1}_n \cdot r_n) \\
				&=(r_1, \dots, r_n)\forall r_i \in R_i,\ i\in \mathbb{Z},\ 1 \le i \le n
			\end{align*}
			\begin{align*}
				(r_1, \dots, r_n) \cdot (\mathds{1}_1, \dots, \mathds{1}_n)
				&=(r_1 \cdot \mathds{1}_1 , \dots, r_n \cdot \mathds{1}_n) \\
				&=(r_1, \dots, r_n)\forall r_i \in R_i,\ i\in \mathbb{Z},\ 1 \le i \le n
			\end{align*}
		Hence shown $(\mathds{1}_1, \dots, \mathds{1}_n)$ is the unity of $R_1 \times, \dots,
			\times R_n$.
		\end{proof}

	% 2
	\item Suppose $R$ is a unital ring. An element x of $R$ is called a unit if it
		has a multiplicative inverse. Let $\mathcal{U}(R)$ be the set of all the
		units of $R$.
		\begin{enumerate}
			% a
			\item Prove that $\mathcal{U}(R)$ is closed under multiplication.\\
				Given $u_1, u_2 \in \mathcal{U}(R)$, wants to show $u_1u_2 \in
				\mathcal{U}(R)$. Since $u_1, u_2 \in \mathcal{U}(R)$, $u_1^{-1},
				u_2^{-1} \in \mathcal{U}(R)$. This means the inverse of $u_1u_2$ exists
				$(u_1u_2)^{-1} = u_2^{-1}u_1^{-1}$. This means that $u_1u_2$ has
				multiplicative inverse, therefore in $\mathcal{U}(R)$. Hence shown
				$\mathcal{U}(R)$ is closed under multiplication.

			% b
			\item Prove that $(\mathcal{U}(R), \cdot)$ is a group.
				\paragraph{Associativity}$\cdot$ operator is associative by definition
				of ring.
				\paragraph{Identity}Since the ring is unital, and the inverse of the
				unity is itself, the unity is in $\mathcal{U}(R)$.
				\paragraph{Inverse}By the definition of $\mathcal{U}(R)$ all elements
				have inverse under multiplication.

			% c
			\item Suppose $R_i$ are unital rings. Prove that $\mathcal{U}(R_1 \times,
				\dots, \times R_n) = \mathcal{U}(R_1) \times \dots \times \mathcal{U}(R_n)$
				Let $r_i$ be any element in $R_i$ where $i \in \mathbb{Z}, 1 \le i \le
				n$.
				\begin{align*}
					\mathcal{U}(R_1\times \dots \times R_n) = (u_1, \dots, u_n) \text{
					such that } \exists (u_1', \dots, u_n'), \\
					(u_1, \dots, u_n)(u_1', \dots, u_n') = (u_1', \dots, u_n')(u_1,
					\dots, u_n) = (\mathds{1}_1, \dots, \mathds{1}_n)
				\end{align*}
				By the definition of element wise multiplication:
				\begin{align*}
					(u_1, \dots, u_n)(u_1', \dots, u_n') &= (u_1u_1', \dots, u_nu_n') 
					= (\mathds{1}_1, \dots, \mathds{1}_n)\\
					(u_1', \dots, u_n')(u_1, \dots, u_n) &= (u_1'u_1, \dots, u_n'u_n)
					= (\mathds{1}_1, \dots, \mathds{1}_n)
				\end{align*}
				This shows that each element $u_i$ has multiplicative inverse. This
				means that $u_i$ is in $\mathcal{U}(R)$. since $u_i$ is a general term
				for $\mathcal{U}(R_i)$, this shows that $\mathcal{U}(R_1 \times, \dots,
				\times R_n) = \mathcal{U}(R_1) \times \dots \times \mathcal{U}(R_n)$.
			
			% d
			\item Find $\mathcal{U}(\mathbb{Z} \times \mathbb{Q})$ \\
				By part c, $\mathcal{U}(\mathbb{Z} \times \mathbb{Q}) =
				\mathcal{U}(\mathbb{Z}) \times \mathcal{U}(\mathbb{Q})$
				Since the only integers with multiplicative inverse in $\mathbb{Z}$ are
				$\pm1$, and $(Q, +, \cdot)$ is a field, $\mathcal{U}(\mathbb{Z} \times
				\mathbb{Q}) = (\pm1, \mathbb{Q})$
		\end{enumerate}
	
	% 3
	\item Show that $\{a+b\sqrt{3} \mid a,b\in\mathbb{Z}\}$ is ring.\\
		\paragraph{Group portion}First show $(\{a+b\sqrt{3} \mid a,b\in\mathbb{Z}\},
		+)$ is a abelian group.
		\subparagraph{Associativity}
		\begin{align*}
			&(a+b\sqrt{3}) + ((a'+b'\sqrt{3}) + (a''+b''\sqrt{3}))\\
		 =&(a+b\sqrt{3}) + (a'+b'\sqrt{3}) + (a''+b''\sqrt{3}) \\
		 =&((a+b\sqrt{3}) + (a'+b'\sqrt{3})) + (a''+b''\sqrt{3})
		\end{align*}
	
		\subparagraph{Identity}$0 + 0\sqrt{3}=0$, $0 + a = a+0 = a$. Identity
		exists in the set under $+$.

		\subparagraph{Inverse}$(a+b\sqrt{3})^{-1}=-a-b\sqrt{3}$, since
		$a+b\sqrt{3} + (-a-b\sqrt{3}) = a-a + b\sqrt{3} - b\sqrt{3} = 0$. The
		inverse exists in the set under $+$.

		\subparagraph{Abelian}$(a+b\sqrt{3}) + (a'+b'\sqrt{3}) = a+b\sqrt{3} +
		a'+b'\sqrt{3} = a'+b'\sqrt{3} + a+b\sqrt{3} = (a'+b'\sqrt{3}) +
		(a+b\sqrt{3})$. Hence shown the group is abelian.

		\paragraph{Multiplication associativity}
		\begin{align*}
			&(a+b\sqrt{3}) \cdot ((a'+b'\sqrt{3}) \cdot (a''+b''\sqrt{3})) \\
			=&(a + b\sqrt{3}) \cdot (a'a'' + a'b''\sqrt{3} + b'\sqrt{3}a'' +
			b'\sqrt{3}b''\sqrt{3}) \\
			=&aa'a'' + aa'b''\sqrt{3} + ab'\sqrt{3}a'' + ab'\sqrt{3}b''\sqrt{3} \\
			&+ b\sqrt{3}a'a'' + b\sqrt{3}a'b''\sqrt{3} + b\sqrt{3}b'\sqrt{3}a'' 
			+ b\sqrt{3}b'\sqrt{3}b''\sqrt{3})\\
			=&(aa' + ab'\sqrt{3}+b\sqrt{3}a' + b\sqrt{3}b'\sqrt{3})(a'' +
			b''\sqrt{3})\\
			=&((a+b\sqrt{3}) \cdot (a'+b'\sqrt{3})) \cdot (a''+b''\sqrt{3})
		\end{align*}

		\paragraph{Distributive property}
		\begin{align*}
			&(a+b\sqrt{3}) \cdot ((a'+b'\sqrt{3}) + (a''+b''\sqrt{3})) \\
			=&(a+b\sqrt{3}) \cdot (a'+b'\sqrt{3} + a''+b''\sqrt{3}) \\
			=&aa'+ab'\sqrt{3} + aa''+b''\sqrt{3} + b\sqrt{3}a'+b\sqrt{3}b'\sqrt{3} +
			b\sqrt{3}a''+b''\sqrt{3} \\
			=&(a+b\sqrt{3}) \cdot (a'+b'\sqrt{3}) + (a+b\sqrt{3}) \cdot (a''+b''\sqrt{3}) \\
		\end{align*}

	% 4
	\item As in problem 3, one can show $F=\{a+b\sqrt{3} \mid a,b\in\mathbb{Q}\}$
		is a ring. Show that $\mathcal{U}(F) = F \setminus \{0\}$; that means any
		non-zero element is a unit.
		\begin{proof}
			Given $a + b\sqrt{3}$, define $(a + b\sqrt{3})^{-1}$ as $\frac{a -
				b\sqrt{3}}{aa - 3bb}$. Wants to show the inverse is an element of the ring.
			\begin{align*}
				(a + b\sqrt{3})\cdot \frac{a - b\sqrt{3}}{aa - 3bb} &= \frac{a -
					b\sqrt{3}}{aa - 3bb}\cdot(a + b\sqrt{3}) =\frac{aa - 3bb}{aa - 3bb} =
					1\\
					\frac{a - b\sqrt{3}}{aa - 3bb} &= \frac{a}{aa-3bb} - \frac{b}{aa-3bb}
					\cdot \sqrt{3}\\
			\end{align*}
			This is of the form $a + b\sqrt{3}$ since $(\mathbb{Q}, +, \cdot)$ is a
			field. It contains inverses for all elements and is closed under addition 
			(denominators are therefore rational). Since $\mathbb{Q}$ is a field,
			$\frac{1}{aa - 3bb}$ is rational since it is the multiplicative inverse of
			$aa - 3bb$ which previously explained to be rational. Hence any non-zero
			element of F is a unit since $\frac{a - b\sqrt{3}}{aa - 3bb}$ is
			demonstrated to be the multiplicative inverse of any element in $F$.
		\end{proof}

	% 5
	\item For a ring $R$, let $R[x] = \{a_0 + a_1x + \dots + a_nx^n \mid a_0,
		\dots, a_n \in \mathcal{R}, n \in \mathbb{Z}^{\ge 0}\}$ be the ring of
		polynomials with coefficients in $R$ and indeterminant x. We add and
		multiply polynomials as usual.
		\begin{enumerate}
			% a
			\item Show that $\mathcal{U}(\mathbb{Z}[x]) = \{\pm 1\}$ \\
				Given $z_x = a_0 + a_1x + \dots + a_nx^n$, it's inverse 
				$z_x^{-1} = a_0' + a_1'x + \dots + a_n'x^n$.
				\begin{align*}
					z_xz_x' = \sum^n_{i=0}\sum^n_{j=0} a_ia_j'x^{i+j} = (1,0, 0, \dots 0)
				\end{align*}
				Assume towards a contradiction that $z_x$ and $z_x'$ are not $\pm 1$.
				This is impossible because the $x$ terms cannot be cancled. Hence it can
				only be the case if the polynomial is $(1, 0, \dots, 0)$ or $(-1, 0,
				\dots, 0)$.

			% b
			\item Show that $2x+1 \in \mathcal{U}(\mathbb{Z}_8[x])$

		\end{enumerate}
	
	% 6
	\item Suppose A is a ring with unity $1$. Suppose there is $a_0 \in A$ such
		that $a^2_0 = 1$. Let $B:=\{a_0 r a_0 \mid r \in A\}$. Prove that $B$ is a
		subring of $A$.\\
		\paragraph{Subtraction}Given any $r$ and $r'$ in $A$, consider $a_0ra_0 -
		a_0r'a_0$:
		\begin{align*}
			a_0ra_0 - a_0r'a_0 = a_0(r-r')a_0
		\end{align*}
		Since $r, r' \in A$, A is a ring, $r-r'$ is also in $A$. Hence the first
		condition is satisfied.
		\paragraph{Multiplication}Given any $r$ and $r'$ in $A$, consider
		$(a_0ra_0) \cdot (a_0r'a_0)$. Since multiplication is associative the
		following holds:
		\begin{align*}
			(a_0ra_0) \cdot (a_0r'a_0) &= a_0r(a_0\cdot a_0)r'a_0 \\
																 &= a_0r\cdot 1 \cdot r'a_0 \\
																 &= a_0rr'a_0 \\
		\end{align*}
		Since $r, r' \in A$, A is a ring, $rr'$ is also in $A$. Hence the second
		condition is satisfied.\\
		Hence shown $B\le A$.

\end{enumerate}

\end{document}

