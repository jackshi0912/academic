\documentclass[12pt]{article}

\usepackage[margin=1in]{geometry}
\usepackage{amsmath, amsthm, amsfonts, mathtools}
\usepackage{enumitem}

\begin{document}

\null\hfill\begin{tabular}[t]{l@{}}
	\textbf{Name: }Huize Shi - A92122910 \\
	\textbf{Discussion: }A04 \\
	\textbf{Homework: }3
\end{tabular}
\noindent\rule{\textwidth}{0.5pt}

\begin{enumerate}
	% 1
	\item
	\begin{enumerate}
		% 1.a
		\item Show that $\mathbb{Z}[w]=\{a+bw \mid a,b \in \mathbb{Z}\}$ is a
			subring of $\mathbb{C}$ where $w = \frac{-1 + \sqrt{-3}}{2}$

			\paragraph{Addition:}
				\begin{align*}
					&\left(a+b\left(\frac{-1 + \sqrt{-3}}{2}\right)\right)-
					\left(c+d\left(\frac{-1 + \sqrt{-3}}{2}\right)\right)\\
					=& \left((a-c)+(b+d)\left(\frac{-1 + \sqrt{-3}}{2}\right)\right)
				\end{align*}
			\paragraph{Multiplication:}
				\begin{align*}
					&\ \ \left(a+b\left(\frac{-1 + \sqrt{-3}}{2}\right)\right) \cdot
						\left(c+d\left(\frac{-1 + \sqrt{-3}}{2}\right)\right)\\
					&= ac + (ad+bc)\left(\frac{-1 + \sqrt{-3}}{2}\right) +
						bd\left(\frac{-1 + \sqrt{-3}}{2}\right)^2\\
					&= (ac - \frac{ad}{2} - \frac{bc}{2} - \frac{bd}{2}) +
						(ad + bc + bd)\frac{\sqrt{-3}}{2}\\
					&= ac - (ad + bc + bd)\frac{1}{2} + (ad + bc +
						bd)\frac{\sqrt{-3}}{2}\\
					&= ac + (ad + bc +
						bd)\frac{-1 + \sqrt{-3}}{2}
				\end{align*}

				Hence shown $\mathbb{Z}[w]$ is a subring of $\mathbb{C}$.\\

		% 1.b
		\item Show that the field of fraction of $\mathbb{Z}[w]$ is
			$\mathbb{Q}[w]=\{a+bw \mid a,b \in \mathbb{Q}\}$.
			\paragraph{Multiplicative inverse (field):}From part a, we can use
			similar process to show that $\mathbb{Q}$ is a subring of $\mathbb{C}$.
			Show there is multiplicative inverses. Let $\hat{w}=\frac{-1 -
				\sqrt{-3}}{2}$
			\begin{align*}
				\frac{1}{a+bw} &= \frac{a+b\hat{w}}{(a+bw)(a+b\hat{w})}\\
 											 &= \frac{a+b\hat{w}}{a^2 - ab + b^2}
			\end{align*}
			\begin{align*}
 				(a+bw)\frac{a+b\hat{w}}{a^2 - ab + b^2} &= \frac{a^2 - ab + b^2}{a^2 - ab + b^2}\\
																								&= 1
			\end{align*}
			Hence we found the inverse of the general term $a+bw$ is therefore
			$\frac{a+b\hat{w}}{a^2 - ab + b^2}$.

			\paragraph{Ring homomorphism:}
			\begin{align*}
				\theta: \mathbb{Z}[w] &\mapsto \mathbb{Q}[w]\\
				\theta(x) &= x
			\end{align*}
			The homomorphism property of identity mapping is trivial. Given a generic
			element in $\mathbb{Q}[w]$, show it is of the form
			$\theta(x)\theta(x)^{-1}$.
			\begin{align*}
				\frac{a}{b} + \frac{c}{d}w &= \frac{ad+bcw}{bd}\\
				&= \theta(ad+bcw)\theta(bd)^{-1}
			\end{align*}

	\end{enumerate}

	%2
	\item
		\begin{enumerate}
			% 2 a
			\item Prove that $<u> = R$ iff $u \in \mathcal{U}$\\
				By definition, $<u> = \{ru \mid r \in R\}$. Since R is a unital commutative
				ring, $1_R \in \{ru \mid r \in R\}$. Assume towards contrary, $u$ is not a
				unit, this means $ru \neq 1_R$. This is a contradiction, hence $u$ must be a
				unit.
				
			% 2 b
			\item Suppose $D$ is an integral domain. Prove that $<a> = <b>
				\Leftrightarrow \exists u\in \mathcal{U}(D) \mid a=bu$ Since both $a,b$
				generated the set, we can represent any element in the set as $bu$ or
				$au'$ This means the following holds:
				\begin{align*}
					a &= bu\\
					b^{-1}aa^{-1} &= b^{-1}bua^{-1}\\
					b^{-1} &= ua^{-1}\\
					b &= u^{-1}a\\
					b &= au'\\
					u^{-1}a &= au'\\
					a^{-1}a &= u'u\\
					u'u &= 1
				\end{align*}
				Hence shown $\exists u\in \mathcal{U}(D) \mid a=bu$.
		\end{enumerate}

	% 3
	\item Given $R_1$ and $R_2$ are unital commutative rings, and $I \lhd R_1
		\times R_2$. Prove that $I = I_1 \times I_2$.
		\begin{align*}
			(x_1, x_2) &\in I\\
			(x_1, x_2)\cdot(r_1, r_2) &\in I\\
			(x_1, x_2)\cdot(r_1, r_2) &= (x_1r_1, x_2r_2) \\
			x_1r_1 &\in I_1\\
			x_2r_2 &\in I_2\\
			(x_1r_1, x_2r_2) &= I_1 \times I_2\\
			I &= I_1 \times I_2
		\end{align*}

	% 4
	\item Prove that $<2,x> \lhd \mathbb{Z}[x]$ is not a principal ideal.
		\paragraph{Constant polynomial:}Assume $<f(x)>$ is constant polynomial,
		the generated set contains the polynomials with even coefficient (multiple
		of 2).
		\paragraph{Degree 1:}Assume $<f(x)>$ is polynomial of degree at least 1, the
		generated set contains only polynomial with no even constant term including
		2.\\
		Hence shown $<f(x)> \neq <2,x>$.

	% 5
	\item
		\begin{enumerate}
			% 5 a
			\item Find the remainder of $102459087$ divided by $9$:
				\begin{align*}
					102459087 &\overset{9}{\equiv} 1+0+2+4+5+9+0+8+7\\
										&\overset{9}{\equiv} 36 \overset{9}{\equiv} 0
				\end{align*}

			% 5 b
			\item Find the remainder of $102459087$ divided by $11$:
					\begin{align*}
						102459087 &\overset{11}{\equiv} 1-0+2-4+5-9+0-8+7\\
											&\overset{11}{\equiv} -6 \overset{11}{\equiv} 5
					\end{align*}

			% 5 c
			\item 

		\end{enumerate}

	% 6
	\item
		\begin{align*}
			\hat{a} &\overset{5}{\equiv} a\\
			\hat{b} &\overset{5}{\equiv} b\\
			f(a+bi) &= \hat{a} \oplus 2\hat{b}\\
							&\overset{5}{\equiv} \hat{a} + 2\hat{b}\\
							&\overset{5}{\equiv} a + 2b
		\end{align*}

		\begin{enumerate}
			% 6 a
			\item Proove homomorphism:
				\paragraph{Addition:} 
				\begin{align*}
				&f((a+bi) + (c+di)) \\
				&= f((a+c) + (b+d)i) \\
				&= \hat{(a+c)} + 2 \hat{(b+d)}\\
				&\overset{5}{\equiv} (a+c) + 2(b+d)\\
				&\overset{5}{\equiv} (a+ 2b) + (c + 2d)\\
				&= \hat{a} + 2 \hat{b} + \hat{c} + 2 \hat{d}\\
				&= f(a+bi) + f(c+di)
				\end{align*}

				\paragraph{Multiplication:}
				\begin{align*}
				&f((a+bi) \cdot (c+di)) \\
				&= f(ac-bd + (ad + cb)i)\\
				&= \hat{(ac-bd)} + 2\hat{(ad + cb)}\\
				&\overset{5}{\equiv}(ac-bd) + 2(ad + cb)\\
				&\overset{5}{\equiv}(a+2b)\cdot(c+2d)\\
				&f(a+bi) \cdot f(c+di)
				\end{align*}

			% 6 b
			\item Show that $<-2 + i> \subseteq ker\ f$ \\
				$f(-2 + i) = -2 + 2 = 0$. $-2 + i$ is in kernel of f. This means the set
				generated by $-2 + i$ is also in kernel of f.

		\end{enumerate}

\end{enumerate}

\end{document}

