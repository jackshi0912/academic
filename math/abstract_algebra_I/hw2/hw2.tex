\documentclass{article}
\usepackage{amsmath, amssymb, amsthm}
\usepackage{scrextend}
\title{Abstract Algebra: Homework 2}
\author{Huize Shi - A92122910}
\date{Thursday January 25, 2018}

\begin{document}
\maketitle

\section*{Section 3}
\subsection*{4.}
	Let $a,\ b \in \mathbb{Z},\ a\neq b$. 
	\begin{flalign*}
		\phi(a+b) &= a+b+1&\\
		\phi(a) + \phi(b) &= (a+1)+(b+1)&\\
		a+b+1 &\neq a+ b+2&\\
		\phi(a+b) &\neq \phi(a) + \phi(b)&
	\end{flalign*}
	$\phi$ is not isomorphic.

\subsection*{17.}
	\subsubsection*{a.}
	\begin{addmargin}[1em]{1em}
		Let $a,\ b \in \mathbb{Z}$. 
		\begin{flalign*}
			a*b &= \phi(a-1) * \phi(b-1)&\\
			&= \phi((a-1)\cdot(b-1))&\\
			&=\phi(ab-a-b+1)&\\
			&=ab-a-b+2&
		\end{flalign*}
		$*:\mathbb{Z}\times\mathbb{Z}\mapsto\mathbb{Z}$ by $*(a, b)=ab-a-b+2$\\
		Identity element for $\langle \mathbb{Z}, *\rangle$ is $\phi(1) = 2$
		\begin{proof} $\phi$ is an isomorphism:\\
			Let $a,\ b \in \mathbb{Z}$. 
			\begin{align*}
				\phi(a\cdot b) &\stackrel{?}{=} \phi(a) * \phi(b)&\\
				ab+1 &\stackrel{?}{=} (a+1)(b+1)-(a+1)-(b+1)+2&\\
				ab+1 &\stackrel{?}{=} (a+1)(b+1)-(a+1)-(b+1)+2&\\
				ab+1 &\stackrel{?}{=} ab+a+b+1-a-1-b-1+2&\\
				ab+1 &= ab+1&
			\end{align*}
			Hence shown $\phi(a\cdot b) = \phi(a) * \phi(b)$, $\phi$ is an isomorphism
			of $\langle \mathbb{Z},\ \cdot\rangle$ with $\langle \mathbb{Z},\ *\rangle$.
		\end{proof}
	\end{addmargin}

	\subsubsection*{b.}
	\begin{addmargin}[1em]{1em}
		Let $a,\ b \in \mathbb{Z}$.
		\begin{flalign*}
			\phi^{-1}(n) = n-1
			a*b &= \phi^{-1}(a+1) * \phi(b+1)&\\
			&= \phi^{-1}((a+1)\cdot(b+1))&\\
			&=\phi^{-1}(ab+a+b+1)&\\
			&=ab+a+b&
		\end{flalign*}
		$*:\mathbb{Z}\times\mathbb{Z}\mapsto\mathbb{Z}$ by $*(a, b)=ab+a+b$\\
		Identity element for $\langle \mathbb{Z}, *\rangle$ is $\phi^{-1}(1) = 0$
		\begin{proof} $\phi$ is an isomorphism:\\
			Let $a,\ b \in \mathbb{Z}$.
			\begin{align*}
				\phi(a*b) &\stackrel{?}{=} \phi(a) \cdot \phi(b)&\\
				\phi(ab+a+b) &\stackrel{?}{=} (a+1) \cdot (b+1)&\\
				ab+a+b+1 &= ab+a+b+1&
			\end{align*}
			Hence shown $\phi(a* b) = \phi(a) \cdot \phi(b)$, $\phi$ is an isomorphism
			of $\langle \mathbb{Z},\ *\rangle$ with $\langle \mathbb{Z},\ \cdot\rangle$.
		\end{proof}
		
	\end{addmargin}

\subsection*{26.}
	\begin{proof} Assume $\phi:S\mapsto S'$ is an isomorphism of $\langle S, *\rangle$
		with $\langle S', *'\rangle$. Wants to show $\phi^{-1}$ is an isomorphism of 
		$\langle S', *'\rangle$ with $\langle S, *\rangle$.\\
		Let $a,\ b \in S$, such that $\phi(a)=a'$, $\phi(b)=b'$, for some
		$a',\ b' \in S'$.
		\begin{align*}
			\phi(a*b) &= \phi(a) *' \phi(b) &\\
			\phi(a*b) &= a' *' b' &\\
			\phi^{-1}(\phi(a*b)) &= \phi^{-1}(a' *' b') &\\
			a*b &= \phi^{-1}(a' *' b') &\\
			\phi^{-1}(a')*\phi^{-1}(b') &= \phi^{-1}(a' *' b') &
		\end{align*}
		By definition of isomorphism, $\phi^{-1}$ is an isomorphism of $\langle S',
		*'\rangle$ with $\langle S, *\rangle$.
	\end{proof}

\subsection*{27.}
	\begin{proof} Assume $\phi:S\mapsto S'$ is an isomorphism of $\langle S, *\rangle$
		with $\langle S', *'\rangle$ and $\psi:S'\mapsto S''$ is an isomorphism of
		$\langle S', *'\rangle$ with $\langle S'', *''\rangle$. Wants to show 
		$\psi \circ \phi$ is an isomorphism of $\langle S, *\rangle$ with $\langle
		S'', *''\rangle$.\\
		Let $a,\ b \in S$, such that $\phi(a)=a'$, $\phi(b)=b'$, for some
		$a',\ b' \in S'$, $\psi(a') = a''$, $\psi(b')=b''$, for some
		$a'',\ b'' \in S''$.
		\begin{align*}
			\phi(a*b) &= \phi(a) *' \phi(b) &\\
								&= a' *' b' &\\
			\psi(a'*' b') &= \psi(a') *'' \psi(b') &\\
										&= a'' *'' b'' &\\
			\psi(\phi(a*b)) &= \psi(a'*' b') = a'' *'' b''&
		\end{align*}
		Hence shown $\psi \circ \phi$ is an isomorphism of $\langle S, *\rangle$ with $\langle
		S'', *''\rangle$.
	\end{proof}

\subsection*{30.}
\begin{proof} ``The operator * is associative.'' is a structural property of a
	binary structure $\langle S',\ *'\rangle$.\\
	Let $\phi: S \mapsto S'$ be an isomorphism of $\langle S,\ *\rangle$ with $\langle S',\
	*'\rangle$.\\
	Let $a,\ b,\ c \in S$, such that $\phi(a)=a'$, $\phi(b)=b'$, $\phi(c)=c'$, for some
	$a',\ b',\ c' \in S'$.

	\begin{align*}
		((a' *' b') *' c') &= (\phi(a) *' \phi(b)) *' \phi(c) &\\
											 &= \phi(a*b) *' \phi(c)&\\
											 &= \phi((a * b) * c))&\\
											 &= \phi(a*(b*c))&\\
											 &= \phi(a) *' \phi(b*c)&\\
											 &= a' *' (\phi(b)*'\phi(c))&\\
											 &= (a' *' (b'*' c'))&
	\end{align*}
	Hence proved that ``the operator * is associative'' is a structural property of a
	binary structure $\langle S',\ *'\rangle$.\\
\end{proof}

\subsection*{32.}
\begin{proof} ``There exists an element b in S such that $b * b = b$'' is a structural 
	property of a binary structure $\langle S',\ *'\rangle$.\\
	Let $\phi: S \mapsto S'$ be an isomorphism of $\langle S,\ *\rangle$ with $\langle S',\
	*'\rangle$.\\
	Assume $\exists b \in S$ such that $b*b=b$. 
	\begin{align*}
		b*b &= b&\\
		\phi(b*b) &= \phi(b)&\\
		\phi(b)*'\phi(b) &= \phi(b)&\\
		b' *' b' &= b'&
	\end{align*}
	Hence proved that ``There exists an element b in S such that $b * b = b$'' is a structural 
	property of a binary structure $\langle S',\ *'\rangle$.
\end{proof}

\section*{Section 4}
\subsection*{4.}
	$\langle \mathbb{Q},\ * \rangle$ is a group structure.
	\paragraph{$\mathcal{G}_1$: Associativity} For $a,\ b,\ c\in \mathbb{Q}$,
	$a*(b*c) = a(bc) = (ab)c = (a*b)*c$.
	\paragraph{$\mathcal{G}_2$: Left identity element} For $a \in 
	\mathbb{Q}$, $1 * a = a$, 1 is the left identity element.
	\paragraph{$\mathcal{G}_3$: $\forall a \in \mathbb{Q},\ 
	\exists a^{-1},\ a^{-1}*a=e$, e is the identity}
	Let $\frac{a}{b} \in \mathbb{Q}$, $\frac{b}{a} * \frac{a}{b}=\frac{b}{a}
	\cdot \frac{a}{b}=1$. Since 1 is the identity, the inverse $\frac{b}{a}$ 
	exist $\forall \frac{a}{b}\in\mathbb{Q}$.

\subsection*{19.}
\subsubsection*{a.}
	\begin{addmargin}[1em]{1em}
		Show that $*:S\times S\mapsto S$\\
		Let $a,\ b\in (\mathbb{R} \setminus \{-1\})$, $a+b+ab \in \mathbb{R}$
		because $\mathbb{R}$ is closed under addition and multiplication.\\
		Check if {-1} is in image of *: 
		\begin{align*}
			a+b+ab &= -1&\\
			a+b+ab+1 &= 0&\\
			(a+1)(b+1)&= 0&\\
			a &=-1&\\ 
			b &=-1&
		\end{align*}
		Since $-1 \notin S$, -1 is not in the image of *. Since $*:S\times S\mapsto
		S$, * is a binary operator on S.
	\end{addmargin}

\subsubsection*{b.}
	\begin{addmargin}[1em]{1em}
		\begin{proof} $\langle S,\ * \rangle$ is a group structure.
		\paragraph{$\mathcal{G}_1$: Associativity} For $a,\ b,\ c\in S$,
			\begin{align*}
				a*(b*c) &\stackrel{?}{=} (a*b)*c&\\
		  	a*(b+c+bc) &\stackrel{?}{=} (a*b) + c + (a*b)c&\\
		  	a + (b+c+bc) + a(b+c+bc) &\stackrel{?}{=}(a+b+ab)+c+(a+b+ab)c&\\
		  	a + b+ ab+c+ca+cb+abc &=a+b+ab+c+ca+cb+abc&
			\end{align*}
			Since $a*(b*c) = (a*b)*c$, * on S is a associative.
		\paragraph{$\mathcal{G}_2$: Left identity element} For $a \in S$
			\begin{align*}
				e * a &= a&\\
				e + a + ea &= a&\\
				e &= 0&
			\end{align*}
			Since 0 * a = a, 0 is the left identity element in S.
		\paragraph{$\mathcal{G}_3$: $\forall a \in S,\ \exists a^{-1},\ a^{-1}*a=e$,
			e is the identity}
			Let $a^{-1},\ a \in S$,
			\begin{align*}
				a^{-1} * a &= 0&\\
				a^{-1} + a + a^{-1}a &= 0&\\
				a^{-1} + a^{-1}a &= -a&\\
				a^{-1} &= \frac{-a}{1+a}&
			\end{align*}
			Since $\frac{-a}{1+a} * a=e$, $\forall a \in S, \frac{-a}{1+a}$ is an
			inverse of a.\\ \\
			Hence prooved $\langle S,\ * \rangle$ is a group structure.
			\end{proof}
	\end{addmargin}

\subsubsection*{c.}
	\begin{addmargin}[1em]{1em}
		Operation * is commutative because $a*b = a+b+ab = b+a+ba = b*a$.
		\begin{align*}
			2*x*3 &= 7 &\\
			(2*3)*x &= 7&\\
			(11)*x &= 7&\\
			(11) + x +11x &= 7&\\
			x +11x &= 7-11&\\
			x &= \frac{7-11}{1+11}&\\
			x &= \frac{-4}{12}&\\
			x &= \frac{-1}{3}&
		\end{align*}
	\end{addmargin}

\subsection*{30.}
\subsubsection*{a.}
	\begin{addmargin}[1em]{1em}
		\begin{proof} * is a Binary operator on $\mathbb{R}^*$\\
			$*:\mathbb{R}^*\times \mathbb{R}^* \mapsto \mathbb{R}^*$ by $a*b =
			|a|b$\\
			Since $|\mathbb{R}| \subset \mathbb{R}$, $\mathbb{R} \cdot \mathbb{R} \in
			\mathbb{R}$, * is binary operator on $\mathbb{R}$, 0 $\notin$ img(*) 
			$\Rightarrow$ * is binary operator on $\mathbb{R}^*$.
			\begin{align*}
				a*b &= 0&\\
				|a|b &= 0&\\
					a &= 0 &\\
					b &= 0 &
			\end{align*}
			Since $0 \notin \mathcal{R}^*$, a and b cannot be 0. 0 is therefore not
			in range of *. Hence proved * is a binary operator on $\mathbb{R}^*$.
		\end{proof}

		\begin{proof} * on $\mathbb{R}^*$ is associative\\
			Let a, b, c in $\mathcal{R}^*$,
			\begin{align*}
				a*(b*c) &= a*(|b|c)&\\
								&= |a||b|c&\\
								&= |ab|c&\\
								&= (a*b)*c\\
			\end{align*}
			Hence proved * on $\mathbb{R}^*$ is associative.
		\end{proof}
		\noindent
		Hence shown * gives an associative binary operation on $\mathbb{R}$.
	\end{addmargin}

\subsubsection*{b.}
	\begin{addmargin}[1em]{1em}
		Left identity e:
		\begin{align*}
			e * a &= a &\\
			|e|a &= a &\\
			e &= 1 &
		\end{align*}
		1 is a left identity of $\langle \mathbb{R}^*, *\rangle$.
		\begin{align*}
			a*a^{-1} &= 1 &\\
			|a|a^{-1} &= 1 &\\
			a^{-1} &= \frac{1}{|a|} &
		\end{align*}
		$\forall a \in \mathbb{R}^*$, $a^{-1} = \frac{1}{|a|}$ is the right inverse of a.
	\end{addmargin}

\subsubsection*{c.}
No because $|1|$ is the identity. That means $a \in \mathbb{R}$ has two inverses:
$a^{-1}$ and $-a^{-1}$.

\subsubsection*{d.}
It is important to enforce that all conditions of the one sided group axioms must
apply to the same side. Mixing and matching sides breaks an important constraint
of the group axioms.

\subsection*{32.}
	\begin{proof}Every group G with identity e and such that $x*x=e$ for all $x\in G$ is
	abelian. Let $a,\ b \in G$
		\begin{align*}
			e&=(a*b)*(a*b)&\\
			e&=(a*a)*(b*b)&\\
			a*b*a*b&=a*a*b*b&\\
			(a^{-1}*a)*b*a*(b*b^{-1})&=(a^{-1}*a)*a*b*(b*b^{-1})&\\
			b*a &= a*b &
		\end{align*}
	Hence every group G with identity e and such that $x*x=e$ for all $x\in G$ is
	abelian.
	\end{proof}

\subsection*{35.}
\begin{proof}
	Assume $(a*b)^2 = a^2 * b^2$, wants to show $a*b = b*a$
	\begin{align*}
		(a*b)^2 &= a^2 * b^2&\\
		(a*b)*(a*b) &= a*a*b*b&\\
		a*b*a*b&=a*a*b*b&\\
		(a^{-1}*a)*b*a*(b*b^{-1})&=(a^{-1}*a)*a*b*(b*b^{-1})&\\
		b*a &= a*b &
	\end{align*}
	Ergo $(a*b)^2 = a^2 * b^2 \Rightarrow a*b = b*a$
\end{proof}

\subsection*{36.}
\begin{proof}
	Assume $a*b = b*a$ wants to show $(a*b)'=a' * b'$
	\begin{align*}
		(a*b)' &= (b*a)'&\\
					 &= a'*b'&\\
	\end{align*}

	Assume $(a*b)'=a' * b'$ wants to show $a*b = b*a$.
	\begin{align*}
		(a*b)' &= a'*b'&\\
		(a*b)' &= (b*a)'&\\
		a*b &= b*a&
	\end{align*}
	Since $((a*b)'=a' * b'\Rightarrow a*b = b*a) \wedge (a*b = b*a
	\Rightarrow (a*b)'=a' * b')$, ergo $(a*b)'=a' * b'\Leftrightarrow a*b = b*a$
\end{proof}
\end{document}

