\documentclass{article}

\usepackage{amsmath, amsthm, amsfonts, mathtools}
\usepackage{enumitem}
\usepackage[margin=1in]{geometry}

\title {Abstract Algebra Homework 8}
\date {March 8, 2018}
\author {Huize(Jack) Shi - A92122910}

\begin {document}
\maketitle

\section*{Section 9}
\subsection*{2. }
$$(1\ 5\ 8\ 7)\ (2\ 6\ 3)$$

\subsection*{3. }
$$(1\ 2\ 3\ 5\ 4)\ (7\ 8)$$

\subsection*{4. }
$$(\cdots\ -1\ 0\ 1\ 2\ \cdots)$$

\subsection*{30. }
n elements are moved because any element that is not in the cycle are not moved
by $\sigma$. The $ith$ elements in the cycle are mapped to the $(i+1 mod n)th$
element of the cycle therefore moved by $\sigma$.

\subsection*{31. }
Show $H \le S_n$:\\
\paragraph{Closure: }Let $\sigma_1 \in H$ moves n elements (n is finite). 
Let $\sigma_2 \in H \mid \sigma_2$ moves m elements (m is finite).  The largest
possible movement as a result from permutation multiplication would be the case
when $\sigma_1$ and $\sigma_2$ are disjoint. In which case the size is $n+m$
which is finite. Therefore H is closed.
\paragraph{Identity: }Let $\sigma_0$ be a orbit of length 0. Since 0 is finite,
and $\sigma_0$ is the identity, H contains the identity element.
\paragraph{Inverse: }Given any $\sigma \in H \mid \sigma = {a_1,\ a_2\ \cdots\
a_n}$, $\sigma^{-1} \in H \mid \sigma = {a_n,\ a_{n-1}\ \cdots\ a_2,\ a_1}$.\\
Hence shown $H \le S_n$.

\section*{Section 10}
\subsection*{4. }
$$<4> = (\{0,\ 4,\ 8\},\ +_n)$$
Since $\mathbb{Z}_{12}$ is Abelian, left coset and right coset are equivalent.\\
\begin{align*}
0+<4> &= \{0,\ 4,\ 8\}\\
1+<4> &= \{1,\ 5,\ 9\}\\
2+<4> &= \{2,\ 6,\ 10\}\\
3+<4> &= \{3,\ 7,\ 11\}
\end{align*}

\subsection*{5. }
Since $\mathbb{Z}_{36}$ is Abelian, left coset and right coset are equivalent.\\
${i+<18> = \{0+i,\ 18+i\}\mid i \in \mathbb{Z}, 0 \le i < 17}$

\subsection*{6. }
${\rho_0,\ \mu_2},\ {\rho_1,\ \delta_2},\ {\rho_2,\ \mu_1},\ {\rho_3,\ \delta_1}$

\subsection*{19. }
\paragraph{a. }True, $aH, H\le G, \forall a\in G$ is always well definined.
\paragraph{b. }True, proved in class.
\paragraph{c. }True, proved in class.
\paragraph{d. }False, each partition can be infinite.
\paragraph{e. }True, subgroups are closed.
\paragraph{f. }False, the cosets could be infinite.
\paragraph{g. }True
\paragraph{h. }True, of course!
\paragraph{i. }False
\paragraph{j. }True

\subsection*{20. }
Impossible. A subgroup H of an abelian group satisfies $a*H = H*a$, which means
left right cosets are the same.

\subsection*{23. }
Impossible. Since each cell has to be non-empty and disjoint, it is impossible
to have a 12 cells partition for an group of order 6.

\subsection*{24. }
Impossible. The number of cells must divide the order of the group.

\subsection*{30. }
Not true. Given $G=S_3$, $H=\{\rho_0,\mu_1\}$, $a = \rho_1$, $b = \mu_3$.
$aH=\{\rho_1,\mu_3\}=bH$, $Ha=\{\rho_1,\mu_2\},\ Hb=\{\rho_2,\mu_3\}$, $Ha
\neq Hb$.

\subsection*{33. }
\begin{align*}
G &= S_3 \\
H &= \{e,\ (1\ 3)\} = (1, 3)H = eH \\
(1\ 2\ 3)H &= \{(1\ 2\ 3),\ (2\ 3)\} \\
(1\ 2\ 3)H &= (2\ 3)H\\
\left((1\ 2\ 3)^2H = (1\ 3\ 2)H \right) &\neq \left((2\ 3)^2H = eH\right)\\
\end{align*}
Hence shown $aH = bH \not\Leftrightarrow a^2H = b^2H$.

\end {document}

