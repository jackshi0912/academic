\documentclass{article}
\usepackage{amsmath, amsthm, amsfonts, mathtools}
\usepackage{enumitem}
\usepackage[margin=1in]{geometry}

\title {Abstract Algebra Homework 7}
\date {March 2, 2018}
\author {Huize(Jack) Shi - A92122910}

\begin {document}
\maketitle

\section*{Section 9:}
\subsection*{6. } 
\begin{equation*}
	\{3n \mid n \in \mathbb{Z}\},\ 
	\{3n+1 \mid n \in \mathbb{Z}\},\ 
	\{3n+2 \mid n \in \mathbb{Z}\}
\end{equation*}

\subsection*{9. }
\begin{equation*}
\Bigl(\begin{matrix}
1 & 2 & 3 & 4 & 5 & 6 & 7 & 8\\
5 & 4 & 3 & 7 & 8 & 6 & 2 & 1\\
\end{matrix}\Bigr)
\end{equation*}

\subsection*{23. }
\begin{enumerate} [label=\alph*. ]
	\item{False.}  % a.
		\begin{equation*}
		\Bigl(\begin{matrix}
		1 & 2 & 3 & 4\\
		2 & 1 & 4 & 3\\
		\end{matrix}\Bigr) = (1, 2)(3, 4)
		\end{equation*}

	\item{True. }  % b.
		A cycle can be written in permutation presentation.

	\item{False. } % c.
		The even and odd properties must be mutually exclusive.

	\item{False. } % d. 
		Counter example: $H = \{(1 2 3), (1 2 3)(1 2 3)\}$
	\item{False. } % e. 
		$\frac{5!}{2} = 60 \neq 120$
	\item{False. } % f. 
		$S_1$ contains only the identity therefore it is trivially cyclic.
	\item{True. } % g. 
	\item{True. } % h. 
	\item{True. } % i. 
	\item{False. } % j. 
		Odd permutations in $S_8$ is not closed. Permutation multiplication of two
		odd permutation result in a even permutation.
\end{enumerate}

\subsection*{29. }
Let $\omega \in H$ be an odd permutation, $\epsilon \in H$ be an even
permutation. We know $\epsilon$ exists if $\omega$ exists because odd
permutation is not a closed, since odd permutation becomes even under
permutation multiplication. Wants to show there exists bijection 
$f(\epsilon) = \sigma\epsilon$. This function maps even permutation to odd
permutation. $f$ is a bijection because $f^{-1}(\tau) = \sigma^{1}\tau$.
$\tau^{-1}$ exists because $\tau$ is a odd permutation in $H$ which is
bijective.\\
This shows that $f$ is a bijection that maps even permutations in H to odd 
permutations in $H$. This means the cardinality of odd permutation in $H$ must equal
to the cardinality of even permutation if there exists a odd permutation. Since
even permutations are closed under permutation multiplicaiton, every subgroup
$H$ of $S_n$ for $n \ge 2$, either all the permutations in $H$ are even or
exactly half of them are even (cardinality are the same between even
permutaitons and odd permutaitons in $H$).

\subsection*{34. }
Let $\tau = (a_1\ a_2\ a_3 \cdots a_{2k}\ a_{2k+1})$
$$\tau^2 = (a_1\ a_3\ a_5 \cdots a_{2k+1}\ a_2\ a_4 \cdots a_{2k})$$
Hence shown square of an odd length cycle is a cycle.

\subsection*{36. }
Wants to show $\lambda_a(g) = ag$ is a bijection from G to G.\\
$\lambda_a^{-1}(g') = a^{-1}g'$ is the inverse. It is well defined because $G$
is a group therefore $a \in G$ implies $a^{-1} \in G$

\end {document}

