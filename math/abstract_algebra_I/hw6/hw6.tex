\documentclass{article}
\usepackage{amsmath}
\usepackage{amsthm}
\usepackage{amsfonts}
\usepackage{mathtools}
\usepackage[margin=1in]{geometry}

\title{Abstract Algebra Homework 6}
\author{Jack Shi - A92122910}
\date{Feb. 21, 2018}

\begin{document}
\maketitle
\section*{Section 8 }
\subsection*{1. }
\begin{equation*}
	\Bigl(\begin{matrix}
		1 & 2 & 3 & 4 & 5 & 6\\
		1 & 2 & 3 & 6 & 5 & 4\\
	\end{matrix}\Bigr)
\end{equation*}
\subsection*{7. }
\begin{equation*}
	\tau^2 = 
	\Bigl(\begin{matrix}
		1 & 2 & 3 & 4 & 5 & 6\\
		4 & 3 & 2 & 1 & 5 & 6\\
	\end{matrix}\Bigr)
\end{equation*}
\begin{equation*}
	(\tau^2)^2 = 
	\Bigl(\begin{matrix}
		1 & 2 & 3 & 4 & 5 & 6\\
		1 & 2 & 3 & 4 & 5 & 6\\
	\end{matrix}\Bigr)
\end{equation*}
$<\tau^2> = \{\tau^2,\ e\}$ \qquad $|<\tau^2>| = 2$

\subsection*{8. }
\begin{align*}
	\sigma^6 &= 
	\Bigl(\begin{matrix}
		1 & 2 & 3 & 4 & 5 & 6\\
		1 & 2 & 3 & 4 & 5 & 6\\
	\end{matrix}\Bigr)\\
	\sigma^{100} &= (\sigma^6)^{16}\sigma^4 = \sigma^4 = 
	\Bigl(\begin{matrix}
		1 & 2 & 3 & 4 & 5 & 6\\
		6 & 5 & 2 & 1 & 3 & 4\\
	\end{matrix}\Bigr)
\end{align*}

\subsection*{33. }
$f_4$ is not a permutation on $\mathbb{R}$ because $f_4$ is not a surjective. -1
is in $\mathbb{R}$ but it is not in the image of $f_4$.

\subsection*{34. }
$f_5$ is not a permutation on $\mathbb{R}$ because $f_5$ is not a injective
$f_5(2) = f_5(-1) = 0$.

\subsection*{41. }
No because b is a particular element in $B$, $\sigma(b) \in B$. This means that 
it could be possible to have another function $\theta$ in this set that sends
$\sigma(b)$ outside of B. This means $\theta(\sigma(b)) \not\in B$ is not in the
set. This means that it is not closed.

\subsection*{42. }
No, the inverse is not strictly enforced. Lets say $\sigma(b) = b+1$ and
$\sigma[B] = \mathbb{Z}^+$. $\sigma^{-1}(1)$ would be undefined in this case.

\subsection*{48. }
\begin{proof} Let c be the element shared between $\mathcal{O}_{a, \sigma}$ and
$\mathcal{O}_{b, \sigma}$. This means that $\sigma^m(a) = c$, $\sigma^n(b) = c$.
Wants to show $\sigma^z(a) = \sigma^{z+k}(b) \mid z \in \mathbb{Z}$.
\begin{align*}
	\sigma^{m-n}(a) &= \sigma^{-n}(\sigma^{m}(a)) \\
								  &= \sigma^{-n}(c) \\
								  &= b
\end{align*}
With this, we can subsitute $b = \sigma^{m-n}(a)$ into $\sigma^z(b)$ to try
to get an relationship to $\sigma^z(a)$.
\begin{align*}
	\sigma^z(b) &= \sigma^z(\sigma^{m-n}(a)) \\
							&= \sigma^{z+m-n}(a) \\
\end{align*}
Since $m-n$ is constant, let $k = m-n$ the definition holds $\sigma^z(a) =
\sigma^{z+k}(b) \mid z \in \mathbb{Z}$. Hence shown $\mathcal{O}_{a,\sigma} =
\mathcal{O}_{b,\sigma}$ when both orbits share an common element.
\end{proof}

\subsection*{49. }
Let $A = \{a_1, a_2, \cdots a_n\}$. Let $\sigma(a_i) = a_{i+1\ mod\ n}$.
$<\sigma>$ defines a group that sends element in A to the next element. Let $H =
<\sigma>$, $|A| = |H| = n$. This satisfies the transitive property becasue given
$a_i$ and $a_j$, let $i < j$, $\sigma^{j-i}(a_i) = a_j \wedge \sigma^{i-j}(a_j)
= a_i$ (basically composing multiple ``move by one'' functions to send any a to
destination element).

\subsection*{52. }
\begin{proof}Given permutation $\rho_a:G \mapsto G$, where $\rho_a(x) = xa \mid a \in G \wedge x
\in G$. Let $H = \{\rho_a \mid a \in G\}$. Closed under permutation
multiplication: let $a, b \ in G$
\begin{align*}
(\rho_a\rho_b)(x) &= \rho_a(\rho_b(x)) \\
									&= \rho_a(xb) \\
									&= xba \\
									&= \rho_{ba}(x)
\end{align*}
$\rho_{ba}(x)$ is in $H$ because $ba \in G$. Identity element would be
$\rho_e(x) \mid x \in G$. Inverse exists as the following holds true
$\rho_a\rho_{-a} = \rho_e \mid a,-a \in G$. Hence H is a group.

Define isomorphism $\phi$, such that $\phi(a) = \rho_a$. This function is
trivially one to one. Since it is shown that $\rho_{ba} = \rho_a\rho_b$, the
following homomorphic property holds:
\begin{align*}
	\phi(ab) &= \rho_{ab} \\
					 &= \rho_a\rho_b\\
					 &= \phi(a)\phi(b)\\
\end{align*}
Hence shown H is a isomorphic to G.
\end{proof}

\end{document}

