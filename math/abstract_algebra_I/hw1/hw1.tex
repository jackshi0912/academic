\documentclass{article}
\usepackage{amsmath}
\usepackage{amssymb}
\title{Homework 1}
\author{Jack Shi - A92122910}
\date{Thursday January 18, 2018}

\begin{document}
\maketitle
\section*{Section 0}
\subsection*{16.}
	\paragraph{a.}
	$\mathcal{P}(\emptyset)=\Big\{\emptyset\Big\} \qquad
	|\mathcal{P}(\emptyset)|=1$
	\paragraph{b.}
	$\mathcal{P}(\{a\})=\Big\{\emptyset,\ a\Big\} \qquad
	|\mathcal{P}(\{a\})|= 2$
	\paragraph{c.}
	$\mathcal{P}(\{a,\ b\})=\Big\{\emptyset,\ \{a\},\ \{b\},\ \{a, b\}\Big\} \qquad
	|\mathcal{P}(\{a\})|= 4$
	\paragraph{d.}
	$\mathcal{P}(\{a,\ b,\ c\})=\Big\{\emptyset,\ \{a\},\ \{b\},\ \{c\},
	\ \{a,\ b\},\ \{b,\ c\},\ \{a,\ c\},
	\ \{a,\ b,\ c\}\Big\} \\
	|\mathcal{P}(\{a\})|= 8$

\subsection*{18.}
	$\forall$ set A and $B=\{1,\ 0\}$, $B^A=\{f\mid f:A\mapsto B\}$, find bijection T such
	that $T:B^A \mapsto \mathcal{P}(A)$\\ \\
	Let T be defined as $T(f)=\{a \in A\mid f(a)=1\}$
	\paragraph{Proof of bijectivity:} 
		\subparagraph{Proof of surjectivity: } Proove that 
			$\forall X \in \mathcal{P}(a),\ \exists\ f$ such that $T(f)=X$.\\
			Since $X\in\mathcal{P}(A)\Rightarrow X\subseteq A$, define f such that 
			$f(a) = \begin{cases}
				1 \qquad a\in X\\
				0 \qquad a\notin X
			\end{cases}$ \\
			This implies that $T(f)=X$. T is therefore surjective.
		\subparagraph{Proof of injectivity: } Proove that
			$\forall\ f_1,f_2\in B^A,\ T(f_1) = T(f_2) = X \Rightarrow f_1(a)=f_2(a),\
			a\in A$\\
			Case 1: $a \in X \Rightarrow f_1(a) = 1 \wedge f_2(a) = 1$\\
			Case 2: $a \notin X \Rightarrow f_1(a) = 0 \wedge f_2(a) = 0$\\
			Since $f_1(a) = f_2(a),\ \forall a \in A$, $f_1 = f_2$, T is injective. \\
		\par Since T is both surjective and injective, T is also bijective. 
		Since there exists bijection T, such that $T:B^A \mapsto \mathcal{P}(A)$, 
		$|B^A| = |\mathcal{P}(A)|$
		

\subsection*{30.}
$x\mathcal{R}y$ in $\mathbb{R}$ if $x \leq y$ is not a equivalence relation
because it is not symmetric. For example $(0 \leq 1) \wedge (1 \nleq 0)$.

\subsection*{31.}
Given $x\mathcal{R}y$ in $\mathbb{R}$ if $|x| = |y|$
\paragraph{Proove R is an equivalence relation:}
	\subparagraph{Reflexive: } $|x| = |x|$ because \\
		$\begin{cases}
			x=x \qquad &x>0\\
			-x=-x \qquad &x<0
		\end{cases}$ \\
	\subparagraph{Symmetric: } $(|x| = |y|) \Rightarrow (|y|=|x|)$ because \\
		$\begin{cases}
			x=x \qquad &(x>0) \wedge (y>0)\\
			-x=y \qquad &(x<0) \wedge (y>0)\\
			x=-y \qquad &(x>0) \wedge (y<0)\\
			-x=-y \qquad &(x<0) \wedge (y<0)\\
		\end{cases}$ \\
	\subparagraph{Transitive: } $(|x| = |y| \wedge |y| = |z|) \Rightarrow
		(|x|=|z|)$ because \\
			$\begin{cases}
				x=y=z \qquad &(x>0) \wedge (y>0) \wedge (z>0)\\
				-x=y=z \qquad &(x<0) \wedge (y>0) \wedge (z>0)\\
				x=-y=z \qquad &(x>0) \wedge (y<0) \wedge (z>0)\\
				x=y=-z \qquad &(x>0) \wedge (y>0) \wedge (z<0)\\
				-x=-y=z \qquad &(x<0) \wedge (y<0) \wedge (z>0)\\
				x=-y=-z \qquad &(x>0) \wedge (y<0) \wedge (z<0)\\
				-x=y=-z \qquad &(x<0) \wedge (y>0) \wedge (z<0)\\
				-x=-y=-z \qquad &(x<0) \wedge (y<0) \wedge (z<0)\\
			\end{cases}$ \\

		\par Since $\mathcal{R}$ is reflexive, symmetric, transitive, $\mathcal{R}$ 
		is a equivalence relation.
\paragraph{Describe patition araising from $\mathcal{R}$: }
	Since $\overline{a}$ and $\overline{-a}$, $a\in\mathbb{R}$, is consisted of
	$\{a,\ -a\}$, the set $\mathbb{R}$ is therefore partitioned into $\{x, y\},\
	x,y\in\mathbb{R} \wedge x=-y$

\subsection*{32.}
	$x\mathcal{R}y$ in $\mathbb{R}$ if $|x - y|\leq 3$ is not a equivalence
	relation because it fails transitivity. $0\mathcal{R}1 \wedge 1\mathcal{R}4
	\nRightarrow 0\mathcal{R}4$ since $|0-1|\leq3$ and $|1-4|\leq3$, however, 
	$|0-4|\nleq3$.
\section*{Section 2}
\subsection*{1.}
	$b*d=e$, \\$c*c=b$,
	\begin{flalign*}
	[(a*c)*e]*a&=[c*e]*a&\\
	&=a*a&\\
	&=a&
	\end{flalign*}
\subsection*{2.}
	\begin{flalign*}
	(a*b)*c&\stackrel{?}{=}a*(b*c)\\
	b*c&\stackrel{?}{=}a*a&\\
	a&=a&
	\end{flalign*}
	It is possible that operation * is associative. To assert that * is associative, 
	it must be prooven that such operation holds true for all other triples. 
	Therefore this computation is not sufficient.
\subsection*{3.}
	\begin{flalign*}
	(b*d)*c&\stackrel{?}{=}b*(d*c)\\
	e*c&\stackrel{?}{=}b*b&\\
	a&\neq c&
	\end{flalign*}
	Operation * is not associative.
\subsection*{4.}
	\begin{flalign*}
	b*e&\stackrel{?}{=}e*b&\\
	c&\neq b&
	\end{flalign*}
	Operation * is not commutative.
\subsection*{18.}
	Operation * on $\mathbb{Z}^+$ as $a*b=a^b$ is a binary operation since $a^b \in
	\mathbb{Z}^+$ is exactly one element in $\mathbb{Z}^+$.
\subsection*{24.}
	\textbf{a. }False \quad \textbf{b. }True \quad \textbf{c. }False \quad 
	\textbf{d.}False \quad \textbf{e. }False\\ 
	\textbf{f. }True \quad \textbf{g. }True \quad \textbf{h. }True \quad 
	\textbf{i. }True \quad \textbf{j. }False 
\subsection*{26.}
	\paragraph{The following holds if * is an associative and commutative binary
	operation on a set S: }
	\begin{flalign*}
	(a*b)*(c*d)&\stackrel{?}{=}[(d*c)*a]*b&\\
	&\stackrel{?}{=}(d*c)*(a*b)&\\
	&\stackrel{?}{=}(a*b)*(d*c)&\\
	&=(a*b)*(c*d)&
	\end{flalign*}
\subsection*{35.}
	Statement is false. \\
	Let * be +, *' be $\cdot$, and S be $\mathbb{Z}$. If a, b, c =
	1, 2, 3 respctively, 
	\begin{flalign*}
	1+(2\cdot3) &\stackrel{?}{=} (1+2)*(1+3)&\\
	1+6 &\stackrel{?}{=} 3*4&\\
	7 &\neq 12&
	\end{flalign*}
\subsection*{36.}
	Given $a,b \in H$, wants to show $(a,b) \in H$. \\
	By definition of H, $(a*x = x*a) \wedge (b*x=x*b),\ \forall x\in S$.\\
	Since * is associative, the following holds:
	\begin{flalign*}
		(a*b)*x&=a*(b*x)&\\
					 &=a*(x*b)&\\
					 &=(a*x)*b&\\
					 &=(x*a)*b&\\
					 &=x*(a*b)&\\
	\end{flalign*}
	By definition of H, $(a*b) \in H$. Hence H is closed under *.

\subsection*{37.}
	Given $a, b \in H$, wants to show $(a, b) \in H$. \\
	By definition of H, $(a*a = a) \wedge (b*b=b)$.\\
	Since * is associative and commutative, the following holds:
	\begin{flalign*}
		(a*b)*(a*b)&=a*(b*a)*b&\\
					 		 &=a*(a*b)*b&\\
					 		 &=(a*a)*(b*b)&\\
					 		 &=(a*b)&\\
	\end{flalign*}
	By definition of H, $(a*b) \in H$. Hence H is closed under *.
	

\end{document}
