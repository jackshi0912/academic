\documentclass{article}
\usepackage{amsmath, amssymb, amsthm}
\usepackage{enumitem}

\title{Abstract Algebra: Homework 3}
\author{Huize Shi - A92122910}
\date {February 2, 2018}

\begin{document}
\maketitle

\section*{Section 5: }
\subsection*{11. }
	Let H be a $n\times n$ matrix with determinant -1. This is not a subgroup of G
because H is not closed. Let $h_1, h_2 \in H$:
	$$det(h_1 \cdot h_2) = det(h_1) \cdot det(h_2)=-1 \cdot -1 =1$$
	Since matrix with determinant 1 is not in H, H is not a subgroup of G.
	
\subsection*{39. }
	\begin{enumerate}[label=\textbf{\alph*}.]
		% a.
		\item True. Associativity is a part of the definition of Group.
		% b.
		\item False. Every element in a group must have a inverse. This means that the
		cancellation law must hold.
		% c.
		\item True. Every group is a subset of itself which is a group.
		% d.
		\item False. Only the group itself is the improper subgroup.
		% e.
		\item False. Only the element(s) whoes absolute value is the smallest are the
			generator(s).
		%f.
		\item False. $<2> = <-2>$
		%g.
		\item False. Set of negative numbers is closed under addition but not closed
			under multiplicaiton.
		%h.
		\item False. Subgroup has additional constraints such as closed under
			opteration.
		%i.
		\item True. It's generated by 1 and 3
		%j.
		\item False. Subset of a group may not contain identity.
	\end{enumerate}

\subsection*{40. }
Given $G = \langle\mathbb{Z}, \cdot\rangle$
$$x^2 = e$$
$$x^2 = 1$$
$$x_1 = 1,\ x_2 = -1$$

\subsection*{41. }
Let $g_1,\ g_2 \in \langle G, *\rangle$, $h_1,\ h_2 \in \langle H, *\rangle$,
$H\le G \Rightarrow h \in G$.
$$h \in G \Rightarrow \phi (h) \in G'$$
Since H is a group, $\phi(H)$ is a group. Since $\phi(h) \in H \wedge \phi(h)
\in G$, $\phi(H) \subseteq \phi(G)$

\subsection*{43. }
\begin{proof} Show that $\{hk\mid h\in H,\ k\in K\}$ is closed under *,
	contain identity element and inverses for all elements:
	Let $A = \{hk\mid h\in H,\ k\in K\}$

	\paragraph{Closure: }Let $a,b \in A \Rightarrow a=hk,\ b=h'k'\mid h,\ h'\in
	H \wedge k,\ k'\in K$. Wants to show $a*b \in A$
	\begin{align*}
		a*b &\in A&\\
		hkh'k' &\in A&
	\end{align*}
	Since A is Abelian, the following is true:
	\begin{align*}
		hh'kk' &\in A&\\
		hh' \in H &\wedge kk' \in K&\\
		a*b &\in A&
	\end{align*}

	\paragraph{Left identity: }Show that $\exists\ e_A \in A 
		\mid e_A*a = a,\ \forall a \in A$ \\
		Let $e_A = e_G \mid e_G * g = g,\ \forall g \in G$. 
		Wants to show $e_A * a = a$, since $H, K \subseteq G \Rightarrow h, k \in G$
		the following holds:
		\begin{align*}
			(e_G)(hk) &= (e_Gh)k&\\
							 &= hk&
		\end{align*}
		Ergo $\exists e_A \in A \mid e_A*a=a, \forall a \in A$

	\paragraph{Left inverse: }Show that $\exists\ a^{-1} \in A 
		\mid a^{-1}*a = e_A=e_G,\ \forall a \in A$ \\
		Since $H,\ K \leq G$, $h^{-1}h = k^{-1}k = e_G=e_A$, let $a^{-1} =
		h^{-1}k^{-1}$:
		\begin{align*}
			a^{-1}a &= (h^{-1}k^{-1})(hk)&\\
							&= (h^{-1}h)(k^{-1}k)&\\
							&= e_Ae_A=e_A&\\
		\end{align*}
		Ergo the inverse $a^{-1}\ \exists \forall a \in A, a^{-1}a=e_A$
		\par Since we have now shown that the set $A = \{hk\mid h\in H,\ k\in K\}$
		is closed under *, associative, has left identity and has left inverse, it
		can be concluded that the set $A = \{hk\mid h\in H,\ k\in K\}$ is a subgroup
		of G.
\end{proof}

\subsection*{51. }
\begin{proof}$H_a = \{x \in G \mid xa=ax\}$
	\paragraph{Closure: }Let $x,y \in H_a \mid ax=xa,\ ay=ya$
	\begin{align*}
		a(xy) &= (ax)y&\\
		      &= (xa)y&\\
		      &= x(ay)&\\
		      &= x(ya)&\\
		      &= (xy)a&\\
	\end{align*}
	Since $a(xy)=(xy)a$ holds, $xy \in H_a$, $H_a$ is closed.

	\paragraph{Left identity: }Let e be the identity in G. Since $a \in G$, The
	following holds $e*a = a$.

	\paragraph{Left inverse: }Let $x \in H_a \mid ax=xa$
	\begin{align*}
		ax&=xa&\\
		a(xx^{-1})&=xax^{-1}&\\
		a&=xax^{-1}&\\
		x^{-1}a&=(x^{-1}x)ax^{-1}&\\
		x^{-1}a&=ax^{-1}&
	\end{align*}
	Ergo, the inverse of a is also in $H_a$
	Hence shown that $H_a = \{x \in G \mid xa=ax\}$ is a subgroup of G.
\end{proof}

\section*{Section 6: }
\subsection*{18. }
\begin{align*}
	gcd(30, 42) &= 6&\\
	\frac{42}{6} &= 7&
\end{align*}
There are 7 elements in $\mathbb{Z}_{42}$ manufactured by 30.

\subsection*{27. }

\begin{align*}
	gcd(0, 12) &= 0 &\\
	gcd(1, 12) &= 1 &\\
	gcd(2, 12) &= 2 &\\
	gcd(3, 12) &= 3 &\\
	gcd(4, 12) &= 4 &\\
	gcd(5, 12) &= 1 &\\
	gcd(6, 12) &= 6 &\\
	gcd(7, 12) &= 1 &\\
	gcd(8, 12) &= 4 &\\
	gcd(9, 12) &= 3 &\\
	gcd(10, 12) &= 2 &\\
	gcd(11, 12) &= 1 &
\end{align*}

Subgroups for $\mathbb{Z}_{12}$ are $<0>$, $<1>$, $<2>$, $<3>$, $<4>$, $<6>$
\subsection*{32. }
	\begin{enumerate}[label=\textbf{\alph*}.]
  	% a.
		\item True. Cyclic group can be expressed as <a>. Since $a^n$ is a
			repeatedly operated on itself, cyclic group is trivially
			Abelian.
		% b.
		\item False. Klein 4-Group is Abelian but not cyclic.
		% c.
		\item False. There is always a number whose absolute distance is closer to 0
			than the generator.
		% d. 
		\item False The group $2\mathbb{Z}$ is only generated by $<2>$ and $<-2>$, 4
			is generated by 2 but it is not a generator of $2\mathbb{Z}$.
		% e.
		\item True. Diagnal symmetric tabel exists for every dimension.
		% f.
		\item False. Klein 4-Group is not cyclic and it is of order 4.
		% g.
		\item False. 9 is coprime to 20, therefore 9 generates $\mathbb{Z}_{20}$,
			but 9 is not a prime.
		% h.
		\item False. $G \cap G$ does not necessarily contain the identity element.
		% i.
		\item True.
		% j.
		\item True. There are at least two number (1 and n-1) are coprime to n when
			n > 2.
	\end{enumerate}

\subsection*{33. }
The Klein 4-Group is Abelian but not cyclic.

\subsection*{44. }
Given a subgroup H, of cyclic group G, H is trivially cyclical if $H = \{e\}$.
Otherwise let a be a generator of the cyclic group H. Let n be the smallest
element $\in \mathbb{Z}^+$ such that $a^n \in H$. For any other member, $a^m \in
H$, division algorithm for n divided by m states that $n=qm \mid
q\in\mathbb{Z}$. Hence shown $a^m = (a^n)^q$.

\subsection*{46. }
	Let $a, b \in G$, show 

\end{document}

