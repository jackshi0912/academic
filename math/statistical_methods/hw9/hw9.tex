\documentclass{article}
\usepackage{amsmath, amsthm, amsfonts, mathtools}
\usepackage{enumitem}
\usepackage[margin=1in]{geometry}
\begin{document}

\section*{1a}
The alternative hypothesis should be $\bar{x} > 78.74$ since the question is
whether American women \textbf{live longer} than American men.

\section*{1b}
The prior null distribution should be normal: $H_0 \sim \mathcal{N}$. The
sampling method should be unbiased thus making sure that the sample distribution
is also normal.

\section*{1c}
\begin{align*}
H_0 &\sim \mathcal{N}(78.74, \frac{5.3}{\sqrt{200}} = 0.375)\\
p(\bar{x} > 79.5 \mid H_0) &= \texttt{pnorm(79.5, mean=78.74, sd=0.375,
																			lower.tail=F)}\\
													 &= 0.0213
\end{align*}
Since $0.0213 < 0.05$, we reject $H_0$ in favor of $H_A$. American women does
live longer than American men.

\section*{2}
$\mu$: made up SSI mean of the phony samples.
\begin{align*}
	H_0:\ \mu &= 62\\
	H_A:\ \mu &< 62\\
	H_0 &\sim \mathcal{N}(62, \frac{7}{\sqrt{40}}=1.107)\\
	\mu &= \texttt{qnorm(0.01, mean=62, sd=1.107, lower.tail=T)}\\
			&= 59.425
\end{align*}
I should make up SSI average $\mu=59.425$ to achieve $\alpha = 0.01$ cutoff.

\section*{3a}
Weight of 16 students $\sim \mathcal{N}(147 \cdot 16,
\frac{10}{\sqrt{16}}\cdot 16) = \mathcal{N}(2352, 40)$

\section*{3b}
$\mu$: The mean weight of the combined college students
\begin{align*}
p &\sim \mathcal{N}(2352, 40)\\
p(\mu > 2500) &= \texttt{pnorm(2500, mean=2352, sd=40, lower.tail=F)}\\
							&= 0.00010780
\end{align*}

\section*{3c}
Let random variable $X$ deonte: the number of trials to get $k^{th}$ success
(success being a zazzle got into elevator).
\begin{align*}
p &\sim NBinom(k=5)\\
\mathbb{E}[X] &= \frac{5}{0.00010780}\\
							&= 46382.19\\
\end{align*}
It is expected that roughly 46383 groups of 16 would need to ride before the
cable snaps.

\section*{4a}
$\sigma_{\bar{x}}$: The standard deviation of the sampling distribution of
standard IQ tests scores for American men.
\begin{align*}
H_0:\ \sigma_{\bar{x}} &= 14\\
H_A:\ \sigma_{\bar{x}} &> 14
\end{align*}

\section*{4b}
We would fail to reject the null hypothesis. It appears that the standard
deviation of the sampling distribution of standard IQ tests scores for American
men is insuffiently different from that of American woman.

\section*{4c}
P-value indicate the probability of seeing an experiement observation appear in
the null distribution $H_0$. A p-value of $0.07$ means that the probability of
seeing the experimental observation appear in the null distribution is $0.07$.
Unless it is below some threshold, we would conclude that the observation is not
rare enough to reject that the $H_0$ distribution models the system correctly,
hence failing to reject the null hypothesis.

\section*{5b}
The exact shape is a Gaussian distribution $\mathcal{N}(2,
\frac{1.155}{\sqrt{300}}) = \mathcal{N}(2, 0.0667)$. The distribution is
centered around 2 with standard deviation of $0.0667$

\section*{5c}
The shape is defined by $p(x) = 0.125x$ it is a left skewed distribution. The
center is at 2.667. The standard deviation is 0.889.

\section*{R1}
$\mu$: The mean of the measurements Michelson made for the speed of light.\\
Hypothesis:
\begin{align*}
	H_0:\ \mu = 300000\\
	H_A:\ \mu \neq 300000
\end{align*}
CV = [299712.44, 299799.99]\\
Since 300000 is not in the confidence interval, we reject $H_0$, it appears that
the the measurements taken by Michelson disagrees with the real speed of light.

\section*{R2}
\begin{align*}
H_0 &\sim \mathcal{N}(299756.217, 107.115)\\
p\_value &= 2 \cdot pnorm(300000, mean=299756.217, sd=107.115, lower.tail=F)\\
p\_value &= 0.023
\end{align*}
The mean and standard deviation for the null distribution are calculated from
the Michelson experiments. The p-value is calculated by calculating the
probability of the true value showing up in the null distribution.

\end{document}

