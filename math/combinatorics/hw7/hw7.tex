\documentclass{article}
\usepackage{amsmath, amssymb, amsthm, mathtools}
\usepackage{enumitem}
\usepackage{listings}
\usepackage[margin=1in]{geometry}
\title{Combinatorics Homework 7}
\author{Jack Shi - A92122910}
\date{March 9th, 2018}

\begin{document}
\maketitle

\begin{enumerate} [label=(\arabic*)]

	% (1)
	\item Give an example showing that the Friendship Theorem does not hold for
		infinite graphs.\\
		Construct the following graph:
			\begin{enumerate} [label=(\roman*)]
				% i
				\item Start with a regular pantagon $G_0=G_5$.
				% ii
				\item Add a vertex that is adjacent (connected) to each pair of
					vertices in $G_i$ to form $G_{i+1}$.
				% iii
				\item Since the graph can be infinite, if this process were to be
					repeated for infinite steps, each pair of vertex shares a common
					neighbor because of step (ii).
			\end{enumerate}
				However, there does not exist a hub vertex since the original pantagon
				does not get connected to a single point. Hence shown Friendship Theorem
				does not hold for infinite graphs.

	% (2)
	\item Let G be the graph with vertex set $V = \{1,\ 2\dots\ n\}$ and edge set
		$\{\{1,\ j\}:2\le j \le n\}$. Calculate the number of r-step walks on G
		which:
		\begin{enumerate} [label=(\alph*)]
			% (a)
			\item Begin and end at 1
				\[
					r-step-walk(r, n) = 
						\begin{cases}
							0, & \text{if } (r) \text{ is odd} \\
							(n-1)^{\frac{r}{2}}, & \text{otherwise}
						\end{cases}
			\]
			% (b)
			\item Begin and end at 2
				\[
					r-step-walk(r, n) = 
						\begin{cases}
							0, & \text{if } (r) \text{ is odd} \\
							(n-1)^{\frac{r-2}{2}}, & \text{otherwise}
						\end{cases}
			\]
			% (c)
			\item Begin at 1 and end at 2
				\[
					r-step-walk(r, n) = 
						\begin{cases}
							0, & \text{if } (r) \text{ is even} \\
							(n-1)^{\frac{r-1}{2}}, & \text{otherwise}
						\end{cases}
			\]
			% (d)
			\item Begin at 2 and end at 3
				\[
					r-step-walk(r, n) = 
						\begin{cases}
							0, & \text{if } (r) \text{ is odd} \\
							(n-1)^{\frac{r-2}{2}}, & \text{otherwise}
						\end{cases}
			\]
		\end{enumerate}

	% (3)
	\item Prove that any group of six people contains either three mutual
		friends, or three mutual strangers. Does the same statement hold for groups
		of five? Explain.\\
		\paragraph{Compute $R(3,3)$: }Given a complete graph $G=(V, E)$ on 6
		vertices. Let 3 edges be blue and 3 be red. Given $v \in V$, since G is
		complete, there are 5 edges with v: $(v, v_1),\ (v, v_2),\ (v, v_3),\ (v,
		v_4),\ (v, v_5)$. Since half of the edges are colored differently from
		the other half, at least 3 edges in the aforementioned set are of same
		color. Lets pick $(v, v_1),\ (v, v_2),\ (v, v_3)$ are blue (if not possible,
		just switch color). This means if $(v_1, v_2),\ (v_2, v_3),\ (v_1, v_3)$ are
		blue, there exist a blue triangle, hence we have three mutual firends. If
		none of the $(v_1, v_2),\ (v_2, v_3),\ (v_1, v_3)$ edges are blue, we have
		an all red triangle and no blue triangle which means we have a group of
		three strangers.
		\paragraph{Group of 5: }No it does not. With group of five, draw a complete
		graph $G_5$ and color all outside edges red and inside edges blue. In this
		case we do not have any triangles of same color, therefore we can not state
		that there exist either a group of 3 strangers or a group of 3 friends.

\end{enumerate}

\end{document}
