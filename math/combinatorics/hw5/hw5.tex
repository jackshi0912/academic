\documentclass{article}
\usepackage{amsmath, amssymb, amsthm, mathtools}
\usepackage{enumitem}
\usepackage{listings}
\usepackage[margin=1in]{geometry}
\title{Combinatorics Homework 5}
\author{Jack Shi - A92122910}
\date{February 16}

\begin{document}
\maketitle

\begin{enumerate} [label=(\textbf{\arabic*})]
	% (1)
	\item Show that the vertices of any graph can be colored black and white in
		such a way that each white vertex has at least as many black neighbours as
		white neighbours, and vice versa.

		\textbf{Solution: }Start by randomly color the graph. Pick each vertex with
		number of same color neighbours greater than that of different color
		neighbours. Flip the color of the vertex. This has two effects: \\\\
		\textbf{First,} the vertex is ``fixed'' in terms of the graph coloring conditions.\\
		\textbf{Second,} the number of different color edge (edge with two different color
		vertices) of the graph increase by at least 1 because before switching
		color, the number of same colored edge for that vertex is greater than that
		of different colored edge. \\\\
		With these two conditions, if this operation is continuously
		performed on the graph, there will be two outcomes:\\\\
		\textbf{Outcome A: }All the edges are of different color, in which case the
		condition is reached because the number of same color neighbours is 0
		therefore number of different color neighbours is greater.\\
		\textbf{Outcome B: }The graph coloring condition is reached before all the
		edges is different color.\\\\
		This algorithm will always converge because as long as Outcome B hasn't been
		reached, it is possible to increase the number of different color edges in
		the graph therefore pushing towards outocme A. Either outcome satisifes the
		coloring condition that the vertices of any graph can be colored black and
		white in such a way that each white vertex has at least as many black
		neighbours as white neighbours, and vice versa.

	% (2)
	\item Solution:
	\begin{lstlisting}
Algorithm n-step-random-walk(startVertex, targetVertex, steps):
	if steps = 0:
		return 1 if startVertex = targetVertex else 0
	sum := 0
	for neighbour in neighbours(startVertex):
		sum += n-step-random-walk(neighbour, targetVertex, steps-1)
	return sum
	\end{lstlisting}

	% (3)
	\item
		\begin{enumerate}
			% (a)
			\item 
				\begin{proof} Since the graph is connected palnar as well as triangle
				free, the nthe following holds. The sum of degree of faces is $2|E|$.
				$|V| - |E| + |F| = 2$. Since graph is triangle free, each frace must
				have degree $\ge 4$. This means that $2|E| \ge 4|F| \Leftrightarrow
				\frac{1}{2}|E| \ge |F|$\\
				This implies the following: 
				\begin{align*}
				|E| - |V| + 2 &\le \frac{1}{2}|E|\\
				\frac{1}{2}|E| - |V| + 2 &\le 0\\
				|E| &\le 2|V| - 4
				\end{align*}
				\end{proof}
			% (b)
			\item
				\begin{proof}
					By Euler's theorem, $|V| - |E| + |F| = 2$. In case of $K_{3,3}$,
					$|F|=2+9-6=5$. We also know that sum of degree of faces is $2|E|$. In
					the case of $K_{3,3}$, we know that each face is bounded by at least 4
					edges, therefore this result in a contradiction as $4*5 \ne 2*9
					\Rightarrow 20 \ne 18$. Since this property does not hold, $K_{3,3}$
					could not possibly be planar.
				\end{proof}

		\end{enumerate}
\end{enumerate}

\end{document}

