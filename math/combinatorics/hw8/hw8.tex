\documentclass{article}
\usepackage{amsmath, amssymb, amsthm, mathtools}
\usepackage{enumitem}
\usepackage{listings}
\usepackage[margin=1in]{geometry}
\title{Combinatorics Homework 7}
\author{Jack Shi - A92122910}
\date{March 16th, 2018}

\begin{document}
\maketitle

\begin{enumerate} [label=(\arabic*)]
	% (1)
	\item For each permutation $\pi$ of $1,\cdots,N$, let $LIS(\pi)$ denote the
		maximal length of an increasing subsequence in $\pi$, and let $LDS(\pi)$
		denote the maximal length of a decreasing subsequence in $\pi$.
		\begin{enumerate} [label=(\alph*)]
			% (a)
			\item Show that $LIS(\pi)LDS(\pi) \ge N$ for all $\pi$.\\
			Computer Erdos-Szekeres number $ES(m, n) = N = (m-1)(n-1)+1$.
			\paragraph{Case 1.}$LIS\ or\ LDS = N$. The condition is trivially
			satisfied.
			\paragraph{Case 2.}$LIS\ or\ LDS \neq N$. The lower bound would be the
			case when either $LIS(\pi)\ \text{or}\ LDS(\pi) = 2$ because there must
			exist a pair of numbers that are out of order otherwise it would be the
			first case. The other number (the one that's not 2) is as small as
			possible without violating Erdos-Szekeres theorem. This condiion is
			reached when $m=n$. Any imbalance would cause either LIS or LDS to be
			bigger.
			\begin{align*}
				N &= (n-1)(n-1) + 1\\
				N-1 &= (n-1)^2\\
				n \ge \frac{N}{2}
			\end{align*}
			Since the smallest possible product of $LIS(\pi)LDS(\pi)$ is
			$\frac{N}{2}*2$. Hence $LIS(\pi)LDS(\pi) >= N$

			% (b)
			\item Let $\pi_N$ be a $random$ permutation of $1,\cdots,N$. Show that the
				expected value of $LIS(\pi_N)$ is at least $\sqrt{N}$.\\
		\end{enumerate}

	% (2)
	\item Show that, in any finite gathering of people, there are at least two
		people who know the same number of people at the gathering (assume that
		“knowing” is a mutual relationship).
		\begin{proof}
			First represent the people as vertices V, and ``mutually knowing'' as
			edges, E, connecting pairs of people. The group of people can be therefore
			represented as a graph $G = (V,\ E)$ Wants to show there exists $v_i,\ v_j
			\in V \mid deg(v_i) = deg(v_j)$
			\paragraph{Case 1: G is a connected finite simple graph }
				Suppose $|V|=n$, the vertex set contains vertices $\{v_1,\ v_2,\dots
				v_n\}$, the set of degrees for each vertex contains $\{1,\ 2, \dots
				n-1\}$. $0$ is not in the possible degree set since the graph is
				connected.\\
				By Pigeonhole Principle, there exist a fiber of the function 
				$f:V \mapsto deg(V)$ that is of size at least $2$. This means that there
				exist at least two people with the same number of ``known people.''
			\paragraph{Case 2: G is not a connected graph }If G is a empty graph, then
			everyone has 0 ``known people'' which satisifies the condition. If there
			exist edges in the graph, isolate each connected graph and based on
			\textbf{Case 1}, there exist two people with the same number of ``known
			people.''
		\end{proof}
		
	% (3)
	\item Show that, given any five points in a unit square, two of these points
		are separated by a distance of at most $\frac{1}{\sqrt{2}}$.\\
		\begin{proof}
			Given a unit square, split the unit square into 4 identicle
			$\frac{1}{2} \times \frac{1}{2}$ squares by connecting the midpoint of
			each opposite sides. Given an assignment function $f$ that maps each point
			to a small square. By pigeonhole principle there exist a fiber of $f$ such
			that the size of the fiber is at least 2. This means that at least two
			points must be in the same small square. The maximum distance between this
			two points would be 
			\begin{align*}
			\sqrt{\left(\frac{1}{2}\right)^2 + \left(\frac{1}{2}\right)^2} 
			&= \sqrt{\frac{1}{4} + \frac{1}{4}}\\
			&= \sqrt{\frac{1}{2}}\\
			&= \frac{1}{\sqrt{2}}
			\end{align*}
		\end{proof}
		
\end{enumerate}
\end{document}

