\documentclass{article}
\usepackage{amsmath, amssymb, amsthm, mathtools}
\usepackage{enumitem}
\usepackage[margin=1in]{geometry}
\DeclarePairedDelimiter{\floor}{\lfloor}{\rfloor}
\DeclarePairedDelimiter\abs{\lvert}{\rvert}

\title{Combinatorics: Homework 2}
\author{Huize Shi - A92122910}
\date{Friday, January 26, 2018}

\begin{document}
\maketitle

\begin{enumerate}[label=\textbf{\arabic*}.]
	\item  % Question 1
		\begin{proof}Assume towards contradiction that $\log{2}{3}$ is a 
			rational number. This implies that $\log_2 3=\frac{a}{b}$ where $a, b
			\in \mathbb{Z}$.\\
			\begin{align*}
				\log_2 3 &= \frac{a}{b}&\\
				2^{\frac{a}{b}} &= 3&\\
				(2^{\frac{a}{b}})^b &= 3^b&\\
				2^a &= 3^b&
			\end{align*}
			This is clearly a contradiction because 2 is not a factor of $3^b,\
			b\in\mathbb{Z}$, but 2 is a factor of  $2^a,\ a\in\mathbb{Z}$.
		\end{proof}

	\item  % Question 2
		By Legendre's Theorem
		$\nu_p(n!)=\sum_{k=1}{\floor*{\frac{n}{p^k}}}$, which gives the exact
		exponent of p in the prime factorization of n!
		\begin{align*}
			{n \choose k} &= \frac{n!}{k!(n-k)!}&\\
			\nu_p\left({n \choose k}\right ) &= \nu_p\left(\frac{n!}{k!(n-k)!}\right )&\\
													 &= \nu_p\left(n!\right ) - (\nu_p(k!) + \nu_p((n-k)!))&\\
													 &= \sum_{i=1}{\floor*{\frac{n}{p^i}}} -
													 \sum_{i=1}{\floor*{\frac{k}{p^i}}} - 
													 \sum_{i=1}{\floor*{\frac{(n-k)}{p^i}}}&\\
													 &= \sum_{i=1}{\left(\floor*{\frac{n}{p^i}} -
													 \floor*{\frac{k}{p^i}} - 
													 \floor*{\frac{(n-k)}{p^i}}\right )}&
		\end{align*}
		The formula for the multiplicity of p in the binomial coefficient ${n \choose
		k}$ is as follows: 
		$$\sum_{i=1}{\left(\floor*{\frac{n}{p^i}} - \floor*{\frac{k}{p^i}} - \floor*{\frac{(n-k)}{p^i}}\right )}$$

	\item \quad  % Question 3
		\begin{enumerate}
			\item  %  3.a
				\begin{proof}
				Given $r = \frac{a}{b},\ r!=0,\ b>0$, r is a rational number in normal form such
				that $\frac{a}{b}$ is a unique, irreducible representation of r. if r is
				not in normal form, normalization of r is trivial by eleminating shared
				factors between a and b.	
				a and b are coprimes because any shared factor would have been
				elimenated during normalization of r. \\
				Let $l,\ i,\ j \in \mathbb{Z},\ l >= 1$, $p \in \mathbb{P}$:

				\begin{align*}
					r = \frac{a}{b} &= \frac{\pm\prod_{i=1} p_i^{k_i}}{\prod_{j=1} p_j^{e_j}}&\\
					&= p_l^{k_l}\frac{\pm\prod_{i=1, i\neq l} p_i^{k_i}}{\prod_{j=1}
					p_j^{e_j}}&\\
				\end{align*}
				Since this is of the form $p^k\frac{a}{b}$, and a,b are coprimes, $p^k$
				is taken out of the prime factors of a, therefore it is not in the prime
				factor of a or b which is coprime to a, this proves that any nonzero rational 
				number r can be written uniquely in the form $p^k\frac{a}{b}$, where k is 
				an integer, a,b, are coprime integers coprime to p, and b is positive.
				\end{proof}

			\item \quad  % 3.b 
				\begin{enumerate}
					\item  % 3.b.i
						\begin{proof}
							$x=0 \Rightarrow |x|_p = 0$ is true by definition.\\
							Assume $|x|_p = 0$, show that $x=0$:
							\begin{align*}
								x &= 0 \Rightarrow |x|_p=0&\\
								x &\neq 0 \Rightarrow |x|_p=p^{-k} \neq 0&\\
								|x|_p &= 0 \Rightarrow x = 0 &
							\end{align*}
							Hence shown $x=0 \Leftrightarrow |x|_p = 0$
						\end{proof}

					\item  % 3.b.ii
						\begin{proof} let $x = p^k\frac{a}{b}$, $y = p^l\frac{m}{n}$. Wants
							to show $|x|_p|y|_p = |xy|_p$
							\begin{align*}
								xy &= p^kp^l\frac{am}{bn}&
							\end{align*}
							Since $p^k$ is coprime with a and b, $p^l$ is coprime with m and
							n. This means p is not a factor of m, n, a, b. This means $p^kq^l$
							is coprime with am and bn. 
							\begin{align*}
								|xy|_p &= (p^kq^l)^{-1} = p^{-k}q^{-l}&\\
								|x|_p|y|_p &= p^{-k}p^{-l}&\\
								|xy|_p &= |x|_p|y|_p&\\
							\end{align*}
							If either x, y is 0, then both sides are 0, the case is trivially
							proved. For all other cases the above processed proved $|xy|_p =
							|x|_p|y|_p$.
						\end{proof}

					\item  % 3.b.iii
						\begin{proof}
							If either x, y = 0:
							$y = 0 \Rightarrow |x|_p=|x|_p$, $x = 0 \Rightarrow |y|_p=|y|_p$.
							This case is trivially proved.
							Let $x = p^k\frac{a}{b}$, $y = p^l\frac{m}{n}$, assume $k\ge l 
							\Rightarrow p^{-l} = max\{p^{-k}, p^{-l}\}$:
							\begin{align*}
								|x+y|_p &= \abs*{p^k\frac{a}{b} + p^l\frac{m}{n}}_p&\\
												&= \abs*{\frac{p^kan + p^lmb}{nb}}_p&\\
												&= \abs*{p^l\frac{p^{k-l}an + mb}{nb}}_p&\\
												&= \abs*{p^l}_p \cdot \abs*{\frac{p^{k-l}an +
													 mb}{nb}}_p&\\
											  &\le p^{-l} \cdot 1 = max\{p^{-k}, p^{-l}\}
							\end{align*}
							Hence shown $|x+y|_p \le max\{p^{-k}, p^{-l}\}$
						\end{proof}
					\end{enumerate}

				\item  % 3.c
					\begin{proof}
						In part (b), $|x+y|_p \le max\{p^{-k}, p^{-l}\}$ is shown to be
						true. $max\{p^{-k}, p^{-l}\} \le p^{-k}+p^{-l}$ is trivially true
						since neither $p^{-k}$ or $p^{-l}$ are negative. Hence the following
						ineuqality is true:
						$$|x+y|_p \le max\{p^{-k}, p^{-l}\}\le p^{-k}+p^{-l}$$
						Ergo $|x+y|_p \le p^{-k}+p^{-l}$ is true.
					\end{proof}

				\item  % 3.d
					\begin{proof}
						From part 3.b.iii, we know the following holds:\\
						Let $x = p^k\frac{a}{b}$, $y = p^l\frac{m}{n}$,
						$$|x+y|_p = \abs*{p^l}_p \cdot \abs*{\frac{p^{k-l}an + mb}{nb}}_p$$
						Assume $|x|_p \neq |x|_p$, then $k>l0 \Rightarrow p^{-l} =
						max\{p^{-k}, p^{-l}\}$ . This means that
						$\abs*{\frac{p^{k-l}an + mb}{nb}}_p$ is 1 because p is coprime to
						an, mb and nb. This means that $|x+y|_p = p^{-l} \cdot 1 =
						max\{p^{-k}, p^{-l}\}$.\\
						Assume $|x|_p = |x|_p$, $\abs*{\frac{p^{k-l}an + mb}{nb}}_p \le 1$ 
						because $p^{k-l} = p^0 = 1$. This term becomes $\abs*{\frac{an +
						mb}{nb}}_p$ which could contain $p^x$ in the numerator. The p-adic
						of this term could therefore be less than 1. This means that $p^{-l}
						= max\{p^{-k}, p^{-l}\}$ no longer holds when $|x|_p = |x|_p$.\\
						It is therefore shown that $|x+y|_p = max{|x|_p, |x|_p}$ whenever
						$|x|_p \neq |y|_p$
					\end{proof}

			\end{enumerate}
\end{enumerate}

\end{document}

