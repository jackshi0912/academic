\documentclass{article}
\usepackage{amsmath, amssymb, amsthm, mathtools}
\usepackage{enumitem}
\usepackage[margin=1in]{geometry}

\title{Combinatorics: Homework 3}
\author{Jack Shi - A92122910}
\date{February 2, 2018}

\begin{document}
\maketitle

\begin{enumerate} [label=\textbf{\arabic*}.]
	% Problem 1
	\item For each positive integer n, prove that there exists a $2^n\times 2^n$
		Hadamard matrix.
	\begin{proof} 
		Show $m\times m$ Hadamard matrices exist for all $m=2^n$.\\
		Let $X=\{x\in\mathbb{N} \mid x \le n\}$. Since $|X|=n$, $|P(X)|=2^n$.
		Let C be subsets of X indexed as $\{C_1,\ C_2,\ \cdots,C_{2^n}\}$.\\
		Let matrix $A=(a_{ij})$ defined as follows: 
			$$a_{ij}=(-1)^{|C_i\cap C_j|}$$
		To claim that A is a Hadamard matrix, $\langle r_i,r_j\rangle=0$ for
		$i\neq j$ must be true.
		From definition of A, the following holds:
			$$\langle r_i,r_j \rangle=\sum_k (-1)^{|C_i\cap C_k| + |C_j\cap C_k|}$$
		Since $C_i \neq C_j$, there exists an element $x \in X$ such that $x\in
		C_i \setminus C_j$ or $x\in C_j \setminus C_i$. Let $x\in C_i \setminus
		C_j$. Since x is a arbituary element in X, half of the subsets contain
		x, the other half does not. Let $C_x$ be all the subsets that contain x,
		$C_{\bar{x}}$ be all the subsets that does not contain x.\\
		Given $\langle r_i,r_j \rangle=\sum_k (-1)^{|C_i\cap C_k| + |C_j\cap C_k|}$,
		for half of the sum $C_k \in C_x$, for the other half $C_k \in C_{\bar{x}}$.
		These two halves have different parity because x is in half of the $C_k$ of
		the sum and not in $C_k$ for the other half. Since half of the sum has
		$(-1)^{odd}$, the other half have $(-1)^{even}$, this result in $(-1) + (1)$
		which is 0.\\

		Hence shown for any number n, there is a $m\times m$ Hadamard matrix such
		that $m = 2^n$
	\end{proof}

	% Problem 2
	\item 
		\begin{proof} Wants to show:
			$$min\{a_1, a_2, \dots a_n\} \le \frac{a_1 + a_2 + \dots + a_n}{n}
			\le max\{a_1, a_2, \dots a_n\}$$
			let $a_{min}$ be the minimum element in series $\{a_1, a_2, \dots a_n\}$,
			$a_{max}$ be the maximum element in series $\{a_1, a_2, \dots a_n\}$. 
			\begin{align*}
				a_{min} \cdot n &\le a_1 + a_2 + \dots + a_n \le a_{max} \cdot n &\\
				\frac{a_{min} \cdot n}{n} &\le \frac{a_1 + a_2 + \dots + a_n}{n} \le
				\frac{a_{max}\cdot n}{n} &\\
				a_{min} &\le \frac{a_1 + a_2 + \dots + a_n}{n} \le a_{max}& \\
				min\{a_1, a_2, \dots a_n\} &\le \frac{a_1 + a_2 + \dots + a_n}{n}
				\le max\{a_1, a_2, \dots a_n\} &
			\end{align*}
		\end{proof}

	% Problem 3
	\item  For each positive integer n, prove that there exists an $n \times n$
		matrix A with $\pm 1$ entries such that $|det A| \ge \sqrt{n!}$
		\begin{proof}
			There are $2^{n^2}$ matrices with $\pm 1$ entries. Let $D_n$ be the mean
			square average:
			$$D_n=\sqrt{\frac{\sum_A (det\ A)^2}{2^{n^2}}}$$
			$$\Rightarrow max_A\ det\ A \ge D_n$$
			This is true as we have shown in previous proof that average is in between
			min and max. By squaring both sides of D and subsituting the definition of
			determinant, the following is true:
			\begin{align*}
				D_n=\sqrt{\frac{\sum_A (det\ A)^2}{2^{n^2}}}
				{D_n}^2=\frac{1}{2^{n^2}} \sum_A \left(\sum_{\pi} (sign\
				\pi)a_{1\pi(1)}a_{2\pi(2)} \dots a_{n\pi(n)}\right)^2
			\end{align*}

			By expanding the squared summation, splitting $\pi$ into $\sigma$ and
			$\tau$, the following holds:
				$${D_n}^2=\frac{1}{2^{n^2}} \sum_A \left(\sum_\sigma \sum_\tau (sign\
				\sigma)(sign\ \tau) a_{1\sigma(1)}a_{1\tau(1)} \dots a_{n\sigma(n)}a_{n\tau(n)} \right)$$
			By applying interchage of summation, we have:
			$${D_n}^2=\frac{1}{2^{n^2}} \sum_{\sigma ,\ \tau}  (sign\ \sigma)(sign\
			\tau) \left(\sum_A a_{1\sigma(1)}a_{1\tau(1)} \dots a_{n\sigma(n)}a_{n\tau(n)} \right)$$

			Since $\sum_A X$ is equivalent as the following: 
			\begin{align*}
				&\sum_{a_{11}, a_{12}, \dots a_{nn}} X &\\ 
				&\sum_{a_{11}=\pm 1} \sum_{a_{12}=\pm 1} \dots
				\sum_{a_{nn}=\pm 1} X &\\
			\end{align*}
			Let $X = a_{1\sigma(1)}a_{1\tau(1)} \dots a_{n\sigma(n)}a_{n\tau(n)}$ we
			have the following:
				$$\sum_{a_{11}=\pm 1} \sum_{a_{12}=\pm 1} \dots
				\sum_{a_{nn}=\pm 1} a_{1\sigma(1)}a_{1\tau(1)} \dots a_{n\sigma(n)}a_{n\tau(n)} $$
			Suppose $\sigma (i) = k \neq \tau (i)$. Every sum contains $a_{ik}$.
			This implies that the entire sum has the factor $\sum_{a_{ik}=\pm 1}
			a_{ik} = 0$. If $\sigma = \tau$ the sum is no longer 0. The long product
			in this case converges to 1, $(sign\ \sigma)^2$ also converges to 1.
			Hence we have the following:
			\begin{align*}
				&\sum_{a_{11}=\pm 1} \dots \sum_{a_{nn}=\pm 1} 1 = 2^{n^2}&\\
				D_n^2 = \frac{1}{2^{n^2}} \sum_\sigma 2^{n^2} = n!
				D_n = \sqrt{n!}
			\end{align*}
			Hence shown there exists an $n \times n$ matrix A with entries $\pm 1$
			such that $|det\ A| \ge \sqrt{n!}$
		\end{proof}
\end{enumerate}

\end{document}
