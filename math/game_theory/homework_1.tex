

\documentclass[addpoints,answers]{exam}

\usepackage{blkarray}
\usepackage{amsmath}
\usepackage{amsthm}
\usepackage{amssymb}    
\usepackage{amsfonts}
\usepackage{tikz}
\usepackage{verbatim}

\sloppy

\usetikzlibrary{arrows,shapes}

\tikzstyle{vertex}=[circle,fill=black,minimum size=3pt,inner sep=0pt]
\tikzstyle{edge} = [draw,thick,-]

\checkboxchar{$\Box$}
\checkedchar{$\blacksquare$}
\CorrectChoiceEmphasis{}


\begin{document}

    \pagestyle{headandfoot}
    \runningheadrule
    \firstpageheader{Math 152}{Homework 1}{April 6, 2018}
    \runningheader{Math 152}{Homework 1, Page \thepage\ of \numpages}{April 6, 2018}

    \firstpagefooter{}{}{}
    \runningfooter{}{}{}
    \begin{flushright}
        \makebox[0.4\textwidth]{Name: Huize Shi}

        \vspace{0.2in}

        \makebox[0.4\textwidth]{Pid: A92122910}
    \end{flushright}

    \begin{questions}
        \question
            Alice and Bob play the following game.
            \begin{itemize}
                \item Initially, there are $20$ numbers: $10$ numbers $1$ and $10$ numbers
                    $2$.
                \item On each step one of the players select two numbers; and if they were
                    the same, replace them by $2$; otherwise, replace them by $1$.
                \item Alice make the first move and they do moves one after another.
            \end{itemize}
            Who is the winner?
            \begin{solutionorbox}[\stretch{1}]
							Alice wins. This is a parity split problem. Regardless of what move was
							chosen, the result is the total amount of numbers is subtracted by
							one. Since there are a total of 20 numbers, and a move is not
							possible when there is only 1 number left, the problem becomes
							subtracting 1 from 19 until 0 is reached. This means that the
							first person wins since 19 is odd.
            \end{solutionorbox}
            \newpage
 
        \question
            In the subtraction game where players may subtract $1$, $2$ or $5$ chips on
            their turn, identify the N and P positions.\\
            \begin{solutionorbox}[\stretch{1}]
							Let $c$ denote the number of chips in the pile.\\
							The positions follows the patter $PNN$ starting from terminal
							(left).
							\begin{proof} proof by induction:
								\paragraph{base case}
							\[
							\begin{bmatrix}
								c   & 0 & 1 & 2 \\
								pos & P & N & N \\
							\end{bmatrix}
							\]

							\paragraph{Induction step:}Assume given 
							\[
							\begin{bmatrix}
								c   & k & k+1 & k+2\\
								pos & P & N   & N  \\
							\end{bmatrix}
							\]
							By the definition of $P$ and $N$ states, the following holds:
							\[
							\begin{bmatrix}
								c   & k+6 & k+7 & k+8\\
								pos & P   & N   & N  \\
							\end{bmatrix}
							\]
							\end{proof}
            \end{solutionorbox}
            \newpage
 
        \question
            Is the Nim position $(1, 3, 5)$ an N-position (explain your answer)?
            \begin{solutionorbox}[\stretch{1}]
							$(1_{10}, 3_{10}, 5_{10}) = (01_{2}, 11_{2}, 101_{2})$, Since
							$01_{2} \oplus 11_{2} \oplus 101_{2} = 111 \neq 0$, the position
							$(1, 3, 5)$ is an N position.
            \end{solutionorbox}
            \newpage
 
        \question
            Consider the Mis\'ere subtraction game where players may subtract 1, 5 or 6
            chips on their turn, identify the N and P positions.
            \begin{solutionorbox}[\stretch{1}]
							The positions follows the patter $PNPNPNNNNNN$ starting from
							terminal position (left).
							\begin{proof} proof by induction:
								\paragraph{base case}
							\setcounter{MaxMatrixCols}{20}
							\[
							\begin{bmatrix}
								c   & 0 & 1 & 2 & 3 & 4 & 5 & 6 & 7 & 8 & 9 & 10 \\
								pos & P & N & P & N & P & N & N & N & N & N & N  \\
							\end{bmatrix}
							\]

							\paragraph{Induction step:}Assume given 
							\[
							\begin{bmatrix}
								c   & k & k+1 & k+2 & k+3 & k+4 & k+5 & k+6 & k+7 & k+8 & k+9 & k+10 \\
								pos & P & N & P & N & P & N & N & N & N & N & N  \\
							\end{bmatrix}
							\]

							By the definition of $P$ and $N$ states, and the step sizes, the following holds:
							\[
							\begin{bmatrix}
								c   & k+11 & k+12 & k+13 & k+14 & k+15 & k+16 & k+17 & k+18 &
								k+19 & k+20 & k+21 \\
								pos & P & N & P & N & P & N & N & N & N & N & N  \\
							\end{bmatrix}
							\]
							Hence prooven the pattern works.
						\end{proof}
            \end{solutionorbox}
            \newpage
 
\end{questions}
\end{document}
