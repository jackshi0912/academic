

\documentclass[addpoints,answers]{exam}

\usepackage{amsmath}
\usepackage{amssymb}    
\usepackage{amsfonts}
\usepackage{tikz}
\usepackage{verbatim}
\usepackage{skak}

\sloppy

\usetikzlibrary{arrows,shapes}

\tikzstyle{vertex}=[circle,fill=black,minimum size=3pt,inner sep=0pt]
\tikzstyle{edge} = [draw,thick,-]

\checkboxchar{$\Box$}
\checkedchar{$\blacksquare$}
\CorrectChoiceEmphasis{}


\begin{document}

    \pagestyle{headandfoot}
    \runningheadrule
    \firstpageheader{Math 152}{Homework 2}{April 20, 2018}
    \runningheader{Math 152}{Homework 2, Page \thepage\ of \numpages}{April 20, 2018}

    \firstpagefooter{}{}{}
    \runningfooter{}{}{}
    \begin{flushright}
        \makebox[0.4\textwidth]{Name:\enspace\hrulefill}

        \vspace{0.2in}

        \makebox[0.4\textwidth]{Pid:\enspace\hrulefill}
    \end{flushright}

    \begin{questions}
        \question
            Compute Grundy function for states of the subtraction game where players may
            subtract $1$, $2$ or $5$ chips on their turn.
            \begin{solutionorbox}[\stretch{1}]
							\begin{align*}
							g_G(0) &= 0\\
							g_G(1) &= 1\\
							g_G(2) &= 2\\
							g_G(3) &= 0\\
							g_G(4) &= 1\\
							g_G(5) &= 2\\
							g_G(6) &= 0\\
							g_G(7) &= 1\\
							g_G(8) &= 2\\
							\end{align*}
								Assume by strong induction, assume p position for $n = 0\ (mod\
								3)$, and n position for
								$n = 1\ (mod\ 3)$, $n = 2\ (mod\ 3)$. Check the following cases for k+1:
							\begin{align*}
								k+1 &= 0\ (mod\ 3) \text{ - k+1 is p position because k, k-1 are n positions}\\
								k+1 &= 1\ (mod\ 3) \text{ - k+1 is n position because k is p positions}\\
								k+1 &= 2\ (mod\ 3) \text{ - k+1 is n position because k-1, is p
								positions}\\
							\end{align*}
							Hence shown the pattern of the game is pnn.
            \end{solutionorbox}
            \newpage
 
        \question
            Compute Grundy function for the Nim position $(1, 3, 5)$.
            \begin{solutionorbox}[\stretch{1}]
							\begin{align*}
								g_G((1, 3, 5)) &= g_G(1)\oplus g_G(3) \oplus g_G(5)\\
															 &= 1 \oplus 3 \oplus 5
															 &= 7_{10}
							\end{align*}
							Since 7 is not 0, (1, 3, 5) is a n position in game of Nim.
            \end{solutionorbox}
            \newpage
 
        \question
            Consider the following game: Alice and Bob are writing from left to right digits
            of a $11$-digit number one by one. Alice wins if the number divides by $7$ and Bob
            wins otherwise (Alice writes the first the digit). Determine who is the winner.
            \begin{solutionorbox}[\stretch{1}]
            \end{solutionorbox}
            \newpage
 
\end{questions}
\end{document}
