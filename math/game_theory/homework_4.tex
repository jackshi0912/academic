\documentclass[addpoints,answers]{exam}

\usepackage{amsmath}
\usepackage{amssymb}    
\usepackage{enumerate}
\usepackage{amsfonts}
\usepackage{tikz}
\usepackage{verbatim}
\usepackage{skak}

\sloppy

\usetikzlibrary{arrows,shapes}

\tikzstyle{vertex}=[circle,fill=black,minimum size=3pt,inner sep=0pt]
\tikzstyle{edge} = [draw,thick,-]

\checkboxchar{$\Box$}
\checkedchar{$\blacksquare$}
\CorrectChoiceEmphasis{}


\begin{document}

    \pagestyle{headandfoot}
    \runningheadrule
    \firstpageheader{Math 152}{Homework 4}{May 4, 2018}
    \runningheader{Math 152}{Homework 4, Page \thepage\ of \numpages}{May 4, 2018}

    \firstpagefooter{}{}{}
    \runningfooter{}{}{}
    \begin{flushright}
        \makebox[0.4\textwidth]{Name: Huize Shi}

        \vspace{0.2in}

        \makebox[0.4\textwidth]{Pid: A92122910}
    \end{flushright}

    \begin{questions}
        \question
            Alice and Bob has several apples and bananas and they want to split these
            fruits. Both of them like both fruits, but value them differently. For Alice,
            $1$ apple is exactly equivalent to $2$ bananas. For Bob, $2$ apples are exactly
            equivalent to $1$ banana.
            
            Show that the way to split all the fruits is Pareto optimal if and only if
            \begin{itemize}
                \item either Alice has no bananas
                \item or Bob has no apples.
            \end{itemize}
            \begin{solutionorbox}[\stretch{1}]
							To prove that the following are Pareto optimal,
            \begin{itemize}
                \item either Alice has no bananas
                \item or Bob has no apples.
            \end{itemize}
						two conditions must be checked: 
						\begin{enumerate}
							\item $A(x^*, y^*) < A(x,y) \Rightarrow B(x^*, y^*) > B(x,y)$
							\item $B(x^*, y^*) < B(x,y) \Rightarrow A(x^*, y^*) > A(x,y)$
						\end{enumerate}
						Let function $A$ represent income for Alice, function $B$ represent
						income for Bob. $a$ stands for the number of apples. $b$ stands for
						the number of bananas.
						\begin{align*}
							A(x^*, y^*) &= 2a\\
							B(x^*, y^*) &= 2b\\
							A(x, y) &= 2(a-n) + k\\
							B(x, y) &= n + 2(b-k)
						\end{align*}
						In case 1 (Alice trade n apples for k bananas and Alice income rose):
						\begin{align*}
							A(x^*, y^*) &< A(x, y)\\
							2a < 2(a-n) + k &\Rightarrow k>2n\\
															&\Rightarrow 0 > 2n -k\\
															&\Rightarrow 2b > 2b + 2n -k\\
												 k>2n	&\Rightarrow 2b > 2b + 2n -k -k - n\\
														  &\Rightarrow 2b > 2(b-k) + n\\
							A(x^*, y^*) < A(x, y)&\Rightarrow B(x^*, y^*) > B(x,y)
						\end{align*}
						In case 2 (Alice trade n apples for k bananas and Bob income rose):
						\begin{align*}
							B(x^*, y^*) &< B(x, y)\\
							2b < 2(b-k) + n &\Rightarrow n>2k\\
															&\Rightarrow 0 > 2k -n\\
															&\Rightarrow 2a > 2a + 2k -n\\
												 n>2k	&\Rightarrow 2a > 2a + 2k -n -n - k\\
														  &\Rightarrow 2a > 2(a-n) + k\\
							B(x^*, y^*) < B(x, y)&\Rightarrow A(x^*, y^*) > A(x,y)
						\end{align*}
						As proven above, either conditions are satisfied for the given
						states to be Pareto optimal.
						
            \end{solutionorbox}
            \newpage
 
        \question
            Let us consider a modified game of Nim: on each turn a player may remove some
            number of pebbles from one pile or split this pile into two piles. Compute the
            Grudny function for this game for all the initial states with one pile.
            \begin{solutionorbox}[\stretch{1}]

							\begin{center}
							\begin{tabular}{c||c|c|c|c|c|c|c}
								x & 0 & 1 & 2 & 3 & 4 & 5 & 6\\
								\hline
								g(x) & 0 & 1 & $mex\{0, 1, 1\oplus 1=0 \}=2$ & $mex\{0, 1, 2,
								2\oplus 1=3 \}=4$ & 3 & 5 & 6
							\end{tabular}
							\end{center}
							The function swiches order on every 3rd and 4th counting from the
							$1^{st}$. $\forall x \in \mathbb{Z}^{\ge 0}$
							\begin{align*}
							g(0) &= 0\\
							g(4x+1) &= 4x+1\\
							g(4x+2) &= 4x+2\\
							g(4x+3) &= 4x+4\\
							g(4x+4) &= 4x+3
							\end{align*}
							\paragraph{Proof by induction:}
								\subparagraph{4x+1}A single pile would have Grundy function
								value from 0 to 4x. Two pile tuples $(4x, 1),
								(4x-1,2),\dots,(2x+1, 2x)$, have even (because the parity bit
								between the two piles are the same) Grundy values. This means
								that $g(4x+1) = 4x+1$.

								\subparagraph{4x+2}A single pile would have Grundy function
								value from 0 to 4x+1. Two pile tuples $(4x+1, 1),
								(4x,2),\dots,(2x+1, 2x+1)$, have odd (because the parity bit
								between the two piles are different), and divisible by 4 Grundy 
								values. This means that $g(4x+2) = 4x+2$.

								\subparagraph{4x+3}A single pile would have Grundy function
								value from 0 to 4x+2. Two pile tuples $(4x+2, 1),
								(4x+1,2),\dots,(2x+2, 2x+1)$, have odd (because the parity bit
								between the two piles are different), Grundy values. However,
								$g(4x+2,1)=4x+3$. This means that the next element in mex set
								would be $g(4x+3) = 4x+4$.

								\subparagraph{4x+4}A single pile would have Grundy function
								value from 0 to 4x+2 and by last part, 4x+4. 
								Two pile tuples $(4x+3, 1),
								(4x+2,2),\dots,(2x+2, 2x+2)$, have 1 mod 4, and even, Grundy
								values. This means that the next element in mex set
								would be $g(4x+4) = 4x+3$ since that is not taken.

            \end{solutionorbox}
            \newpage
 
\end{questions}
\end{document}
