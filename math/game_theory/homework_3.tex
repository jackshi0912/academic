\documentclass[addpoints,answers]{exam}

\usepackage{amsmath}
\usepackage{amssymb}    
\usepackage{amsfonts}
\usepackage{tikz}
\usepackage{verbatim}
\usepackage{skak}

\sloppy

\usetikzlibrary{arrows,shapes}

\tikzstyle{vertex}=[circle,fill=black,minimum size=3pt,inner sep=0pt]
\tikzstyle{edge} = [draw,thick,-]

\checkboxchar{$\Box$}
\checkedchar{$\blacksquare$}
\CorrectChoiceEmphasis{}


\begin{document}

    \pagestyle{headandfoot}
    \runningheadrule
    \firstpageheader{Math 152}{Homework 3}{April 27, 2018}
    \runningheader{Math 152}{Homework 3, Page \thepage\ of \numpages}{April 27, 2018}

    \firstpagefooter{}{}{}
    \runningfooter{}{}{}
    \begin{flushright}
        \makebox[0.4\textwidth]{Name:Huize Shi}

        \vspace{0.2in}

        \makebox[0.4\textwidth]{Pid:A92122910}
    \end{flushright}

    \begin{questions}
        \question[10]
            Two players have two boards $8 \times 8$ and $9 \times 9$, one by one they put
            rooks on these boards such that none of the rooks attack each other (on each
            turn a player can put a rook on only one board). Who is the
            winner in this game.
            \begin{solutionorbox}[\stretch{1}]
							The first player is always the winner of this game. The first
							player will take the center of the $9 \times 9$ board on the first
							turn. For any subsequent turns, the first player will place rooks
							symetric to the second player along either diagonal on the same
							board. Since the center is claimed on the $9\times 9$ board and
							$8\times 8$ board does not have a center point, this is possible
							for the first player. Since rook attack all horizontal and
							vertical squares, diagonal mirroring positions never attack
							eachother therefore if the second player has a move, the first
							player will always have a move.
            \end{solutionorbox}
            \newpage
 
        \question
            Compute the Grundy function for states of the subtraction game with two piles of
            chips where players may subtract $1$, $2$ or $5$ chips from one of the piles on
            their turn.
            \begin{solutionorbox}[\stretch{1}]
							\begin{center}
							\begin{tabular}{c||c|c|c|c|c|c|c}
								x & 0 & 1 & 2 & 3 & 4 & 5 & 6\\
								\hline
								g(x) & 0 & 1 & 2 & 0 & 1 & 2 & 0 
							\end{tabular}
							\end{center}
							Assume $g(x) = x\ mod\ 3$ for $x \le k$. Show that $g(k+1)$
							holds.
							\begin{center}
							\begin{tabular}{c||c|c|c}
								k+1 mod 3 & 0 & 1 & 2\\
								\hline
								mex\{g(x-1), g(x-2), g(x-5)\} & mex\{2, 1, 1\}=0 & mex\{2, 0,
								2\}=1 & mex\{1, 0, 0\}=2
							\end{tabular}
							\end{center}

							Hence shown $g(x) = x\ mod\ 3$ is the correct Grundy function for one
							subtraction game.\\
							By the Grundy theorem, the grundy function for two subtraction
							game would be the following: 
							\begin{equation*}
								g_{G_1, G_2}(x_1, x_2) = (x_1\ mod\ 3)\oplus (x_2\ mod\ 3)
							\end{equation*}
							\begin{center}
							\begin{tabular}{c|c c c}
								$x_2$ \slash $x_1$& 0 & 1 & 2\\
								\hline
								0 & 0 & 1 & 2\\
								1 & 1 & 0 & 3\\
								2 & 2 & 3 & 0\\
							\end{tabular}
							\end{center}

            \end{solutionorbox}
            \newpage
 
        \question
            Let $G_1$ be the subtraction game where on their turn a player may
            remove $1$ or $2$ coins, and where there are $10$ coins initially. Let $G_2$ be
            the game of Nim with three piles, of sizes $1$, $6$, $7$. List all winning moves
            in $G_1 + G_2$
            \begin{solutionorbox}[\stretch{1}]
							\begin{center}
							\begin{tabular}{c||c|c|c|c|c|c|c}
								x & 0 & 1 & 2 & 3 & 4 & 5 & 6\\
								\hline
								g(x) & 0 & 1 & 2 & 0 & 1 & 2 & 0 
							\end{tabular}
							\end{center}
							Assume $g_{G_1}(x) = x\ mod\ 3$ for $x \le k$. Show that $g(k+1)$
							holds.
							\begin{center}
							\begin{tabular}{c||c|c|c}
								k+1 mod 3 & 0 & 1 & 2\\
								\hline
								mex\{g(x-1), g(x-2)\} & mex\{2, 1\}=0 & mex\{2, 0,\}=1 & mex\{1, 0\}=2
							\end{tabular}
							\end{center}

							Hence shown $g_{G_1}(x) = x\ mod\ 3$ is the correct Grundy function for one
							subtraction game.\\\\

							Winning move would be one that move both games to p positions.
							This is because if both games are in p positions after the first
							move, the first player can win any game that the second player
							make a play on for every move. In this case, the game of Nim is
							already in p position since $1 \oplus 6 \oplus 7 = 0$. The players
							only winning move would be taking 1 from the subtraction game,
							since $g_{G_1}(9)=9\ mod\ 3=0$, the condition is met for p
							position for both games. \\\\
							Note: the converse strategy (making both games n-positions) does not work,
							since sending both games to n positions does not result in winning
							since an n position can go to another n position, but a p position
							can only go to n position. The second player can mave a move such
							that a n-position game remains in n-position.
            \end{solutionorbox}
            \newpage
 
\end{questions}
\end{document}
