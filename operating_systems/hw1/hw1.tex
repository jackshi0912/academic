\documentclass{article}
\usepackage{amsmath, amssymb, amsthm, mathtools}
\usepackage{enumitem}
\usepackage[margin=1in]{geometry}

\title{Operating Systems: Homework 1}
\author{Jack Shi - A92122910}

\begin{document}
\maketitle

\begin{enumerate} [label=\textbf{\arabic*}.]
	% Problem 1
	\item Missing mechanism:
		\paragraph{Without privileged instructions}Since manipulating protected
		control registers is a privileged instruction, without it, the user would be
		able to bypass the dual-mode operation and make their instructions kernel
		operations as they please. This can result in security issues especially if
		the user decide to manage I/O directly which is usually a system call
		through the kernel.
		\paragraph{Without memory protection}Programs would be able to access
		eachother's memory address space. This could allow malicious software to
		steal data or manipulate other programs by illegally access, change their
		memory data.
		\paragraph{Without timer interrupts}This would result in the inability to
		timeout process that take too long or stuck in an infinite loop.

	% Problem 2
	\item System without privileged mode:
		\paragraph{Possible to be secure:}Programs could be forced to run in VMs
		such as the JVM where the enviornment is controlled and the program's access
		is restricted.
		\paragraph{Not possible:}In case that the program must run on bare-metal for
		high efficiency or performance reasons, the lack of a privileged mode could
		result in kernel corruption.

	% Problem 3
	\item Which instructions should be privileged?
		\begin{enumerate} [label=\textbf{\alph*})]
			% 3a.
			\item Set value of timer \\
				\textbf{privileged:} Changing the timer could change the behavior of
				the interrupts which is a kernel level operation.
			% 3b.
			\item Read the clock \\
				\textbf{privileged:} System time can be used for timing attacks. The
				system clock is also managed by the kernel.
			% 3c.
			\item Clear memory\\
				\textbf{privileged:} Because of memory protection, processes should not
				be able to clear memory anywhere. However, within it's own allocated
				address space, it fine.
			% 3d.
			\item Turn off interrupts\\
				\textbf{privileged:} If any process is allowed to turn off interrupts,
				then time-out timer interrupt would not be able to prevent malicious
				infinite loops.
			%3 e.
			\item Switch from user to monitor(kernel) mode\\
				\textbf{privileged:} There is no point is separating modes if the user
				can just switch to kernel mode as they please.
		\end{enumerate}

	% Problem 4
	\item Condition of failure:\\
		\paragraph{open}\texttt{pathname} refer to block device, and that block
		device is being used by the system. \texttt{open} fails with error
		\texttt{EBUSY}.
		\paragraph{read} \texttt{file descriptor} does not refer to a opened file.
		\texttt{read} fails with error \texttt{EBADF}.
		\paragraph{fork} \texttt{fork} fails with error \texttt{ENOMEM} when there
		is not sufficient memory for another process.
		\paragraph{exec} If the header of the file for \texttt{exec} is not
		recognized, \texttt{exec} will result in error \texttt{ENOEXEC}.
		\paragraph{unlink} If the \texttt{pathname} points to somewhere outside of
		the alloted address sapce, \texttt{unlink} result in error \texttt{EFAULT}.

	% Problem 5
\item Challenges when copying parameters from user to kernel modes:\\
	\textbf{Challenge 1:} The program in user mode does not have the privilege to
	build a function stack in kernel mode and copy over the parameters.\\
	\textbf{Challenge 2:} If the parameters contains pointerrs, dereferencing will
	require violating the user/kernel mode boundaries as well.\\
	\textbf{Solution: } Temporary grant the caller kernel privileges.

	% Problem 6
	\item Use hardware interval timer to keep track of the time of day:\\
		\textbf{Solution:} Since time of day is usually descretized to seconds, the
		OS can set 1 second long interval timers that update the system time on
		1Hz interrupt.
	% Problem 7
	\item Divise subsitute for traps using interrupts and/or exceptions:\\
		\textbf{Solution:} It is possible if the user get kernel privilege and
		implement it's own version of the interrupt handler.
	% Problem 8
	\item Consider C program:
		\begin{enumerate} [label=\textbf{\alph*}.]
			% 8 a.
			\item 8 forks total are created.
			% 8 a.
			\item The statement is executed twice.
		\end{enumerate}
\end{enumerate}

\end{document}

